\section{SER small-step semantics}
\label{appendix:ser-semantics}

%
%\subsection{Semantics}
%
The set \(V\) is a finite set of numeric constants; booleans use $0/1$. We respectively denote with \(VARS\) and \(vars\) the (finite) sets of global and local variables. $\rho:{vars}\to V$, and $g:{VARS}\to V$ respectively map a local or global variable to its current value in \(V\).
%
Configurations of both programs and expression are denoted as $\cfg{e}{\rho}{g}$, with \(e\) being a valid \toolname{} expression/program. 
%
Expression steps are denoted $(\step)$, while program steps are denoted $(\pstep)$, and may comprise of a sequence of expression steps.
%
%program configurations $\langle g , \Pi\rangle$ with \(g\) being a global state and \(\Pi\) being set of in-flight \((e,\ell)\) pairs.
%
%We denote with \(\ell_0\) the initial local state of every packet.

\smallskip
\noindent\textit{Expression step} $(\step)$.
\begin{mathpar}
	\inferrule*[right=ND-0]{ }{\cfg{\nondet}{\rho}{g} \step \cfg{0}{\rho}{g}}
	\and
	\inferrule*[right=ND-1]{ }{\cfg{\nondet}{\rho}{g} \step \cfg{1}{\rho}{g}}\\
	
	\inferrule*[right=LOCAL-READ]{\rho(x)=v}{\cfg{x}{\rho}{g} \step \cfg{v}{\rho}{g}}
	\and
	\inferrule*[right=GLOBAL-READ]{g(X)=v}{\cfg{X}{\rho}{g} \step \cfg{v}{\rho}{g}}\\
	
	\inferrule*[right=LOCAL-ASSIGN-STEP]{\cfg{e}{\rho}{g} \step \cfg{e'}{\rho'}{g'}}{\cfg{x := e}{\rho}{g} \step \cfg{x := e'}{\rho'}{g'}}
	\and
	\inferrule*[right=LOCAL-ASSIGN-DONE]{ }{\cfg{x := v}{\rho}{g} \step \cfg{v}{\update{\rho}{x}{v}}{g}}
	
	\inferrule*[right=GLOBAL-ASSIGN-STEP]{\cfg{e}{\rho}{g} \step \cfg{e'}{\rho'}{g'}}{\cfg{X := e}{\rho}{g} \step \cfg{X := e'}{\rho'}{g'}}
	\and
	\inferrule*[right=GLOBAL-ASSIGN-DONE]{ }{\cfg{X := v}{\rho}{g} \step \cfg{v}{\rho}{\update{g}{X}{v}}}
	
	\inferrule*[right=EQ-L]{\cfg{e_1}{\rho}{g} \step \cfg{e_1'}{\rho'}{g'}}{\cfg{e_1 == e_2}{\rho}{g} \step \cfg{e_1' == e_2}{\rho'}{g'}}
	\and
	\inferrule*[right=EQ-R]{\cfg{e_2}{\rho}{g} \step \cfg{e_2'}{\rho'}{g'}}{\cfg{v_1 == e_2}{\rho}{g} \step \cfg{v_1 == e_2'}{\rho'}{g'}}
	\and
	\inferrule*[right=EQ-T]{v_1=v_2}{\cfg{v_1 == v_2}{\rho}{g} \step \cfg{1}{\rho}{g}}
	\and
	\inferrule*[right=EQ-F]{v_1\neq v_2}{\cfg{v_1 == v_2}{\rho}{g} \step \cfg{0}{\rho}{g}}
	
	\inferrule*[right=SEQ-STEP]{\cfg{e_1}{\rho}{g} \step \cfg{e_1'}{\rho'}{g'}}{\cfg{e_1 ; e_2}{\rho}{g} \step \cfg{e_1' ; e_2}{\rho'}{g'}}
	\and
	\inferrule*[right=SEQ-DONE]{ }{\cfg{v ; e_2}{\rho}{g} \step \cfg{e_2}{\rho}{g}}
	
	\inferrule*[right=IF-GUARD]{\cfg{e_1}{\rho}{g} \step \cfg{e_1'}{\rho'}{g'}}{\cfg{\ifkw(e_1)\{e_2\}\elsekw\{e_3\}}{\rho}{g} \step \cfg{\ifkw(e_1')\{e_2\}\elsekw\{e_3\}}{\rho'}{g'}}
	\and
	\inferrule*[right=IF-T]{ }{\cfg{\ifkw(1)\{e_2\}\elsekw\{e_3\}}{\rho}{g} \step \cfg{e_2}{\rho}{g}}
	\and
	\inferrule*[right=IF-F]{ }{\cfg{\ifkw(0)\{e_2\}\elsekw\{e_3\}}{\rho}{g} \step \cfg{e_3}{\rho}{g}}
	
	\inferrule*[right=WHILE-UNFOLD]{ }{\cfg{\whilekw(e_1)\{e_2\}}{\rho}{g}
		\step
		\cfg{\ifkw(e_1)\{\,e_2 ; \whilekw(e_1)\{e_2\}\,\}\elsekw\{0\}}{\rho}{g}}
\end{mathpar}

\noindent\textit{Program step} $(\pstep)$ and scheduling.
\begin{mathpar}
	\inferrule*[right=YIELD]{\cfg{e}{\rho}{g} \step \cfg{\yieldkw \cdot e'}{\rho'}{g'}}{\cfg{e}{\rho}{g} \pstep \cfg{e'}{\rho'}{g'}}
	\\
	\inferrule*[right=START]{\cfg{e_i}{\rho}{g}\pstep \cfg{e'_i}{\rho'}{g'} \quad i\in\{1,\dots,n\}}{\langle \requestkw\ name_i\{e_i\}, \rho, g \rangle \pstep  \cfg{e'_i}{\rho'}{g'} }
	\and
	\inferrule*[right=TERMINATE]{\cfg{e}{\rho}{g} \step \cfg{v}{\rho'}{g'} \quad v\in{V}}{\cfg{e}{\rho}{g} \pstep \cfg{v}{\rho'}{g'}}	
\end{mathpar}



\smallskip
\noindent
\textbf{Note.}
Instead of defining a \texttt{spawn} instruction, as exists in some languages --- \toolname{} captures \textit{external} spawning via requests.
%
This setting can equivalently capture self-spawning (by using additional global variables), while translating more naturally to the networking domain --- in which threads are captured by packets sent by an external user.