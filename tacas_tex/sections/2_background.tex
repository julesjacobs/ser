\section{Background}
\label{sec:background}

%\medskip
%\noindent
\textbf{Petri Nets.}
A \emph{Petri net} is a tuple
\(
N = (P, T, \mathsf{pre}, \mathsf{post}, M_0),
\)
where \(P\) is a finite set of \emph{places}, \(T\) is a finite set of \emph{transitions},
\(\mathsf{pre},\mathsf{post}:T\to\mathbb N^P\) are the input/output vectors, and
\(M_0\in\mathbb N^P\) is the initial marking (token distribution). We use the terms \textit{state} and \textit{marking} interchangeably.

\noindent We say that a transition \(t\in T\) is \emph{enabled} at a marking \(M\in\mathbb N^P\) iff
\(
M \;\ge\; \mathsf{pre}(t)
\quad(\text{coordinate‐wise}).
\)
i.e., $M$ provides at least as many tokens as required by $\mathsf{pre}(t)$.
When $t$ is enabled at $M$, firing $t$ produces a new marking $M'$ defined as
\(
M' \;=\; M \;-\;\mathsf{pre}(t)\;+\;\mathsf{post}(t),
\)
which consumes input tokens and produces output tokens. We write $M \xrightarrow{t} M'$ when firing $t$ transforms $M$ to $M'$.
%
A marking \(M'\) is \emph{reachable} from $M$ via sequence $\sigma = t_1 \ldots t_k$ if there exist markings $M_0, \ldots, M_k$ where $M = M_0$, $M' = M_k$, and $M_i \xrightarrow{t_{i+1}} M_{i+1}$ for all $i$. We write $M \xrightarrow{\sigma} M'$.
%%
The set $R(N)$ contains all reachable markings:
\(
R(N) = \{M \mid \exists \sigma \in T^* .\ M_0 \;\xrightarrow{\sigma}\!^*\; M\}
\)
%
The \emph{reachability problem} asks, given a Petri net $N$ and a marking $M$, whether $M$ is reachable from $M_0$, the initial marking of $N$.  Surprisingly, even in the
\emph{unbounded} setting (where places may hold arbitrarily many tokens) this problem is
decidable~\cite{Ma81,Ko82,La92}, although with
Ackermann-complete complexity~\cite{CzWo22,Le22}.
%
An example of a toy Petri Net, and both reachable and unreachable markings, appears in Appendix~\ref{appendix:toyPN}.
%
%We focus on the verification of \textit{reachability properties}, meaning properties on the states that a net $N$ can reach. 
We check whether a marking satisfying formula $F$ (a combination of linear constraints) is reachable. The property is \textit{reachable} (\sat) if some $M \in R(N)$ satisfies $F$ (denoted $M \models F$), and \textit{unreachable} (\unsat) otherwise. 

\medskip
\noindent
\textbf{Verdict proofs.} 
%
If a property $F$ is reachable in $N$, then a witness sequence $\sigma \in T^*$ and marking $M$ such that $ M_0 \;\xrightarrow{\sigma}\; M \text{ and }M \models F$
%\[
%M_0 \;\xrightarrow{\sigma}\; M \text{ and }M \models F
%\] 
serves as a proof of reachability. One can verify the claim by simulating $\sigma$ from $M_0$ and checking that the resulting marking satisfies $F$.
%
If $F$ is unreachable, there exists~\cite{Le09} an inductive certificate $C$ (a Presburger formula) satisfying:
(i) $M_0 \models C$ (initial state satisfies $C$);
(ii) if $M \models C$ and $M \xrightarrow{t} M'$, then $M' \models C$ (inductiveness); and
(iii) $C \Rightarrow \neg F$ (certificate implies property is false).

%
%\begin{enumerate}
%	\item $M_0 \models F$,
%	\item For all transitions $M \xrightarrow{t} M'$, if $M \models C$ then $M' \models C$,
%\item $C \Rightarrow \neg F$, i.e., all markings satisfying $C$ falsify $F$.
%\end{enumerate} 

\medskip
\noindent
\textbf{Semilinear sets and Parikh’s theorem.}
A set \(S\subseteq\mathbb N^k\) is \emph{semilinear} iff
\[
S \;=\; \bigcup_{i=1}^m \Bigl\{\mathbf b_i + \sum_{j=1}^{r_i} n_j\,\mathbf p_{i,j}
\;\Big|\; n_j\in\mathbb N\Bigr\}.
\]

for $b_i, p_{i,j}\in \mathbb N^k$ being k-dimensional vectors of non-negative values.
%
Semilinear sets coincide exactly with the sets definable in \textit{Presburger arithmetic}~\cite{Pr29}.
%
By Parikh's theorem~\cite{Parikh66}, the \textit{Parikh Image} of any context-free language is semilinear, and there is an effective construction mapping each word in the language to a
finite description of its Parikh Image.

\medskip
\noindent
\textbf{Deciding serializability in unbounded systems.}
%
We build on Bouajjani et al.~\cite{BoEmEnHa13}, who proved that serializability is decidable for unbounded systems by reducing it to Petri net reachability. They showed this as a special case of \textit{bounded-barrier linearizability}.