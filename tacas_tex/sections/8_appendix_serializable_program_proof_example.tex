
\section{Example: Serializable Program}
\label{appendix:ns-serializable}


Now, we observe again the adjusted program with a spin-lock (as previously described in Listing~\ref{lst:MotivatingExample3Ser}), of which we depicted figures of the corresponding Network System (Fig.~\ref{fig:code3ExampleNS}), Serializability NFA (Fig.~\ref{fig:code3ExampleNFA}), and the interleaving Petri net (Fig.~\ref{fig:code3ExamplePN}) in Appendix~\ref{appendix:MoreNsExamples}.
%
In this case, serializability corresponds to the Petri net being unable to reach a marking satisfying the same semilinear formula \(\mathcal {F}\) as in the non-serializable case described in the main text (subsec.~\ref{subsec:SerToNsTranslation}):

\[
\mathcal {F}:
\quad
P_1 = 0 \wedge 
\textcolor{blue}{P_2} \ge 0 \wedge \textcolor{blue}{P_3} \ge 0  \wedge P_4 = 0
\wedge P_5 = 0 \wedge P_6 = 0 \wedge \textcolor{red}{P_7} \ge 0 \wedge \textcolor{red}{P_8} \ge 1.
\]

% (note however, that this is not always the case). 
%(but this time each place $P_i$ corresponds to the new PN). 
%following formula:
%
%\[
%\textcolor{blue}
%P_1 = 0 \wedge 
%{P_2} \ge 0 \wedge \textcolor{blue}{P_3} \ge 0  \wedge P_4 = 0
%\wedge P_5 = 0 \wedge P_6 = 0 \wedge \textcolor{red}{P_7} = 0 \wedge \textcolor{red}{P_8} \ge 1.
%\]


In addition, although the target set is the same as in the previous example, the Petri net places $(P_1,\ldots,P_8)$ encode different states that correspond to the updated program. For instance, now each place in the PN that encodes a global state accounts for two global variables, \texttt{X} and \texttt{L}, and the initial global state corresponds to the place encoding the initial assignment \textcolor{blue}{[X=0, L=0]}, etc.
%
Furthermore, unlike the case in Listing~\ref{lst:MotivatingExample2NonSer} (covered in subsec.~\ref{subsec:SerToNsTranslation}), this target set of markings (encoding request/response pairs of non-serial executions) is \textit{unreachable}, as witnessed by the inductive invariant:


\[
\begin{aligned}
	&(P_{1},\textcolor{blue}{P_{2}},\textcolor{blue}{P_{3}},P_{4},P_{5},P_{6},\textcolor{red}{P_{7}},\textcolor{red}{P_{8}})
	\;\mapsto\;\\
	&\quad
	\exists\,e_{0},\dots,e_{5}\ge0.\;
	\Bigl(
	e_{2}-e_{1}+\textcolor{blue}{P_{3}}-1=0\;\land\;
	e_{2}+P_{1}-e_{5}=0\;\land\;
	P_{5}-e_{1}+e_{4}=0\;\land\\
	&\qquad\quad
	-\,e_{4}+\textcolor{red}{P_{7}}=0\;\land\;
	P_{6}+e_{3}-e_{0}=0\;\land\;
	\textcolor{red}{P_{8}}-e_{3}=0\;\land\\
	&\qquad\quad
	-\,e_{2}+e_{1}+e_{0}+P_{4}=0\;\land\;
	-\,e_{2}+e_{1}+\textcolor{blue}{P_{2}}=0
	\Bigr)
	\;\land\;
	\bigl(P_{4}-1\ge0\;\lor\;\textcolor{blue}{P_{3}}-1\ge0\bigr).
\end{aligned}
\]


We then revert and project it on the request/response pairs of the Network System.
%
We get the following inductive invariants for each of the two (reachable) global states:

\begin{proof}
	
	\medskip\noindent
	For global state \textcolor{blue}{[L=0,X=0]} the projected invariant is:
	\[
	\bigl(\,\text{\color{ForestGreen}$\blacklozenge_{\text{main}}$}/\text{\color{red}$\blacklozenge_{0}$},\;
	\text{\color{ForestGreen}$\blacklozenge_{\text{main}}$}/\text{\color{red}$\blacklozenge_{1}$}\bigr)
	\;\mapsto\;
	\exists\,e_{0},\dots,e_{5}\ge0.\;
	\begin{aligned}[t]
		& e_{2}-e_{1}=0,\quad
		e_{2}-e_{5}=0,\quad
		-e_{1}+e_{4}=0,\\
		& -e_{4}+\bigl(\text{\color{ForestGreen}$\blacklozenge_{\text{main}}$}/\text{\color{red}$\blacklozenge_{1}$}\bigr)=0,\quad
		-e_{0}+e_{3}=0,\\
		& -e_{3}+\bigl(\text{\color{ForestGreen}$\blacklozenge_{\text{main}}$}/\text{\color{red}$\blacklozenge_{0}$}\bigr)=0,\quad
		-e_{2}+e_{1}+e_{0}=0,\\
		& -e_{2}+e_{1}=0.
	\end{aligned}
	\]
	\noindent From 
	\[e_{1}=e_{2}=e_{4}=e_{5}=(\;
	\text{\color{ForestGreen}$\blacklozenge_{\text{main}}$}/\text{\color{red}$\blacklozenge_{1}$}),\;
	e_{0}=e_{3}=
	(\text{\color{ForestGreen}$\blacklozenge_{\text{main}}$}/\text{\color{red}$\blacklozenge_{0}$})
	\]
	
	it follows that \[-e_{2}+e_{1}+e_{0}=0\;\Longrightarrow\;e_{0}=0,\]
	
	thus: 
	\[
	(	\text{\color{ForestGreen}$\blacklozenge_{\text{main}}$}/\text{\color{red}$\blacklozenge_{0}$})
	=0 \quad \text{and} \quad (	\text{\color{ForestGreen}$\blacklozenge_{\text{main}}$}/\text{\color{red}$\blacklozenge_{1}$})
	=0
	\]
	
	indicating that  (\(\text{\color{ForestGreen}$\blacklozenge_{\text{main}}$}/\text{\color{red}$\blacklozenge_{0}$}\)) cannot be obtained from the global state
	\textcolor{blue}{[L=0,X=0]}.
	
	\medskip\noindent
	In the second case, for the global state \textcolor{blue}{[L=1, X=1]}
	the projected invariant is:
	
	
	\[
	\bigl(\,\text{\color{ForestGreen}$\blacklozenge_{\text{main}}$}/\text{\color{red}$\blacklozenge_{0}$},\;
	\text{\color{ForestGreen}$\blacklozenge_{\text{main}}$}/\text{\color{red}$\blacklozenge_{1}$}\bigr)
	\;\mapsto\;
	\exists\,e_{0},\dots,e_{5}.\;\bot,
	\]
	which is unsatisfiable. Hence, no completed request/response pair, and in particular no (\(\text{\color{ForestGreen}$\blacklozenge_{\text{main}}$}/\text{\color{red}$\blacklozenge_{0}$}\)) can be produced from this state via \textit{any} execution. Intuitively, this aligns with the fact that there cannot be any output generated via an interleaving, given that the spin-lock is acquired (\textcolor{blue}{[L=1]}).
	%	\guy{Mark/Jules is this part correct?}
	
	\medskip
	\noindent\textbf{Conclusion.}
	In every reachable state, no request/response pair of the form
	($	\text{\color{ForestGreen}$\blacklozenge_{\text{main}}$}/\text{\color{red}$\blacklozenge_{0}$})
	$
	can occur. Consequently, the only possible pairs are
	($	\text{\color{ForestGreen}$\blacklozenge_{\text{main}}$}/\text{\color{red}$\blacklozenge_{1}$})
	$,
	all of which lie within the NFA’s language for serial executions.
	Hence, the program is serializable. Moreover, as proven in subsection~\ref{appendix:subsec:InductiveInvariantExample},
	these invariants are inductive: they hold in the initial state and are preserved under every transition.
\end{proof}

%	\medskip\noindent
%	\textbf{Conclusion.}
%	In all reachable states
%	it holds that there cannot be any request/response pair of type
%	(	$	\text{\color{ForestGreen}$\blacklozenge_{\text{main}}$}/\text{\color{red}$\blacklozenge_{0}$}
%	$).
%	%
%	Furthermore, this indicates that the only attainable request/response pairs are of the form 	($	\text{\color{ForestGreen}$\blacklozenge_{\text{main}}$}/\text{\color{red}$\blacklozenge_{1}$})
%	$, which are included in the language of of NFA for serial executions. Thus, this program is serializable.
%	%
%	We further show (in Appendix~\ref{appendix:InductiveInvariantExample}) that these invariants are \textit{inductive}: they encompass the system’s initial state and, once satisfied, remain true for all subsequent executions.




%\newpage


%\subsection{Time/Space Complexity}
%
%\guy{Should we add something about time/space complexity?}

%\newpage

\subsection{Proof of Inductive Invariant}
\label{appendix:subsec:InductiveInvariantExample}


\begin{proof}
	
	Define the predicate
	\[
	\begin{aligned}
		I(P_{1},\dots,\textcolor{red}{P_{8}})
		:={}&
		(P_{1},\textcolor{blue}{P_{2}},\textcolor{blue}{P_{3}},P_{4},P_{5},P_{6},\textcolor{red}{P_{7}},\textcolor{red}{P_{8}})
		\;\mapsto\;\\
		&\quad
		\exists\,e_{0},\dots,e_{5}\ge0.\;
		\Bigl(
		e_{2}-e_{1}+\textcolor{blue}{P_{3}}-1=0\;\land\;
		e_{2}+P_{1}-e_{5}=0\;\land\;
		P_{5}-e_{1}+e_{4}=0\;\land\\
		&\qquad\quad
		-\,e_{4}+\textcolor{red}{P_{7}}=0\;\land\;
		P_{6}+e_{3}-e_{0}=0\;\land\;
		\textcolor{red}{P_{8}}-e_{3}=0\;\land\\
		&\qquad\quad
		-\,e_{2}+e_{1}+e_{0}+P_{4}=0\;\land\;
		-\,e_{2}+e_{1}+\textcolor{blue}{P_{2}}=0
		\Bigr)
		\;\land\;
		\bigl(P_{4}-1\ge0\;\lor\;\textcolor{blue}{P_{3}}-1\ge0\bigr).
	\end{aligned}
	\]
	
	
	\medskip\noindent
	\textbf{(1) Initialization.}
	The initial marking has $\textcolor{blue}{P_{3}}=1$ and $P_{1}=\textcolor{blue}{P_{2}}=P_{4}=P_{5}=P_{6}=\textcolor{red}{P_{7}}=\textcolor{red}{P_{8}}=0$.
	Choose $e_{0}=\cdots=e_{5}=0$.  Then
	\[
	e_{i}\ge0,\quad
	e_{2}-e_{1}+\textcolor{blue}{P_{3}}-1=0-0+1-1=0,\;\dots,\;-e_{2}+e_{1}+P_{2}=0,
	\]
	and 
	\[
	P_{4}-1\ge0\;\lor\;\textcolor{blue}{P_{3}}-1\ge0
	\;=\;-1\ge0\;\lor\;0\ge0
	\;=\;\texttt{FALSE}\;\lor\;\texttt{TRUE}
	\;=\;\texttt{TRUE}.
	\]
	Thus $I$ holds initially.
	
	\medskip\noindent
	\textbf{(2) Consecution.}
	One checks for each transition $t_{k}$ of the Petri net that
	\[
	I(M)\;\Longrightarrow\;I\bigl(t_{k}(M)\bigr).
	\]
	In each case, the same $(e_{0},\dots,e_{5})$ can be adjusted (per the \texttt{SMT} certificate) to show that the eight equalities and the disjunction remain valid. See our accompanying artifact~\cite{ArtifactRepository} for generating a full proof in the standard \texttt{SMT-LIB} format~\cite{BaStTi10}.
	
	\medskip\noindent
	\textbf{(3) Refutation of the property.}
	Suppose by contradiction that there exists a marking $P$ for which both $I(P)$ and $\mathcal {F}(P)$ hold:
	\[
	\mathcal {F}(P):\quad
	P_{1}=0,\;
	\textcolor{blue}{P_{2}}\ge0,\;
	\textcolor{blue}{P_{3}}\ge0,\;
	P_{4}=0,\;
	P_{5}=0,\;
	P_{6}=0,\;
	\textcolor{red}{P_{7}}\ge0,\;
	\textcolor{red}{P_{8}}\ge1.
	\] 
	
	\noindent
	From
	\[
	e_{2}-e_{1}+\textcolor{blue}{P_{3}}-1=0
	\quad\text{and}\quad
	-e_{2}+e_{1}+\textcolor{blue}{P_{2}}=0
	\]
	we get
	\[
	\textcolor{blue}{P_{2}}=1-\textcolor{blue}{P_{3}}.
	\]
	From
	\[
	\textcolor{red}{P_{8}}-e_{3}=0
	\quad\text{and}\quad
	P_{6}+e_{3}-e_{0}=0
	\]
	and from the assumption that $P_6=0$, we get
	\[
	e_{0}=e_{3}=\textcolor{red}{P_{8}}.
	\]
	
	
	\noindent
	Similarly, the invariant equalities 
	$(-\,e_{2}+e_{1}+e_{0}+P_{4}=0)$ and $(	-\,e_{2}+e_{1}+\textcolor{blue}{P_{2}}=0)$
	induce
	\[
	\textcolor{blue}{P_{2}}=P_{4}+e_{0}=P_{4}+\textcolor{red}{P_{8}},
	\]
	thus, and as we also assume that $P_4=0$, then:
	\[
	\textcolor{red}{P_{8}}=\textcolor{blue}{P_2}-P4=(1-\textcolor{blue}{P_{3}})-P_{4}=1-\textcolor{blue}{P_{3}}-0=1-\textcolor{blue}{P_{3}}
	\]
	
	
	
	
%	\medskip
	\noindent
	However, $\mathcal {F}(P)$ also induces $\textcolor{blue}{P_{3}}\ge0$ and $\textcolor{red}{P_{8}}\ge1$, and hence $\textcolor{blue}{P_{3}}=0$.  
	Furthermore, as our invariant includes a conjunction with $\bigl(P_{4}-1\ge0\;\lor\;\textcolor{blue}{P_{3}}-1\ge0\bigr)$, then it necessarily holds that \(P_{4}\ge1\). This contradicts \(P_{4}=0\) as required for the semilinear set to be reachable.
	%
	Thus, $I\land\mathcal {F}$ is unsatisfiable, i.e., 
	%\[
	$
	I(P)\;\Longrightarrow\;\neg\mathcal {F}(P)$.
	%\]
	This completes the proof that $I$ is an inductive invariant refuting property $\mathcal {F}$.
\end{proof}


%\newpage