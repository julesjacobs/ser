%% Set to false for submission!
\newif\iftodo\todofalse

\documentclass[noacm,acmsmall,anonymous]{acmart}
%\documentclass{article}
\settopmatter{printfolios=false,printccs=false,printacmref=false}
\raggedbottom
\renewcommand\footnotetextcopyrightpermission[1]{} % removes footnote with DOI
\renewcommand{\keywords}[1]{}  % removes "Additional keywords and phrases"
\renewcommand{\acmSubmissionID}[1]{} % removes SUBMISSION ID
\let\oldmaketitle\maketitle
\renewcommand{\maketitle}{
	\oldmaketitle
	\pagestyle{plain}  % empty headers and footers on all pages
	\thispagestyle{plain}  % empty header and footer on the first page
}

\usepackage{array}
\usepackage{graphicx}
\usepackage{multirow}
\usepackage{colortbl}
\usepackage{float}
\usepackage{fancybox}
\usepackage{tcolorbox}
\usepackage{mathtools}
\usepackage{amsmath}
\usepackage{amsthm}
\usepackage{arydshln} % Package for dashed lines
% Disable \Bbbk redefinition if it exists
\let\Bbbk\relax
\usepackage{amssymb}  % for loaner compilation
\usepackage{mathpartir}
\usepackage{geometry}
\usepackage{hyperref}
\usepackage{xcolor}
\usepackage{tikz}
\usepackage{ulem}
\usepackage{cancel} % For middle strikethrough
\usepackage{stmaryrd}
\usepackage{graphicx}    % To include PDF files
\usepackage{subfig}  % To create subfigures
\usepackage{listings}
\usetikzlibrary{petri, positioning}


\newcommand{\jj}[1]{\textcolor{orange}{\textbf{Jules:} #1}}

%\definecolor{ForestGreen}{rgb}{0.13, 0.55, 0.13} % Optional manual definition
%\definecolor{forestgreen}{RGB}{0, 128, 0} produces a darker green.


\definecolor{lightgray}{rgb}{0.9,0.9,0.9}


\geometry{a4paper, margin=1in}

\title{Serializability in Programmable Networking Services}
\author{}
\date{\today}

\definecolor{ForestGreen}{RGB}{34, 139, 34}

% Define the \INITNETWORK command
\newcommand{\INITNETWORK}{\textcolor{ForestGreen}{\textbf{\textsc{INITIALIZE\_NETWORK}}}}
\newcommand{\TERMINATE}{\textcolor{ForestGreen}{\textbf{\textsc{TERMINATE}}}}
\newcommand{\FIRING}{\textcolor{ForestGreen}{\textbf{\textsc{FIRING}}}}
\newcommand{\TOKENPi}{\textcolor{ForestGreen}{\textbf{\textsc{TOKEN}}\_$p_i${}}}
\newcommand{\ADDTOKENPi}{\textcolor{ForestGreen}{\textbf{\textsc{ADD\_TOKEN\_}}$p_i${}}}
\newcommand{\REMOVETOKENPi}{\textcolor{ForestGreen}{\textbf{\textsc{REMOVE\_TOKEN\_}}$p_i{}$}}
\newcommand{\REMAININGINITIALTOKENPi}{\textcolor{ForestGreen}{\textbf{\textsc{REMAINING\_INITIAL\_TOKEN\_}}$p_i${}}}


\newcommand{\SPAWN}{\textcolor{blue}{\texttt{SPAWN}}}
\newcommand{\DONE}{\textcolor{blue}{\texttt{DONE}}}
\newcommand{\DROP}{\textcolor{red}{\texttt{DROP}}}
\newcommand{\observation}{Observation}
\newcommand{\HB}{\sqsubset}
\newcommand{\heapAfterTick}{\mathsf{tick\_clock}}
\newcommand{\sendEvent}{\mathsf{SEND}}
\newcommand{\receiveEvent}{\mathsf{RECEIVE}}
\newcommand{\readEvent}{\mathsf{R}}
\newcommand{\writeEvent}{\mathsf{W}}
\newcommand{\F}{\mathsf{F}}
\newcommand{\Sw}{\mathsf{Sw}}
\newcommand{\yield}{\mathsf{yield}}
\newcommand{\p}{\mathsf{p}}
\newcommand{\s}{\mathsf{s}}
\let\C\relax  % Undefine the existing \C command
\newcommand{\C}{\mathcal{C}}
\newcommand{\T}{\mathcal{T}}
\newcommand{\ifExp}{\texttt{if}}
\newcommand{\elseExp}{\texttt{else}}
\newcommand{\whileExp}{\texttt{while}}
\newcommand{\exit}{\texttt{exit}}
\newcommand{\before}{\mathsf{before}}
\newcommand{\after}{\mathsf{after}}
\newcommand{\packetMatch}{\mathsf{packet\_match}}
\newcommand{\switch}{\mathsf{switch}}
\newcommand{\switches}{\mathsf{switches}}
\newcommand{\spawn}{\mathsf{spawn}}
\newcommand{\send}{\mathsf{send}}
\newcommand{\receive}{\mathsf{receive}}
\newcommand{\receiveAndReconcile}{\mathsf{receive\_and\_reconcile}}
\newcommand{\reconcile}{\mathsf{reconcile}}
\newcommand{\append}{\mathsf{append}}
\newcommand{\Actions}{\mathsf{Actions}}
\newcommand{\LocalState}{\mathsf{LocalState}}
\newcommand{\EventDomain}{\mathsf{Event}}
\newcommand{\event}{\mathsf{event}}
\newcommand{\idx}{\mathsf{idx}}
\newcommand{\cur}{\mathsf{cur}}
\newcommand{\src}{\mathsf{src}}
\newcommand{\Clock}{\mathsf{Clock}}
\newcommand{\VAR}{\mathsf{VAR}}
\newcommand{\var}{\mathsf{var}}
\newcommand{\timeField}{\mathsf{current\_clock}}
\newcommand{\srcField}{\mathsf{src}}
\newcommand{\dstField}{\mathsf{dst}}
\newcommand{\srcClock}{\mathsf{src\_clock}}
\newcommand{\dstClock}{\mathsf{dst\_clock}}
\newcommand{\variableField}{\mathsf{var}}
\newcommand{\valueField}{\mathsf{val}}
\newcommand{\BooleanExp}{\mathsf{Boolean\_Exp}}
\newcommand{\Event}{\mathsf{Event}}
\newcommand{\Exp}{\mathsf{Exp}}
\newcommand{\Thread}{\mathsf{Thread}}
\newcommand{\Heap}{\mathsf{Global\_Heap}}
\newcommand{\State}{\mathsf{State}}
\newcommand{\Pk}{\mathsf{Pk}}
\newcommand{\M}{\mathsf{M}}
\newcommand{\Bag}[1]{\mathsf{Bag}(#1)}
\newcommand{\N}{\mathbb{N}}
\newtheorem{lemma}{Lemma}
% Define corollary environment
\newtheorem{corollary}{Corollary}
%\newtheorem{definition}{Definition}



% Define the lemma and proof environment
%\newtheorem{lemma}{Lemma}[section] % Numbering by section
%\newenvironment{proof}[1][Proof]{\noindent \textbf{#1.} }{\hfill$\square$}

\renewcommand{\paragraph}[1]{\vspace{1mm}\noindent{\bf #1}\ }

\newcommand{\todo}[1]{{\color{red}{\textbf{[TODO]} #1}}}

\newcommand{\nate}[1]{\marginpar{\textcolor{red}{Nate: #1}}}
\newcommand{\jules}[1]{\marginpar{\textcolor{green}{Jules: #1}}}
\newcommand{\guy}[1]{\marginpar{\textcolor{blue}{Guy: #1}}}

\newcommand{\definition}[1]{%
	\vspace{2mm}%
	\noindent%
	\textbf{Definition:} \textit{#1}%
	\vspace{2mm}%
}

%\newtheorem{definition}{Definition}[section] % This creates a definition 
%%environment with numbering by section

%% Disable \Bbbk redefinition if it exists
%\let\Bbbk\relax
%\usepackage{amssymb}    % Load after to avoid conflicts


% Python style for code
\lstdefinestyle{pythonStyle}{
	language=Python,                           % Set language to Python
	backgroundcolor=\color{white},             % Set background color
	basicstyle=\ttfamily\footnotesize,         % Set font to typewriter 
	%(monospace) and size
	keywordstyle=\color{blue}\bfseries,        % Keywords in blue and bold
	commentstyle=\color{purple},               % Comments in purple
	stringstyle=\color{red},                   % Strings in red
	numbers=left,                              % Line numbers on the left
	numberstyle=\tiny\color{gray},             % Line numbers style
	stepnumber=1,                              % Number every line
	numbersep=5pt,                             % Distance between line numbers 
	%and code
	showstringspaces=false,                    % Don't show space characters
	breaklines=true,                           % Wrap long lines of code
	frame=single,                              % Draw a frame around the code
	rulecolor=\color{black},                   % Frame color
	captionpos=b,                              % Set caption position to bottom
	escapeinside={\%*}{*)},                    % Allow LaTeX comments inside 
	%code
	xleftmargin=1em, 
	% Left margin for code 
	%indentation
	% Left margin for code indentation
	keepspaces=true                            % This ensures spaces are kept 
	%in the code block
}

\begin{document}
	


\begin{abstract}
        Add abstract here
\end{abstract}

\maketitle

\section{Introduction}
\label{sec:introduction}

\todo{Introduction goes here.}

\todo{We motivate the problem of deciding serializability in programmable networks.}

\todo{We talk about some related work if relevant.}

\todo{We show that it's interesting with an example.}

\todo{We describe our main results.}

\paragraph{Contributions:}
\begin{itemize}
    \item Novel notion of serializability (``atomicity'') applicable to network systems
    \item Decidability results (1 main theorem: \textbf{automatically proving unbounded serializability}, 2 extra theorems: ser=ser decidable, ser=int undecidable)
    \item Implementation of decision procedure
    \item Advances in semilinear sets, Petri net reachability heuristics that makes the decision procedure work
\end{itemize}
\section{Problem Definition}
\label{sec:problemDefinition}

\begin{figure}[t]
    \begin{align*}
    e ::= &&&\\
       | & \; 0 \mid 1 \mid 2 \mid \ldots                                && \text{(numeric constants)} \\
       | & \; \nondet                                 && \text{(nondeterministic value: 0 or 1)}\\
       | & \; x := e                            && \text{(write to local variable / packet field)} \\
       | & \; x                                 && \text{(read from local variable / packet field)} \\
       | & \; X := e                            && \text{(write to global variable / switch variable)} \\
       | & \; X                                 && \text{(read from global variable / switch variable)} \\
       | & \; e_1 == e_2                        && \text{(equality test)} \\
       | & \; e_1 ; e_2                         && \text{(sequencing)} \\
       | & \; \ifkw(e_1)\{e_2\}\elsekw\{e_3\} && \text{(conditional)} \\
       | & \; \whilekw(e_1)\{e_2\}              && \text{(loop)} \\
       | & \; \yieldkw                      && \text{(yields to scheduler)}
    \end{align*}
    \caption{Syntax of expressions}
    \label{fig:syntax}
\end{figure}
    
A Network System $\mathcal{N}$ is a tuple $(G, L, \mathit{Req}, \mathit{Res}, g_0, \delta, \mathit{req}, \mathit{resp})$ where:
\begin{itemize}
\item $G$ is a set of global states representing switch state
\item $L$ is a set of local states representing packet contents
\item $\mathit{Req}$ is a set of request events
\item $\mathit{Res}$ is a set of response events
\item $g_0 \in G$ is the initial global state
\item $\mathit{req} \subseteq \mathit{Req} \times L$ is a request transition relation, which describes which requests turn into which packets
\item $\mathit{resp} \subseteq L \times \mathit{Res}$ is a response transition relation, which describes which packets turn into which responses
\item $\delta \subseteq (G \times L) \times (G \times L)$ is a transition relation for packet processing, which describes an atomic step of packet processing that may change the global state.
\end{itemize}

\begin{figure}[t]
    \centering
    \renewcommand{\arraystretch}{1.6}
    \[
    \begin{array}{c}
    \textbf{States and Transitions:}
    \\
    \quad
    \text{A (global) \emph{network state} is a triple }(g,\mathcal{P},M)\text{ where:}
    \\
    \quad
    g \in G \text{ is the current global switch state,}
    \\
    \quad
    \mathcal{P} \in \text{Multiset}(L \times \mathit{Req}) \text{ is a multiset of in-flight packets,}
    \\
    \quad
    M \in \text{Multiset}(\mathit{Req} \times \mathit{Res}) \text{ is a multiset of request--response pairs already completed.}
    \end{array}
    \]

    \[
    \begin{array}{c}
    \textbf{Initial state:}
    \\
    \quad (g_0,\,\varnothing,\,\varnothing)
    \end{array}
    \]

    \[
    \begin{array}{c}
    \textbf{Transition rules:}
    \\[1em]
    \text{(New Request)}\quad\infer{
    (r,\ell)\,\in\,\mathit{req}
    }
    {(g,\;\mathcal{P},\;M) \;\longrightarrow\; (g,\;\mathcal{P} \uplus \{(\ell,r)\},\;M)}
    \\[2em]
    \text{(Packet Step)}\quad\infer{
    ((g,\ell),\,(g',\ell')) \,\in\, \delta
    }
    {(g,\;\mathcal{P} \uplus \{(\ell,r)\},\;M) \;\longrightarrow\; (g',\;\mathcal{P} \uplus \{(\ell',r)\},\;M)}
    \\[2em]
    \text{(Response)}\quad\infer{
    (\ell,s)\,\in\,\mathit{resp}
    }
    {(g,\;\mathcal{P} \uplus \{(\ell,r)\},\;M) \;\longrightarrow\; (g,\;\mathcal{P},\;M \uplus \{(r,s)\})}
    \end{array}
    \]

    \[
    \begin{array}{c}
    \textbf{Complete runs:}
    \\
    \quad (g_0,\,\varnothing,\,\varnothing) \;\longrightarrow\; (g_1,\,\mathcal{P}_1,\,M_1) \;\longrightarrow\; \cdots \;\longrightarrow\; (g_n,\,\mathcal{P}_{n-1},\,M_{n-1}) \;\longrightarrow\; (g_n,\,\varnothing,\,M_n)
    \\[1em]
    \textbf{Interleaved run: } \text{the } \mathcal{P}_i \text{ can have more than one packet, and } \mathcal{P}_n = \varnothing \\
    \textbf{Serial run: } \text{ each } \mathcal{P}_i \text{ has at most one packet, and } \mathcal{P}_n = \varnothing\\
    \text{Int}(\mathcal{N}) = \{ M \in \text{Multiset}(\mathit{Req} \times \mathit{Res}) \mid \exists \text{ interleaved run } (g_0,\,\varnothing,\,\varnothing) \;\longrightarrow^*\; (g_n,\,\varnothing,\,M) \}\\
    \text{Ser}(\mathcal{N}) = \{ M \in \text{Multiset}(\mathit{Req} \times \mathit{Res}) \mid \exists \text{ serial run } (g_0,\,\varnothing,\,\varnothing) \;\longrightarrow^*\; (g_n,\,\varnothing,\,M) \}\\
    \end{array}
    \]

    \caption{State-transition rules for executions of
    \(\mathcal{N} = (G, L, \mathit{Req}, \mathit{Res}, g_0, \delta, \mathit{req}, \mathit{resp})\).
    A transition \(\longrightarrow\) modifies the triple \((g,\mathcal{P},M)\) by either (1) introducing a new request, (2) processing a packet step via \(\delta\), or (3) consuming a packet to produce its response.  When no more steps are possible, the result set \(M\) is the final multiset of request--response pairs that arose during the run.
    Runs are called \emph{serial} if there is at most one packet in flight at any time, whereas \emph{interleaved} runs may have multiple packets in flight at once.}
    \label{fig:network-transitions}
\end{figure}

\begin{definition}[Interleaved executions \(\mathrm{Int}(\mathcal{N})\)]
    Let \(\mathcal{N} = (G, L, \mathit{Req}, \mathit{Res}, g_0, \delta, \mathit{req}, \mathit{resp})\).
    An \emph{interleaved execution} begins in global state \(g = g_0\) with no packets in flight.
    At any point, one may:
    \begin{itemize}
        \item introduce a new request \(r \in \mathit{Req}\), producing an in-flight packet \(\ell \in L\) if \((r,\ell) \in \mathit{req}\);
        \item take an in-flight packet \(\ell\) and transition \(\ell \mapsto \ell'\) and \(g \mapsto g'\) if \(((g,\ell),(g',\ell'))\in\delta\);
        \item consume an in-flight packet \(\ell\) to produce a response \(s\) if \((\ell,s)\in \mathit{resp}\).
    \end{itemize}
    Each request \(r\) must eventually yield exactly one corresponding response \(s\) in a valid execution.
    The set of all finite multisets of \((r,s)\) pairs realizable by such interleavings is:
    \[
        \mathrm{Int}(\mathcal{N})
        \;=\;
        \bigl\{
        M \;\in\; \text{Multiset}(\mathit{Req}\times \mathit{Res})
        \;\mid\;
        M \text{ is realized by some interleaved run of }\mathcal{N}
        \bigr\}.
    \]
\end{definition}

\begin{definition}[Serial executions \(\mathrm{Ser}(\mathcal{N})\)]
In a \emph{serial} execution, requests are processed one at a time:
\begin{enumerate}
    \item Start in \(\,g_0\).
    \item Take a single request \(r\in \mathit{Req}\), convert it to a packet, process that packet until a response \(s\in \mathit{Res}\) is produced.
    \item Update the global state accordingly and repeat with the next request.
\end{enumerate}
No two requests overlap in processing.
The set of all finite multisets of \((r,s)\) pairs realizable by such one-at-a-time runs is:
\[
    \mathrm{Ser}(\mathcal{N})
    \;=\;
    \bigl\{
    M \;\in\; \text{Multiset}(\mathit{Req}\times \mathit{Res})
    \;\mid\;
    M \text{ is realized by some serial run of }\mathcal{N}
    \bigr\}.
\]
\end{definition}

\begin{theorem}[Serializability]
\label{thm:int-eq-ser}
Whether
\(
    \mathrm{Int}(\mathcal{N}) \;=\; \mathrm{Ser}(\mathcal{N})
\)
holds is decidable.
\end{theorem}

\begin{theorem}[Serial Equivalence]
\label{thm:ser-eq-ser}
Whether
\(
    \mathrm{Ser}(\mathcal{N}) \;=\; \mathrm{Ser}(\mathcal{N})
\)
holds is decidable.
\end{theorem}

\begin{theorem}[Interleaved Equivalence]
\label{thm:int-eq-int}
Whether
\(
    \mathrm{Int}(\mathcal{N}) \;=\; \mathrm{Int}(\mathcal{N})
\)
holds is undecidable.
\end{theorem}
Todo

(We don't do netkat any more.)
\section{Formulations}
\label{sec:formulations}

The.. 
\section{Proofs}
\label{sec:proofs}

The theorems..
\section{Implementation}
\label{sec:implementation}

The implementation..


\newpage
\section{Related Work}
\label{sec:relatedWork}

Related Work includes...


Serializability first introduced by Eswaran et al.~\cite{EsGrKoTr76}.  It is the first to put forth serializability as a correctness condition for concurrent transaction execution.
The paper also covers conflict serializability. 
%
Papadimitriou~\cite{Pa79} proved that even deciding the history of a single interleaving is serializable is NP-hard.

conflict serializability is enforced during runtime in 
with with pessimistic locking approaches (e.g, 2-Phase locking~\cite{BeHaGo87}), or with optimistic locking approaches (e.g., Optimistic Concurrency Control (OCC))~\cite{KuRo81, BuMo06}


Alur et al.~\cite{AlMcPe96}... 
cover conflict serializability (not "regular" serializability, which is what we do). Furthermore, a main caveat is that they focus on a bounded number of transactions



continued by~\cite{BoEmEnHa13}.
In the followup paper (Boujjani et al.) - they also cover conflict serializability, but find a stronger result than Alur, based on unbounded transactions. They find an interesting result that although you can have an infinite conflict graph (when having infinite transactions), then you can still decide conflict serializability in the unbounded case by finding a cycle in the graph when it's non (conflict) serializable, and the cycle length surprisingly does not depend on the number of transactions, which is pretty cool. Another point is that they define a VASS (=Petri Net) that represents the interleaving, and their definition for it is similar to our PN. They then modify it to include a conflict cycle. The most relevant part to us in this paper is that it's on an unbounded number of transactions and also, that they represent Int(S) with a VASS that is similar to us. Still, it's not our notion of serializability (and indeed, they have EXPTIME complexity, while we're probably Ackermann complete?).



Another line of work leverages the highly expressive \textit{Logic of Temporal Actions} (TLA)~\cite{La94}. 
%
These work encode
serializability in TLA+~\cite{SoVaVi20, Ho24},...model checkers (such as TLC and Apalache)~\cite{YuMaLa99, KoKuTr19}.
%
TLA+ can indeed encode an infinite number of transactions. For example, here is the TLA+ spec for encoding serializability.
However, for doing model checking on a TLA spec (with the TLC model checker) --- the model checker takes a .cfg file as additional input, in in the .cfg you explicitly specify all of the sets in the model, and these have to be finite. You can see this here where the model checking file needs to encode in advance the number of transactions (see attached figure)



\todo{go over the paper and its citing papers}
Me:
1992 paper today, they seem to model a concurrent execution with petri nets but they don't ask if all executions are serializable which is our subject matter

Furthermore, other work cover additional consistency models, such as causal consistency, which was put forth by Lamport~\cite{La78}, en extended to shared memory systems as \textit{causal memory}~\cite{AhNeBuKoHu95}. (include causal + consistency, designed in COPS~\cite{LlFrKaAn11}). The have been a plethora of works on model checking systems that adhere to causal consistency, and hence the complexity of such procedures~\cite{BoEnGuHa17,ZeBiBoEnEr19,LaBo20}

\todo{go over Mark's notes}



\todo{go over Espinoza complexity results}

Our work also builds upon both theoretical literature, as well as practical results, pertaining to Petri Nets~\cite{Reisig12,Mu89}.
%
In terms of theory, our undecidability result is based on a classic result by Hack~\cite{Ha76}, showing that, given two Petri Nets, it is undecdiable to answer whether they have equivalent reachbility sets. Hack based his result on the work of Rabin (which was never published). These undecdiability results follow from a series of reductions, originating from Hilbert's 10th problem, i.e., deciding if a Diophantine polynomial has an integer root (a problem that was proved undecidable by Matijas{\'e}vi{\v{c}}~\cite{Ma70}).
%
Later, Jan{\v{c}}ar~\cite{Ja95} simplified this proof by using Petri Nets to simulate 2-counter Minsky Machines, which are univerally comptuable and hence undecidable~\cite{Mi67}. Moreover, Jan{\v{c}}ar's result is stronger as it shows that this equality is undecidable even for Petri Nets with five unbounded places~\cite{Ja95}.
%
We refer the reader to a survey by Esparza and Nielsen~\cite{EsNi94} for a comprehensive summary on additional decidability results pertaining to Petri Nets.


Deciding reachaility for Petri Nets:

- Mayr~\cite{Ma81} was the first to put forth an algorithm for deciding reachability for Petri Nets in the unbounded case (note that for a bounded net this is trivial as you can enumerate all reachable markings.)

- This algorithm was later improved and simplified by Kosaraju~\cite{Ko82}, and then by Lambert~\cite{La92}.

- Very recently, the Complexity was recently proven to be Ackermann complete~\cite{CzWo22}, indicating it inherently infeasible in practice to solve on large problems.

These algorithms have inspired various Petri Net reachability tools, such as K-Reach~\cite{DiLa20} and SMPT~\cite{AmDa23}, which employs an SMT-based approach~\cite{AmBeDa21, AmDaHu22} which reduces the reachability problem to a satisfiability query (that is subsequently dispatched to the state-of-the-art Z3 solver~\cite{DeBj08}).

 


\section{Discussion}
\label{sec:discussion}

\subsection{Conclusion}

To the best of our knowledge, ours is the first 
tool capable of verifying serializability in unbounded domains.
%
While these contributions represent significant advances, to our knowledge, our 
work is the first to:
(i) Decide serializability universally --- \textit{considering all executions} 
purely through program semantics and final states, independent of read/write 
conflicts; 
(ii) Support \textit{unbounded} transaction systems; and
(iii) Provide a complete end-to-end implementation.

\todo{Limitations?}
Examples we cannot solve, future work that would help
To conclude..


\subsection{Future Work}
Next..

We plan to use polyhedral reductions~\cite{AmBeDa21} that are structural reduction~\cite{Be87,BeLeDa20} of the form $(N_1, m_1) \vartriangleright_E (N_2, m_2)$, where $(N_1, m_1)$ is the Petri net we want to analyze, $(N_2, m_2)$ is a reduced version of this net (easier to model check), and $E$ is a Presburger formula that permits to reconstruct of the state space of $N_1$ from that of $N_2$. We plan to leverage this mechanism to trace back certificates obtained on the reduced net $N_2$ to the original net $N_1$, that would be the only kind of structural reduction for which such operation is possible.

\todo{Different notions of serializability}
\begin{itemize}
    \item \todo{Current notion: clients independently submit a request and get a response, and later they all get together and see if what they got was serializable}
    \item \todo{Stronger: clients are not independent, or sequentially execute some parts. General: we have some happens-before on the requests/responses}
    \item \todo{Weaker: the clients cannot communicate with each other afterwards to determine whether what they got was serializable, or they can only communicate in a limited way}
    \item \todo{Infinite / unbounded executions}
\end{itemize}




\newpage

{
	\bibliographystyle{abbrv}
	\bibliography{references}
}

\newpage
\appendix
\renewcommand{\thesection}{\Alph{section}}

%\section*{Appendix}
%
%\input{ADD_APPENDIX_FILE_HERE.tex}

\end{document}

