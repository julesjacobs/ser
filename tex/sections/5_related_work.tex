\section{Related Work}
\label{sec:relatedWork}

% from the CAV-2010 Vafeiadis paper (+ some other papers in our slack RW thread):
%
%1. model checking techniques - find violations, but do not prove correctness as they work on finite-state systems
%
%2. static (and specifically, shape) analysis techniques (sometimes) work on unbounded threads but ,any of these require linearization points (annotated manually or automatically). Also, a ``failed'' proof can indicate incorrect linearization.
%
%3. manual verification efforts

\subsection{Notions of serializability.}
\label{sec:related:notions-of-serializability}

\jules{Main goal of this section is to place our notion of serializability in context of the literature. If our notion matches some existing, then we should say so. If it does not, we should somehow argue that it is novel but reasonable.}
\todo{Describe the differences between our notion and the others}
\todo{Find the notion(s) most closely related to ours.}

Serializability was first formalized by Eswaran et al.~\cite{EsGrKoTr76} as a correctness condition for concurrent transaction execution. Their work also introduced what was later referred to as conflict serializability, a stricter variant that requires equivalence to a serial schedule solely by reordering non-conflicting operations. Papadimitriou~\cite{Pa79, Pa86} later proved that determining serializability even for a given, single interleaving is NP-hard. 
%
Moreover, although conflict serializability is more conservative measure than serializability, it is easier for database schedulers to enforce during runtime by various approaches. 
%
These approaches are typically categorized as either \textit{pessimistic} locking approaches, e.g, Two-Phase Locking~\cite{BeHaGo87}, or alternatively --- \textit{optimistic} locking approaches, e.g., Optimistic Concurrency Control (OCC)~\cite{KuRo81, BuMo06}.
%
Both approaches ensure acyclicity in the conflict graph --- a necessary and sufficient condition for conflict serializability~\cite{SiMa10}. However, because these approaches \textit{ignore program semantics}, these may incorrectly reject executions that, although not conflict-serializable, are still valid serializable executions, i.e., have the same result as a serial execution. In terms of verification, Nagar and Jagannathan~\cite{KaJa18} presented an automatic static analysis technique to find violations of conflict serializability
%
Sinha et al.~\cite{SiMaWaGu11b} present a sound and incomplete predictive analysis technique for detecting violations of conflict serializability.


Several works have proposed relaxations of the (strong) consistency notion of serializability guarantees. Rastogi et al.~\cite{RaMeBrKoSi93} introduced predicate-wise serializability (PWSR), which preserves database invariants while permitting non-atomic transactions. 
%
Other relaxations focus on weaker consistency models: Lamport’s causal consistency~\cite{La78}, later generalized to shared memory as causal memory~\cite{AhNeBuKoHu95} and implemented in systems like COPS~\cite{LlFrKaAn11}, has inspired extensive research on model checking and complexity analysis~\cite{BoEnGuHa17,ZeBiBoEnEr19,LaBo20}. 
%
Brutschy et al.~\cite{BrDiMuVe17} present a dynamic analysis algorithm and a tool that checks whether a given program execution is conflict serializable, in an eventually consistent data store. In a followup work~\cite{BrDiMuVe18} further bridges these concepts by statically detecting non-(conflict)-serializable behaviors in causally consistent databases.
%
%Rinetzky et al.~\cite{RiBoRaSaYa} present a conservative static analysis techniques for verifying view-serializability on specific program sub-types.

\guy{Mark, please see my masked comment on RiBoRaSaYa. Do we need to mention 
them?}

\todo{start}

Work on Serializability;

- There are works that focus on manual proofs of serializability. For example, Tasiran~\cite{Ta08} proved serializability of the Bartok STM, and Colvin et al.~\cite{CoGrLuMo06} prove that a list-based set algorithm is linearizable by simulating the observed behavior with input/output automata.
\guy{Mark could you take a 2nd look at CoGrLuMo06? Not sure if this is considered manual. One paper which I put in the slack channel mentioned that it semiautomatically verifies linearizability of an implementation.}
Additional verification attempt of linearizability have been presented both with (e.g.~\cite{CoGrLuMo06}) and without (e.g.~\cite{DoGrLuMo04}) the use of proof assistants.

- Strict serializability (a.k.a. SSR~\cite{Pa79}) is a stronger consistency notion in which an execution need not only be serializable but also respects the real-time ordering of the transactions, i.e., it is not enough for the interleaving to be equivalent to a serial one, but the serial execution must also preserve the ordering of the transactions.
%
Guerraoui et al.~\cite{GuHeJoSi08} present a algorithm for model checking strict serializability fro two threads. The work was later extended~\cite{GuHeSi11}, to include a manual proof for an arbitrary number of threads.
%
Konig and Wehrheim recently proved that that it is possible to decide whether all executions of a program are strictly serializable, given that the transactions are live~\cite{KoWe21}.


Work on linearizability:

- linearizability (defined by Herlihy and Wing~\cite{HeWe87, HeWi90}) of a concurrent data structure implies that every concurrent execution with operations on the data structure appears as if the operations occurred atomically, while respecting real-time ordering and obeying the object specifications. 
%
Equivalently, this can be viewed as a case of strict serializability in which each transaction consists of a single operation, operating on a single concurrent object~\cite{WaSt06a}.




- techniques used directly and indirectly for testing 
linearizability~\cite{WiGo93, PrGr12, PrGr13, EmEn17}. For example, 
Lowe~\cite{Lo17} presents a testing framework for linearizability by randomly 
generating histories and subsequently testing if the generated histories are 
linearizable.

- In a series of papers, Wang and Stoller put forth runtime techniques for detecting violations serializability (also termed \textit{atomicity})~\cite{WaSt06a} as well as fragments of serializability such as conflict serializability and view-serializability~\cite{WaSt06b}. They do so by  checking whether a given execution can be recombined to generate non-serializable executions.
%
Sinha and Malik~\cite{SiMa10} also put forth a conflict serializability checker during runtime.

- Linearizability has also been proven for specific data types. For example. 
Wing and Gong~\cite{WiGo93} prove it for  (unbounded) FIFO queues, (unbounded) 
priority queues and other data structures. Chakraborty et al.~\cite{ChHeSeVa15} 
later provided a method for checking linearizability of queue-based algorithms, 
without the use of linearization points. Cern{\`y} et al.~\cite{CeRaZuChAl10} 
present CoLT, a model checker for linearizability of singly-linked heap-based 
objects. However, their approach is complete only with regard to a bounded 
number of threads. 
%
Bouajjani et al.~\cite{BoEnWa17} present a recursive algorithm for idnetifying 
linearizability violations in priority queues (based on register automata), in 
a method similar to the one for finding linearizability violations stacks and 
(regular) queues~\cite{BoEmEnHa18}, nonfinite-state automata.



We define a recursive procedure
queues requires register automata w




- verification approaches for linearizability:



(i) model checking techniques include:

Line-up~\cite{BuDeMuTa10} (built on the CHESS model checker~\cite{MuQaBaBaNaNe08}) includes a heuristic-driven technique that searches for violations of linearizability by enumerating all possible serializations. Similar to spirit is LinTSO~\cite{BuGoMuYa12} which search for linearizability violations in the Total Store
Order (TSO) weak memory model.
%
Violat~\cite{EmEn19} generates tests to identify linearizability violations, by 
enumeration linearizations efficiently, per program schema (instead of per 
execution, as in~\cite{BuDeMuTa10}).
%
Additional model checking techniques were proposed by Burnim et al.~\cite{BuNeSe11}, which is similar to  Line-up~\cite{BuDeMuTa10} but based on leveraging bridge predicates~\cite{BuSe09}. 
Liu et al.~\cite{LiChLiSuZhDo12} build upon PAT~\cite{SuLuDoPa09} (also used in~\cite{LiChLiSu09,Zh11}) and verify linearizability through the lens of refinement checking optimization, for a finite number of threads.
%
\guy{Mark, could you please check LiChLiSu09,Zh11}
%
Recently, Golovin et al.~\cite{GoKoVa25} presented \textit{RELINCHE}, a model checker for bounded-linearizability, in which a predefined number of operations can be invoked invoked.
%
Other automatic linearizability checking tools include the CDSSpec specification checker under the C/C++ 11 memory model, and Lincheck~\cite{KoDeSoTsAl23} for verifying linearizability in JVM by Ou and Demsky~\cite{OuDe17}. 
%
Burckhardt et al.~\cite{BuAlMa07} employ a SAT solver and check for linearizability violations of specific client programs.
%
As far as we are aware, unlike our algorithm, none of these tools afford complete coverage for the case of unbounded threads.
%

Other model checking techniques (e.g.,~\cite{Fl04}) rely on specifying linearization points, i.e., points in which the event occurs logically. We note that identifying all such points can be quite challenging~\cite{VeYaYo09}.
%
These include the work of Vechev et al.~\cite{VeYaYo09}, built upon SPIN~\cite{Ho97}, and extending the authors' PARGLIDER tool~\cite{VeYa08}~\footnote{Note that Vechev et al. can also apply their technique without linearization points, but solely on bounded executions.}
%
Static analysis techniques may prove linearizability for the 
bounded~\cite{AmRiReSaYa07, BeLeMaRaSa08, MaLeSaRaBe08} and unbounded cases~\cite{BeLeMaRaSa08, Va09, 
Va10}, but rely on heuristics and the manual/automatic annotation of 
linearization points. 
%
Lian and Feng~\cite{LiFe13} propose a sound program logic that can prove 
linearizability with non-fixed linearization points.
%
Other techniques that depend or linearization points 
include~\cite{OhRiVeYaYo10, ZhPeHa15, AbJoTr16}. 
%
\guy{Mark can you take a look at Va10, LiFe13, and ZhPeHa15?}

We note that both model-checking techniques (e.g.,~\cite{CeRaZuChAl10}) and static analysis techniques which rely on linearization points are inconclusive, as any failed proof can be due to incorrect annotation of the linearization points, as previously observed~\cite{BoEmCoHa15}.
%
Shacham et al.~\cite{ShBrAiSaVeYa11} use dynamic analysis to identify violations of linearizability in concurrent data structures, and combine it with a manual proof when there technique did not find a violation.
%
We also mention the symbolic-reasoning-based (incomplete) approach by Emmi et al.~\cite{EmEnHa15} to identify violations of \textit{observational refinement} --- a property equivalent to linearizability in some settings~\cite{FiOhRiYa10, BoEmCoHa15}.
%
Zhang et al.~\cite{ZhChWa13}, relaxed the notion of linearizability to quasi linearizability, and put forth a method to identify violations thereof in concurrent data structures.



\todo{Guy: read more about shape/static analysis techniques}




(ii) Other techniques for linearizability proof include the use of theorem provers (e.g.~\cite{CoDoGr05, DeScWe11}).


- complexity results for linearizability:

Conflict serializability

Alur et al.~\cite{AlMcPe96} prove that conflict serializability is decidable 
(in PSPACE) and that linearizability is decidable (in EXPSPACE) for concurrent 
system with a bounded number of threads.
%
Bouajjani et al.~\cite{BoEmEnHa13} extend these results and prove that conflict serializability is decidable even for the unbounded case. The authors also show that linearizability, on the other hand, in undecidable for the unbounded case, except for setting (which the authors define as \textit{bounded-barrier linearizability}) in which the problem is decidable for an unbound number of operations.
%
We note that the proof of Bouajjani et al. is reminiscent to our setting, as the authors also reduce the problem to a Petri Net reachability problem.
%
However, as far as we are aware, there is no tool implementing this approach and verifying linearizability with Petri Nets.
%
In followup work, Bouajjani et al. prove that linearizability is decidable in 
the unbounded case for specific abstract state types~\cite{BoEmEnHa18}, in 
which the authors rely on checking coverability in a Vector Addition System 
with States (VASS), which was proven to be in EXPSPACE~\cite{Ra78}.

- result for conflict serializability

Various work focus on checking conflict serializability, .e.g. Farzan and Mahusudan~\cite{FaMa08} present a monitoring-based decision procedure for conflict serializability of a bounded number of operations, and Flanagan et al.~\cite{FlFrYi08} present Velodrome --- a dynamic analyzer for conflict serializability. Additional dynamic approaches include~\cite{FlFr04, XuBoRa05, WaSt06a, CoOlPnTuZu07, EmMaMa10, SiMaWaGu11a} and others.
%
%Other works~\cite{XuBoRa05} have also put forth techniques to automatically detect conflict-serializability violations.
%
Hatcliff et al>~\cite{HaRoDw04} demonstrate the use of the Bogor model checker~\cite{RoDwHa03} and check atomicity w.r.t. Liptopn's reduction theory of left/right movers~\cite{Li75} (which is reminiscent of conflict-serializability)
%
Von Praun and Gross~\cite{VoGr04} present an unsound static analysis technique for identifying potential atomicity violations. Additional static analysis techniques are based on type systems, e.g.~\cite{FlQa03, FlFrLiQa08}.
%
We also note that conflict serializability has been studied in relation to weak memory models~\cite{EnFa16}.


\todo{end}


\subsection{Deciding serializability.}
\label{sec:related:deciding-serializability}

\jules{Main goal of this section is to describe all the related work on deciding serializability, from deciding it on traces, to bounded model checking, to unbounded ones. Key distinction is on two axes: (1) the notion of serializability used (2) whether they \emph{prove} serializability of unbounded systems or whether they just model check.}

The \textit{membership problem} of serializability, is deciding whether a specific interleaving is serializable. This has been proven to be NP-complete by Papadimitriou~\cite{Pa79}, a result that was later extended~\cite{BiEn19} to other consistency models.
%
The \textit{correctness problem} on the other hand, is much harder, and pertains to deciding whether \textit{all} executions of a program ar serializable.

%
Alur et al.~\cite{AlMcPe96} established that correctness problem for conflict serializability is decidable (and in PSPACE) for a bounded transaction systems. Bouajjani et al.~\cite{BoEmEnHa13} later proved that decidability is also extended to unbounded systems (and EXPTIME-complete). Their key insight reveals that while the conflict graph becomes infinite, cycle detection, and thus conflict serializability, is independent of transaction count. By modeling transactions via Vector Addition Systems (equivalent to Petri Nets), they provide a finite framework for analyzing infinite behaviors. This approach inspired our use of Petri Nets to capture Int(S).
%
We also note that the correctness problem of linearizability is in 
EXPSPACE~\cite{AlMcPe96} when bounding the number of threads, and undecidable 
otherwise~\cite{BoEmEnHa13}. The easier, membership problem is NP-complete in 
general, as proven by Gibbons and Korach~\cite{GiKo97}. Their result was later 
extended to \textit{collections}~\cite{EmEn18}.
\guy{Mark could you check EmEn18?}

A separate research direction attempts to \textit{directly} verify serializability (or conflict-serializability~\cite{CoOlPnTuZu07}), without limiting it to conflict restrictions, as done in other works. Towards this end, the expressive Temporal Logic of Actions (TLA)~\cite{La94} is used to encode a formal specification that validates whether only serializable executions occur. While TLA can naturally encode ``real'' serializability (based on final-state equivalence), existing TLA-based approaches~\cite{SoVaVi20, Ho24} remain limited to bounded transaction systems. This limitation stems from TLA/TLA+ model checkers like TLC and Apalache~\cite{YuMaLa99, KoKuTr19}, which require finite-state verification and cannot handle unbounded transaction counts.

%
While these contributions represent significant advances, to our knowledge, our work is the first to:
(i) Decide serializability universally --- \textit{considering all executions} purely through program semantics and final states, independent of read/write conflicts; 
(ii) Support \textit{unbounded} transaction systems; and
(iii) Provide a complete end-to-end implementation.

\subsection{Petri Nets, VAS(S) \& Semilinear sets, Presburger arithmetic.}
\label{sec:related:petri}

\jules{The main goal of this section is to place our technical methods in context. Should include stuff on petri reachability as well as semilinear sets / presburger. We should contrast our implementation methods with existing stuff as far as we want to claim novelty (e.g., of the semilinear set implementation and heuristics or of the Petri net reduction heuristics)}

In addition, our work builds on both theoretical and practical advances in Petri net research~\cite{Mu89, Es96, Re12, EsNi24}. The undecidability we prove for equivalence of interleavings stems from Hack’s seminal result~\cite{Ha76, HaThesis76} showing the undecidability of reachability set equivalence for Petri Nets. This undecidability originates in a series of reductions from Hilbert’s 10th problem, specifically the possibility of determining whether there exists an integer root for Diophantine equations, a problem that was later proven undecidable by Matijasēvič~\cite{Ma70}.
%
Jančar~\cite{Ja95} later provided an alternative proof to this undecidability result, by showing that Petri nets can simulate universal (and thus undecidable) 2-counter Minsky machines~\cite{Mi67}. In addition, Jančar further strengthened the original result by proving that undecidability holds even for Petri nets with just five unbounded places.

Furthermore, our approach also builds on Petri net reachability algorithms, which determine whether a given marking is attainable. While the solution is straightforward for bounded nets (through exhaustive enumeration), the solution for the unbounded case is highly nontrivial, and was first solved by Mayr~\cite{Ma81}, with subsequent improvements by Kosaraju~\cite{Ko82} and Lambert~\cite{La92}. Recent work~\cite{CzWo22} has also established this problem is Ackermann-complete, implying that, although decidable, it is practically infeasible to solve on large nets in the worst case.

These theoretical advances in Petri Net reachability have given rise to a plethora of practical tools, including K-Reach~\cite{DiLa20}, DICER~\cite{XiZhLi21}, MARCIE~\cite{HeRoSc13}, and others. 
%
Specifically, our implementation leverages SMPT~\cite{AmDa23}, a state-of-the-art Petri Net reachability tool that combines SMT-solving with structural invariants~\cite{AmBeDa21, AmDaHu22}. At a high level, SMPT formulates reachability as satisfiability queries (dispatched to the Z3 solver~\cite{DeBj08}) while curtailing the search space by proactively inferring invariants on the net's structure.
%
%We refer the reader to a survey by Esparza and Nielsen~\cite{EsNi94} (recently republished in~\cite{EsNi24}) for a comprehensive summary of additional decidability results pertaining to Petri Nets.


 
 






%Serializability first introduced by Eswaran et al.~\cite{EsGrKoTr76}. It is the first to put forth serializability as a correctness condition for concurrent transaction execution.
%The paper also covers conflict serializability --- a strictly stronger consistency property than serializability, that does not only require that the final state of the system be attained by an equivalent serial execution, but also, that this equivalence be attained by allowing only specific (``non-conflicting'') operations to be reordered.
%%
%Papadimitriou~\cite{Pa79} proved that it is NP-hard to decide whether even the history of a single interleaving is serializable. 
%%
%Moreover, although conflict serializability is more conservative measure than serializability, it is easier to enforce during runtime by various approaches. 
%%
%These approaches are typically categorized as either \textit{pessimistic} locking approaches, e.g, 2-Phase Locking~\cite{BeHaGo87}, or alternatively --- \textit{optimistic} locking approaches, e.g., Optimistic Concurrency Control (OCC)~\cite{KuRo81, BuMo06}.
%%
%
%Furthermore, most work, both in theory and in practice, focuses on proving theorems on conflict serializability, due to is being more straightforward, and corresponding to the programs dependency graph, and \textit{without taking the actual semantics into account}.
%%
%Alur et al.~\cite{AlMcPe96} cover the complexity for deciding conflict serializability, given a bound on the number of transactions. 
%%
%This work was later continued by Bouajjaniet al.~\cite{BoEmEnHa13}, which demonstrate that the problem of deciding whether a program with an \textit{unbounded} number of transaction is conflict serializable, is also decidable and is EXPTIME-complete. The authors show that although the conflict graph in such a case is infinite (and hence, infeasible to traverse) --- conflict serializability can still be decided as the size of the cycle (if it exists), surprisingly, does not depend on the number of transactions. The authors also emulate multiple transactions in a shared memory system with a Vector addition system, and equivalent object to a Petri Net. We took inspiration by defining a Petri Net to capture Int(S). 
%
%In another line of work, there is an attempt to \textit{directly} validate (regular, non-conflicting) serializability by encoding this specification in the highly expressive \textit{Logic of Temporal Actions} (TLA)~\cite{La94}. 
%%
%Although, unlike the aforementioned works, TLA itself allows encoding serializability in the original form (focusing on the final state of the variables), such works~\cite{SoVaVi20, Ho24} cannot validate this behavior for an \textit{unbounded} number of transactions. This is because, although TLA/TLA+ allow encoding the properties of interest, their model checkers (such as TLC and Apalache)~\cite{YuMaLa99, KoKuTr19} can only operate on a finite and predefined number of transactions.
%%
%Although these works present important progress, as far as we are aware, our work is the first to decide serializability for all executions, based solely on the program semantics and final state, regardless of read/write conflicts. Furthermore, ours is the first to handle the unbounded case; and to supply an actual end-to-end implementation.
%
%
%Other work relaxes the (strong) consistency notion of serializability and allows weaker consistency notions. For example, Rastogi et al.~\cite{RaMeBrKoSi93}
%introduce \textit{predicate-wise serializability} (PDSR) --- a relaxation of serializability in which transactions might not be atomic, but are still required to maintain some desired database consistency predicate
%%
%Furthermore, other relaxations focus on weaker consistency models. One such model is causal consistency, which was put forth by Lamport~\cite{La78}, en extended to shared memory systems as \textit{causal memory}~\cite{AhNeBuKoHu95}. (include causal + consistency, designed in COPS~\cite{LlFrKaAn11}). The have been a plethora of works on model checking systems that adhere to causal consistency, and hence the complexity of such procedures~\cite{BoEnGuHa17,ZeBiBoEnEr19,LaBo20}.
%%
%We also note that some work combine various consistency notions. These include the recent work by Brutschy et al.~\cite{BrDiMuVe18}, who put form a method to statically detect non-serializable executions on top of
%causally-consistent databases.
%
%
%
%
%Our work also builds upon both theoretical literature, as well as practical results, pertaining to Petri Nets~\cite{Mu89, Es96, Re12}.
%%
%Firstly, our undecidability result is based on a classic result by Hack~\cite{Ha76, HaThesis76}, showing that, given two Petri Nets, it is undecidable to answer whether they have equivalent reachability sets. Hack based his result on the work of Rabin (which was never published). These undecidability results follow from a series of reductions, originating from Hilbert's 10th problem, i.e., deciding if a Diophantine polynomial has an integer root (a problem that was proved undecidable by Matijas{\'e}vi{\v{c}}~\cite{Ma70}).
%%
%Later, Jan{\v{c}}ar~\cite{Ja95} proved this result by demonstrating that Petri Nets can simulate 2-counter Minsky Machines~\cite{Mi67}, which are universally computable and hence undecidable. Moreover, Jan{\v{c}}ar strengthened the original result and proved that reachability equivalence is undecidable even for Petri Nets with five unbounded places~\cite{Ja95}.
%%
%
%Our decision procedure itself is based on an algorithm for deciding whether a given marking is reachable, for a Petri Net.
%%
%Mayr~\cite{Ma81} was the first to put forth an algorithm for this problem given a (potentially, unbounded) Petri Net (note that for a bounded case this is straightforward, as you can enumerate all reachable markings.)
%%
%Mayr's reachability algorithm was later improved and simplified by Kosaraju~\cite{Ko82}, and then again by Lambert~\cite{La92}.
%%
%Very recently, this problem was also proven to be Ackermann complete~\cite{CzWo22}, implying that, although decidable, it is practically infeasible to solve on large nets.
%%
%Furthermore, these theoretical algorithms have inspired various tools, such as K-Reach~\cite{DiLa20}, DICER~\cite{XiZhLi21}, MARCIE~\cite{HeRoSc13}, and others. 
%%
%Specifically, our tool employs SMPT~\cite{AmDa23}, a state-of-the-art Petri Net reachability tool, which employs an SMT-based approach~\cite{AmBeDa21, AmDaHu22}. SMPT curtails the search space by reducing the reachability problem to a satisfiability query (that is subsequently dispatched to the Z3 solver~\cite{DeBj08}) and inferring invariants on the net's structure.
%%
%We refer the reader to a survey by Esparza and Nielsen~\cite{EsNi94} (recently republished in~\cite{EsNi24}) for a comprehensive summary on additional decidability results pertaining to Petri Nets.

