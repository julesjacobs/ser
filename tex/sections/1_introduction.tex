\section{Introduction}
\label{sec:introduction}

\todo{Introduction goes here.}

\todo{We motivate the problem of deciding serializability in programmable networks.}

\todo{We talk about some related work if relevant.}

\todo{We show that it's interesting with an example.}

\todo{We describe our main results.}

\todo{Important point: mention the french people here early on, so that the reviewers know that we are aware of them. Also explain clearly what is new here: a tool that can actually run and output certificates.}

\paragraph{Contributions:}
\begin{itemize}
    \item Novel notion of serializability (``atomicity'' or ``semantic serializabillity'') applicable to network systems (\Cref{sec:problem-definition,sec:related:notions-of-serializability})
    \item Decidability results (1 main theorem: \textbf{automatically proving unbounded serializability}, 2 extra theorems: ser=ser decidable, int=int undecidable) (\Cref{sec:formal-results,sec:related:deciding-serializability})
    \item Implementation of decision procedure. Advances in semilinear sets, Petri net reachability heuristics that makes the decision procedure work. (\Cref{sec:implementation,sec:related:petri})
     \item optimizations
     \item case study - real world problems
     \item proof certificate checker
\end{itemize}

\newpage


\paragraph{Motivation:}

example - 1

	
	
\noindent
\begin{minipage}[t]{0.45\textwidth}
	\begin{minipage}[t]{\textwidth}
		\begin{lstlisting}[caption={Without yield or lock (serializable)}]
    request foo: 
        X := 1 
        // no yield
        y := X 
        X := 0
        return y 
		\end{lstlisting}
	\end{minipage}
	\vspace{1em}
	\begin{minipage}[t]{\textwidth}
		\begin{lstlisting}[caption={With yield (not serializable)}]
    request foo: 
        X := 1 
        yield 
        y := X + 1
        X := 0
        return y 	
		\end{lstlisting}
	\end{minipage}
\end{minipage}%
\hfill
\begin{minipage}[t]{0.45\textwidth}
	\begin{lstlisting}[caption={With yield and lock (serializable)}]
    request foo: 
        // lock
        while (L == 1): 
            yield
        L := 1 

        X := 1
        yield
        y := X 
        X := 0

        // unlock    
        L := 0
        return y 
	\end{lstlisting}
\end{minipage}


%\todo{guy - START}

\[
req \coloneq 
\Bigg \{
\Bigg [
\begin{array}{c c c}
	\text{
		\begin{tikzpicture}[baseline=(textnode.base)]
			\node[
			draw=black,
			fill=ForestGreen!20,
			text=black,
			diamond,
			aspect=2,
			inner sep=4pt
			] (textnode) {foo};
		\end{tikzpicture}
	} 
	&
	\rightarrow
	&
	\bigg(
	\begin{tabular}{c c}
		\text{\large\textcolor{Peach}{[y=0]}},\quad & 
		\begin{minipage}{0.14\linewidth}
			\begin{lstlisting}[language=CustomPseudoCode,numbers=none]
X := 1 
yield 
y := X + 1
X := 0
return y
			\end{lstlisting}
		\end{minipage}
	\end{tabular}
	\bigg)
\end{array}
\Bigg ]
\Bigg \}
\]


\[
resp \coloneq
\Bigg \{
\Bigg [ 
\begin{array}{c c c}
	\bigg(
	\begin{tabular}{c c}
		\text{\large\textcolor{Peach}{[y=0]}},\quad & 
		\begin{minipage}{0.11\linewidth}
			\begin{lstlisting}[language=CustomPseudoCode,numbers=none]
return y
			\end{lstlisting}
		\end{minipage}
	\end{tabular}
	\bigg)
	&
	\rightarrow
	&
	\text{
		\begin{tikzpicture}[baseline=(textnode.base)]
			\node[
			draw=black,                      % ← outline color
			fill=RedViolet!20,            % ← light green fill
			text=black,
			diamond,
			aspect=2,
			inner sep=4pt
			] (textnode) {0};
		\end{tikzpicture}
	} 
\end{array}
\Bigg ]
,
%\quad
\Bigg [ 
\begin{array}{c c c}
	\bigg(
	\begin{tabular}{c c}
		\text{\large\textcolor{Peach}{[y=1]}},\quad & 
		\begin{minipage}{0.11\linewidth}
			\begin{lstlisting}[language=CustomPseudoCode,numbers=none]
return y
			\end{lstlisting}
		\end{minipage}
	\end{tabular}
	\bigg)
	&
	\rightarrow
	&
	\text{
		\begin{tikzpicture}[baseline=(textnode.base), scale=0.7]
			\node[
			draw=black,                      % ← outline color
			fill=RedViolet!20,            % ← light green fill
			text=black,
			diamond,
			aspect=2,
			inner sep=4pt
			] (textnode) {1};
		\end{tikzpicture}
	} 
\end{array}
\Bigg ] 
\Bigg \} 
\]





\[
\begin{array}{l}
	\delta := 
	\Bigg\{ 
	\Bigg[
	\Bigg(
	\text{\Large\textcolor{NavyBlue}{[X=0]}},\quad
	\bigg(
	\begin{tabular}{c c}
		\text{\large\textcolor{Peach}{[y=0]}},\quad & 
		\begin{minipage}{0.14\linewidth}
			\begin{lstlisting}[language=CustomPseudoCode,numbers=none]
X := 1
yield
y := X + 1
X := 0
return y
			\end{lstlisting}
		\end{minipage}
	\end{tabular}
	\bigg)
	\Bigg)
	\rightarrow
	\Bigg(
	\text{\Large\textcolor{NavyBlue}{[X=1]}},\quad
	\bigg(
	\begin{tabular}{c c}
		\text{\large\textcolor{Peach}{[y=0]}},\quad & 
		\begin{minipage}{0.14\linewidth}
			\begin{lstlisting}[language=CustomPseudoCode,numbers=none]
y := X + 1
X := 0
return y
			\end{lstlisting}
		\end{minipage}
	\end{tabular}
	\bigg)
	\Bigg)
	\Bigg]
	\bigcup\ \ldots, \\[1em]
	\quad
	\ldots
	\bigcup\ 
	\Bigg[
	\Bigg(
	\text{\Large\textcolor{NavyBlue}{[X=1]}},\quad
	\bigg(
	\begin{tabular}{c c}
		\text{\large\textcolor{Peach}{[y=0]}},\quad & 
		\begin{minipage}{0.14\linewidth}
			\begin{lstlisting}[language=CustomPseudoCode,numbers=none]
y := X + 1
X := 0
return y
			\end{lstlisting}
		\end{minipage}
	\end{tabular}
	\bigg)
	\Bigg)
	\rightarrow
	\Bigg(
	\text{\Large\textcolor{NavyBlue}{[X=0]}},\quad
	\bigg(
	\begin{tabular}{c c}
		\text{\large\textcolor{Peach}{[y=1]}},\quad & 
		\begin{minipage}{0.11\linewidth}
			\begin{lstlisting}[language=CustomPseudoCode,numbers=none]
return y
			\end{lstlisting}
		\end{minipage}
	\end{tabular}
	\bigg)
	\Bigg)
	\Bigg]
	\bigcup\ \ldots
	\Bigg\}
\end{array}
\]

\todo{emphasize that there are states we didn't add}


