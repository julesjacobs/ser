\section{Introduction}
\label{sec:introduction}

\todo{Introduction goes here.}

\todo{We motivate the problem of deciding serializability in programmable networks.}

\todo{We talk about some related work if relevant.}

\todo{We show that it's interesting with an example.}

\todo{We describe our main results.}

\paragraph{Contributions:}
\begin{itemize}
    \item Novel notion of serializability (``atomicity'' or ``semantic serializabillity'') applicable to network systems (\Cref{sec:problem-definition,sec:related:notions-of-serializability})
    \item Decidability results (1 main theorem: \textbf{automatically proving unbounded serializability}, 2 extra theorems: ser=ser decidable, ser=int undecidable) (\Cref{sec:formal-results,sec:related:deciding-serializability})
    \item Implementation of decision procedure. Advances in semilinear sets, Petri net reachability heuristics that makes the decision procedure work. (\Cref{sec:implementation,sec:related:petri})
\end{itemize}

\newpage


\paragraph{Motivation:}

bla 1

	
	
\noindent
\begin{minipage}[t]{0.45\textwidth}
	\begin{minipage}[t]{\textwidth}
		\begin{lstlisting}[caption={Without yield or lock (serializable)}]
    request foo: 
        X := 1 
        // no yield
        y := X 
        X := 0
        return y 
		\end{lstlisting}
	\end{minipage}
	\vspace{1em}
	\begin{minipage}[t]{\textwidth}
		\begin{lstlisting}[caption={With yield (not serializable)}]
    request foo: 
        X := 1 
        yield 
        y := X + 1
        X := 0
        return y 	
		\end{lstlisting}
	\end{minipage}
\end{minipage}%
\hfill
\begin{minipage}[t]{0.45\textwidth}
	\begin{lstlisting}[caption={With yield and lock (serializable)}]
    request foo: 
        // lock
        while (L == 1): 
            yield
        L := 1 

        X := 1
        yield
        y := X 
        X := 0

        // unlock    
        L := 0

        return y 
	\end{lstlisting}
\end{minipage}

\vspace{2em}
bla 2

% Second row
\noindent
\begin{minipage}[t]{0.45\textwidth}
	\begin{lstlisting}[caption={Not serializable: {(A,0),(A,0)}}]
		request A: 
		    x := FLAG 
		
		    if (?): 
		        yield
		    // no else
		
		
		    FLAG := 1 
		    return x
	\end{lstlisting}
\end{minipage}%
\hfill
\begin{minipage}[t]{0.45\textwidth}
	\begin{lstlisting}[caption={Serializable}]
		request A: 
		    x := FLAG
		
		    if (?):
		        yield
		    else:
		        x := 1 - x
		
		    FLAG := 1
		    return x
	\end{lstlisting}
\end{minipage}

\vspace{2em}
bla 3

% Third row
\noindent
\begin{minipage}[t]{0.45\textwidth}
	\begin{lstlisting}[caption={Fred (serializable)}]
		request incr: 
		    while (X == 3):
		        yield
		        
		        
		    X := X + 1
		
		
		request decr: 
		    while (X == 0): 
		        yield
		        
		        
		    X := X - 1
	\end{lstlisting}
\end{minipage}
\hfill
\begin{minipage}[t]{0.45\textwidth}
	\begin{lstlisting}[caption={Fred2 (not serializable)}]
		request incr:
		    while (X == 3):
		        yield
		    y := X
		    yield
		    X := y + 1
		
		
		request decr: 
		    while (X == 0):
		        yield
		    y := X
		    yield
		    X := y - 1
	\end{lstlisting}
\end{minipage}
	

\begin{minipage}[t]{1.0\textwidth}
\begin{lstlisting}[caption={Snapshot-based monitor deactivation (not serializable, as it can return a sume of 0 active monitors)}]
	// initialize both monitors to be active
    N_1_ACTIVE := 1
    N_2_ACTIVE := 1
	
    request main:
        while (X == 0):
            yield
		
        // take snapshot
        n_1_active_snapshot := N_1_ACTIVE
        n_2_active_snapshot := N_2_ACTIVE
        yield
		
        if (n_1_active_snapshot == 1) and (n_2_active_snapshot == 1):
            // if both nodes active --- choose which one to deactivate 
            if (?): 
                N_1_ACTIVE := 0
            else:
                N_2_ACTIVE := 0

        return N_1_ACTIVE + N_2_ACTIVE  // total active nodes
\end{lstlisting}
\end{minipage}
	
%only 0 in non-serializable runs!

	
\newpage