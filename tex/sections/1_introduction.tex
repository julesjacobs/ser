\section{Introduction}
\label{sec:introduction}

For concurrent systems, from databases to software-defined networks (SDNs), a cornerstone correctness criterion is \emph{serializability}: every concurrent execution must produce outcomes equivalent to some serial ordering of requests. Violations of serializability can lead to subtle anomalies, such as lost updates in databases or routing cycles in SDNs.
While we can check serializability for a fixed number of transaction requests with a fixed number of execution steps by enumerating all interleavings, the problem is undecidable in the presence of unbounded state, requiring techniques such as runtime verification or incomplete bounded model checking \cite{WaSt06a,WaSt06b,FlFrYi08,FaMa08,SiMaWaGu11a,SiMaWaGu11b,Pa79,AlMcPe96,BiEn19}.

However, \citet{BoEmEnHa13} have shown (as a special case of bounded-barrier linearizability) that the problem is decidable for when the size of each request-local state and the shared global state is bounded even for an \emph{unbounded} number of requests each performing an \emph{unbounded} number of steps. The purpose of this paper is to make this theoretical decidability result a reality by designing practical algorithms that either prove serializability (generating a proof certificate) or prove non-serializability (generating a counter-example trace).
% 
We illustrate the problem by example:

% examples in Listings~\ref{lst:MotivatingExample1Ser},~\ref{lst:MotivatingExample2NonSer}, and ~\ref{lst:MotivatingExample3Ser}, written in our modeling language called Ser.

\noindent
\begin{minipage}[t]{0.55\textwidth}
	\begin{minipage}[t]{\textwidth}
		\begin{lstlisting}[caption={Without yield or lock (serializable)},
			label={lst:MotivatingExample1Ser}]
  // request handler invoked by clients          
  request main: 
      X := 1 // X is global (uppercase)
      y := X // y is local (lowercase)
      X := 0
      return y 
		\end{lstlisting}
	\end{minipage}
	\vspace{1em}
	\begin{minipage}[t]{\textwidth}
		\begin{lstlisting}[caption={With yield (not serializable)},
			label={lst:MotivatingExample2NonSer}]
  request main: 
      X := 1 
      yield // let another request run
      y := X // can read 0!
      X := 0
      return y 	
		\end{lstlisting}
	\end{minipage}
\end{minipage}%
\hfill
\begin{minipage}[t]{0.35\textwidth}
	\begin{lstlisting}[caption={With yield and lock (serializable)},
		label={lst:MotivatingExample3Ser}]
  request main: 
      // lock
      while (L == 1): 
          yield
      L := 1 

      X := 1
      yield
      y := X 
      X := 0

      // unlock    
      L := 0
      return y 
	\end{lstlisting}
\end{minipage}

These examples are written in our modeling language called Ser.
A Ser program has a set of named \textbf{request handlers} (one handler \texttt{main} in the examples) that are arbitrarily invoked concurrently by the external environment.
Each incoming request processes its request handler's body until it returns a value as its \textbf{response}. Concurrency is managed by the \textbf{yield} statement, which pauses the current request and gives other requests a chance to run. Ser programs have uppercase \textbf{global shared variables} (\texttt{X} in the examples) and lowercase \textbf{request-local variables} (\texttt{y} in the examples).



%
The first program (Listing~\ref{lst:MotivatingExample1Ser}) is clearly serializable because there are no yields, and hence, no interleavings: each \texttt{main} request returns 1.
In the second program (Listing~\ref{lst:MotivatingExample2NonSer}), the yield allows interleavings and is \emph{not} serializable; consider two concurrent requests to \texttt{main}:
\begin{enumerate}
\item Request A executes \texttt{X := 1} then yields to Request B
\item Request B executes \texttt{X := 1}, yields to itself, reads \texttt{X} (getting 1), sets \texttt{X := 0}, and returns 1
\item Request A resumes, reads \texttt{X} (now 0), and returns 0
\end{enumerate}
This produces the multiset \{(\texttt{main}, 0), (\texttt{main}, 1)\} of (request, response) pairs, which is impossible in any serial execution (where all \texttt{main} requests return 1 and never 0).
Of course, having yields does not guarantee that an execution is necessarily not serializable, as observed in the third snippet (Listing~\ref{lst:MotivatingExample3Ser}). This program uses an additional lock variable ``L'', which guarantees that even if an interleaving occurs, the program is semantically equivalent to the first one.
%
These examples demonstrate that reasoning about serializability can be complex even for very simple programs with few requests running concurrently.
\vspace{-.5em}
\paragraph{Problem Definition.}
Formally, we define the \textbf{observable execution} of a Ser program as a multiset of (request, response) pairs. The \textbf{observable behavior} of a Ser program is the set of all possible observable executions that can occur such that the requests are executed concurrently to obtain their paired responses.
A program is \textbf{serializable} if every observable behavior is achievable serially (without interleavings). That is, a Ser program is serializable if its semantics does not change when all yield statements are removed.
%
\emph{The goal of this paper is to design and develop the Ser language and decision procedure for this problem.} In particular, Ser can prove serializability \textbf{automatically} without requiring any manual proof by the user.
\vspace{-.5em}
\paragraph{Challenges.}
To our knowledge, no prior implementation exists that can automatically generate proof certificates for this class of concurrent systems.
Why not?
Our decision procedure builds on Bouajjani et al.'s reduction from serializability to Petri net reachability~\cite{BoEmEnHa13}. However, since Petri net reachability is Ackermann-complete~\cite{CzWo22}, a naive implementation would fail on all but the simplest programs. 
\vspace{-.5em}
\paragraph{Our Approach.}
To address this, we first introduce the abstraction of \textit{network systems} (NS) --- abstract concurrent programs where users send \textit{requests} that manipulate local and shared state before returning \textit{responses}. Ser programs are compiled into NS, on which our decision procedure operates via reduction to Petri net and semilinear set analysis.

As a backend solver, we use SMPT~\cite{AmDa23}, which is a state-of-the-art tool for Petri net reachability.
We note that while our approach is sound (never incorrectly claims serializability), the underlying SMPT Petri Net tool may time out on complex instances, limiting completeness in practice (which is unavoidable for any Petri Net tool, given the Ackermann-completeness of the problem).

We developed several techniques to make the approach practical, including Petri Net pruning, semilinear set compression, and additional manipulations with Presburger formulas.
These optimizations reduce the search space by orders of magnitude, enabling us to successfully verify the serializability of non-trivial programs.

We evaluated our approach on programs with features such as loops, branching, locks, and nondeterminism. Our benchmarks include SDN-inspired examples such as stateful firewalls, BGP routing, and online shopping systems.

To our knowledge, this leads to the first \emph{implemented} decision procedure that: (i) automatically \textit{proves} serializability for unbounded executions; (ii) generates \textit{proof certificates}; and (iii) handles non-trivial programs.


\paragraph{Contributions.}
After a tour of examples in \Cref{sec:tour}, we present the following contributions:
\begin{itemize}
    \item \Cref{sec:problem-definition} introduces the notion of a Network System (NS), a concurrent program abstraction that captures the essence of concurrent systems.
    \item \Cref{sec:formal-results} presents decidability results (a theorem on serializability; two on equivalence), presents the core decision procedure with proof certificates, and presents techniques for semilinear set reductions and Petri-net reductions.
    \item \Cref{sec:implementation} presents the implementation of the Ser toolchain.
    \item \Cref{sec:evaluation} presents case studies on modeling in Ser examples from domains such as SDNs and databases, and presents our extensive evaluation of the toolchain.
\end{itemize}


Finally, we discuss related work in \Cref{sec:related-work} and conclude in \Cref{sec:discussion}.
Our tool, benchmarks, and experiments are available as an anonymous artifact~\cite{ArtifactRepository} and will be permanently hosted with the paper’s final version. There is also a technical appendix accompanying this paper.