%% Appendix
\appendix

\section{Proof of Inductive Invariant}

\begin{proof}
	
Define the predicate
\[
\begin{aligned}
	I(P_{1},\dots,P_{8})
	:={}&
		(P_{1},\textcolor{black}{P_{2}},\textcolor{black}{P_{3}},P_{4},P_{5},P_{6},\textcolor{black}{P_{7}},\textcolor{black}{P_{8}})
		\;\mapsto\;\\
		&\quad
		\exists\,e_{0},\dots,e_{5}\ge0.\;
		\Bigl(
		e_{2}-e_{1}+\textcolor{black}{P_{3}}-1=0\;\land\;
		e_{2}+P_{1}-e_{5}=0\;\land\;
		P_{5}-e_{1}+e_{4}=0\;\land\\
		&\qquad\quad
		-\,e_{4}+\textcolor{black}{P_{7}}=0\;\land\;
		P_{6}+e_{3}-e_{0}=0\;\land\;
		\textcolor{black}{P_{8}}-e_{3}=0\;\land\\
		&\qquad\quad
		-\,e_{2}+e_{1}+e_{0}+P_{4}=0\;\land\;
		-\,e_{2}+e_{1}+\textcolor{black}{P_{2}}=0
		\Bigr)
		\;\land\;
		\bigl(P_{4}-1\ge0\;\lor\;\textcolor{black}{P_{3}}-1\ge0\bigr).
	\end{aligned}
	\]
	
	
	\medskip\noindent
	\textbf{(1) Initialization.}
	The initial marking has $P_{3}=1$ and $P_{1}=P_{2}=P_{4}=P_{5}=P_{6}=P_{7}=P_{8}=0$.
	Choose $e_{0}=\cdots=e_{5}=0$.  Then
	\[
	e_{i}\ge0,\quad
	e_{2}-e_{1}+P_{3}-1=0-0+1-1=0,\;\dots,\;-e_{2}+e_{1}+P_{2}=0,
	\]
	and 
	\[
	P_{4}-1\ge0\;\lor\;P_{3}-1\ge0
	\;=\;-1\ge0\;\lor\;0\ge0
	\;=\;\texttt{FALSE}\;\lor\;\texttt{TRUE}
	\;=\;\texttt{TRUE}.
	\]
	Thus $I$ holds initially.
	
	\medskip\noindent
	\textbf{(2) Consecution.}
	One checks for each transition $t_{k}$ of the Petri net that
	\[
	I(M)\;\Longrightarrow\;I\bigl(t_{k}(M)\bigr).
	\]
	In each case the same $(e_{0},\dots,e_{5})$ can be adjusted (per the SMT certificate) to show the eight equalities and the disjunction remain valid. See the attached file in our artifact showing the full proof in the standard \texttt{SMT-LIB} format.
	
	\medskip\noindent
	\textbf{(3) Refutation of the property.}
	Suppose for contradiction that both $I(P)$ and it holds that:
	\[
	\phi(P):\quad
	P_{1}=0,\;
	P_{2}\ge0,\;
	P_{3}\ge0,\;
	P_{4}=0,\;
	P_{5}=0,\;
	P_{6}=0,\;
	P_{7}=0,\;
	P_{8}\ge1.
	\] 
	
	\noindent
	From
	\[
	e_{2}-e_{1}+P_{3}-1=0
	\quad\text{and}\quad
	-e_{2}+e_{1}+P_{2}=0
	\]
	we get
	\[
	P_{2}=1-P_{3}.
	\]
	From
	\[
	P_{8}-e_{3}=0
	\quad\text{and}\quad
	P_{6}+e_{3}-e_{0}=0
	\]
	and from the assumption that $P_6=0$, we get $e_{0}=e_{3}=P_{8}$.
	
	
	\noindent
	Similarly, the invariant equalities 
	$(-\,e_{2}+e_{1}+e_{0}+P_{4}=0)$ and $(	-\,e_{2}+e_{1}+\textcolor{black}{P_{2}}=0)$
	induce
	\[
	P_{2}=P_{4}+e_{0}=P_{4}+P_{8},
	\]
	so
	\[
	P_{8}=P_2-P4=(1-P_{3})-P_{4}=1-P_{3}-0=1-P_3.
	\]
	as we also assume that $P_4=0$.


	
	
	\noindent
	But $\phi$ also gives $P_{3}\ge0$ and $P_{8}\ge1$, hence $P_{3}=0$.  
	Furthermore, as our invariant includes a conjunction with $\bigl(P_{4}-1\ge0\;\lor\;\textcolor{black}{P_{3}}-1\ge0\bigr)$, and as (if we assume by negation that the semilinear set is reachable), thus $P_4=0$, and in order for both the invariant and the property to hold, then necessarily $P_3 \ge 1$, in contradiction with $P_3=0$.
	%
	  Thus $I\land\phi$ is unsatisfiable, i.e., 
	%\[
	$
	I(P)\;\Longrightarrow\;\neg\phi(P)$
	.
	%\]
	This completes the proof that $I$ is an inductive invariant refuting the given property.
\end{proof}


\newpage