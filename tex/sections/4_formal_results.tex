\section{Formal results}
\label{sec:formal-results}

%\begin{itemize}
%	\item Results that we rely on (petri nets, semilinear sets)
%	\item The base algorithm described in math (without any optimizations)
%	\item Time complexity
%	\item proof for correctness of bidirectional pruning
%	\item mathematical description of optimizations
%\end{itemize}


\subsection{Background}

\paragraph{Petri nets.}
A \emph{Petri net} is a tuple
\[
N = (P, T, \mathsf{pre}, \mathsf{post}, M_0),
\]
where \(P\) is a finite set of \emph{places}, \(T\) a finite set of \emph{transitions},
\(\mathsf{pre},\mathsf{post}:T\to\mathbb N^P\) the input/output vectors, and
\(M_0\in\mathbb N^P\) the initial marking.  A marking \(M\) is \emph{reachable} if
\[
M_0 \;\xrightarrow{\sigma}\; M
\quad\text{for some}\;\sigma\in T^*.
\]
%
A transition \(t\in T\) is \emph{enabled} at a marking \(M\in\mathbb N^P\) iff
\[
M \;\ge\; \mathsf{pre}(t)
\quad(\text{coordinate‐wise}).
\]
Firing \(t\) yields the new marking
\[
M' \;=\; M \;-\;\mathsf{pre}(t)\;+\;\mathsf{post}(t).
\]

%
The \emph{reachability problem} asks, given $(N,M)$, whether $M$ is reachable from $M_0$.  Surprisingly, even in the
\emph{unbounded} setting (where places may hold arbitrarily many tokens) this problem is
decidable~\cite{Ma81,Ko82,La92}, although with
Ackermann-complete~\cite{CzWo22} complexity.
%
An example for a toy Petri Net, and both reachable and unreachable markings, appears in Appendix~\ref{appendix:toyPN}.

\paragraph{Semilinear sets and Park’s theorem.}
A set \(S\subseteq\mathbb N^k\) is \emph{semilinear} iff
\[
S \;=\; \bigcup_{i=1}^m \Bigl\{\mathbf b_i + \sum_{j=1}^{r_i} n_j\,\mathbf p_{i,j}
\;\Big|\; n_j\in\mathbb N\Bigr\}.
\]

For $b_i, p_{i,j}\in \mathbb N^k $ being k-dimensional vectors of non-negative values.
%
Semilinear sets coincide exactly with the sets definable in \textit{Presburger arithmetic}~\cite{Pr29}.
%
By Parikh's theorem~\cite{Parikh66}, the Parikh Image of any context-free language is semilinear, and there is an effective construction mapping each word in the language to a
finite description of its Parikh Image.

\paragraph{Deciding serializability in unbounded systems.}

Finally, we build upon the theoretical findings of Bouajjani et al.~\cite{BoEmEnHa13} which prove the decidability of serializability in the unbounded case (originally referred to as a type of \textit{k-bounded-barrier linearizability}): $	\mathrm{Int}(\mathcal{N}) \;=\; \mathrm{Ser}(\mathcal{N})$.



\subsection{The Algorithm (without Optimizations)}

Let \(\mathcal S\) be the parsed Network System, with set of global states \(G\).  

\begin{enumerate}
	\item  \textbf{Serializability automaton.}  
	We define an NFA
	
\[
\mathcal A_{\mathrm{ser}}(\mathcal S)
= \bigl(Q,\Sigma,\delta,q_0,F\bigr),
\quad
Q = G
\quad\text{(the set of the NS’s global states)}
\]
\[
\Sigma
= \Bigl\{
{\color{ForestGreen}\blacklozenge_{\mathit{req}}}\,/\,%
{\color{red}\blacklozenge_{\mathit{resp}}}
\;\Big|\;
{\color{ForestGreen}\blacklozenge_{\mathit{req}}}\in\mathit{Req},\;
{\color{red}\blacklozenge_{\mathit{resp}}}\in\mathit{Resp}
\Bigr\}
\]
\[
\delta \;\subseteq\; Q \times \Sigma_{	{\color{ForestGreen}\blacklozenge_{\mathit{req}}}\,/\,%
	{\color{red}\blacklozenge_{\mathit{resp}}}} \times Q,
\quad
\bigl(q \xrightarrow{%
	{\color{ForestGreen}\blacklozenge_{\mathit{req}}}\,/\,%
	{\color{red}\blacklozenge_{\mathit{resp}}}%
} q'\bigr)
\;\Longleftrightarrow\;
\begin{array}{l}
	\mathcal S \text{ at global state } q
	\;\text{issues}\;
	{\color{ForestGreen}\blacklozenge_{\mathit{req}}},\\[0.5ex]
	\text{after completion of the request, it receives}\;
	{\color{red}\blacklozenge_{\mathit{resp}}},\\[0.5ex]
	\text{arriving at global state } q'
\end{array}
\]

	
	(i.e.\ each transition is a request/response pair).  Its language
	\(L(\mathcal A_{\mathrm{ser}})\subseteq\Sigma^*\) is exactly the set of serial
	request/response traces, and we define the semilinear set of serial executions
	\[
	\mathsf{Ser}(\mathcal S)
	\;=\;
	\mathsf{Parikh}\bigl(L(\mathcal A_{\mathrm{ser}})\bigr)
	\;\subseteq\;\mathbb N^{|\Sigma|}.
	\]
	
	\item 
	\textbf{Interleaving Petri Net.}
	
	We build
	\[
	N_{\mathrm{int}}(\mathcal S)
	= (P,\,T,\,\mathsf{pre},\,\mathsf{post},\,M_0),
	\]
	where
	\[
	P
	=
	P_G \;\cup\; P_{L} \;\cup\; P_{Req,Resp}
	\]
	
	for 
	\[
	P_G 
	= \{\,p_g \mid g\in G\}
	\quad 
	P_L 
	= \{\,p_\ell \mid \ell\in L\}
	\quad
	P_{Req,Resp}=
	\{\,p_{{\color{ForestGreen}\blacklozenge_{\mathit{req}}}/{\color{red}\blacklozenge_{\mathit{resp}}}} \mid
	{\color{ForestGreen}\blacklozenge_{\mathit{req}}}\in \mathit{Req}, {\color{red}\blacklozenge_{\mathit{resp}}}\in \mathit{Resp}\},
	\]
	with \(G\) the global states, \(L\) the in‐flight local states (request‐state + remaining program), and \(\mathit{Resp}\) the response labels.
	
	Transitions are partitioned as
	\[
	T = T_{\mathit{req}} \;\cup\; T_{\delta}\;\cup\;T_{\mathit{resp}},
	\]
	where
	\[
	T_{\mathit{req}} = \{\,t_{{\color{ForestGreen}\blacklozenge_{\mathit{req}}}} \mid {\color{ForestGreen}\blacklozenge_{\mathit{req}}}\in\mathit{Req}\},\quad
	T_{\delta} = \{\,t_{(\ell,g)\to(\ell',g')} \mid (\ell,g)\xrightarrow{}(\ell',g')\in\delta_{\mathcal S}\},\quad
	T_{\mathit{resp}} = \{\,t_{{\color{red}\blacklozenge_{\mathit{resp}}}} \mid {\color{red}\blacklozenge_{\mathit{resp}}}\in\mathit{Resp}\}.
	\]
	
	Their pre/post vectors are:
	\[
	\begin{alignedat}{3}
		\mathsf{pre}\bigl(t_{{\color{ForestGreen}\blacklozenge_{\mathit{req}}}}\bigr)
		&= \mathbf0, &
		\mathsf{post}\bigl(t_{{\color{ForestGreen}\blacklozenge_{\mathit{req}}}}\bigr)
		&= \mathbf1_{p_{\ell_r}}, 
		&&\text{for each }{\color{ForestGreen}\blacklozenge_{\mathit{req}}}\in\mathit{Req},\\
		\mathsf{pre}\bigl(t_{(\ell,g)\to(\ell',g')}\bigr)
		&= \mathbf1_{p_\ell} + \mathbf1_{p_g}, &
		\mathsf{post}\bigl(t_{(\ell,g)\to(\ell',g')}\bigr)
		&= \mathbf1_{p_{\ell'}} + \mathbf1_{p_{g'}}, 
		&&\text{for each }(\ell,g)\!\to\!(\ell',g')\in\delta_{\mathcal S},\\
		\mathsf{pre}\bigl(t_{{\color{red}\blacklozenge_{\mathit{resp}}}}\bigr)
		&= \mathbf1_{p_\ell}, &
		\mathsf{post}\bigl(t_{{\color{red}\blacklozenge_{\mathit{resp}}}}\bigr)
		&= \mathbf1_{p_{{\color{ForestGreen}\blacklozenge_{\mathit{req}}}/{\color{red}\blacklozenge_{\mathit{resp}}}}}, 
		&&\text{for each }{\color{red}\blacklozenge_{\mathit{resp}}}\in\mathit{Resp}\ (\ell\text{ the matching local state).}
	\end{alignedat}
	\]
	
	Finally, the initial marking is
	\[
	M_0(p_{g_0}) = 1,
	\quad
	M_0(p) = 0 \text{ for all }p\neq p_{g_0},
	\]
	where \(g_0\) is the initial global state of \(\mathcal S\).  
	
	
	
	Define the projection of all reachable marking to the places representing completed request/response pairs:
	\[
	\pi \;:\;\mathbb N^P \;\longrightarrow\;\mathbb N^{P_R}
	\quad\bigl(\pi(M)\bigr)(p_{{{\color{ForestGreen}\blacklozenge_{\mathit{req}}}/{\color{red}\blacklozenge_{\mathit{resp}}}}})\;=\;M(p_{{{\color{ForestGreen}\blacklozenge_{\mathit{req}}}/{\color{red}\blacklozenge_{\mathit{resp}}}}})\text{ for }p_{{{\color{ForestGreen}\blacklozenge_{\mathit{req}}}/{\color{red}\blacklozenge_{\mathit{resp}}}}}\in P_{Req,Resp}.
	\]
	Then the set of all  ${{\color{ForestGreen}\blacklozenge_{\mathit{req}}}/{\color{red}\blacklozenge_{\mathit{resp}}}}$ pairs of the NS, attained by any interleaving, is:
	\[
	\mathsf{Int}(\mathcal S)
	\;=\;
	\bigl\{\;\pi(M)\;\bigm|\;M_0 \xrightarrow{*} M\text{ in }N_{\mathrm{int}}(\mathcal S)\bigr\}.
	\]
	
	\item \textbf{Non-serializable set.}  
	Let
	\(\;\mathsf{NonSer}(\mathcal S)=\mathbb N^{|\Sigma|}\setminus \mathsf{Ser}(\mathcal S)\), i.e., all ${{\color{ForestGreen}\blacklozenge_{\mathit{req}}}/{\color{red}\blacklozenge_{\mathit{resp}}}}$ pairs that \textit{cannot} be attained via a serializable execution.
	
	\item \textbf{Reachability encoding.}  
	Ask whether there exists a marking \(M\) of \(N_{\mathrm{int}}(\mathcal S)\) such that
	\[
	M_0 \xrightarrow{*} M
	\quad\wedge\quad
	\pi(M)\in \mathsf{NonSer}(\mathcal S).
	\]
	
	\item \textbf{Decision.}  
	\begin{itemize}
		\item \sat: yields a counterexample interleaving \(M\) with
		\(\pi(M)\notin \mathsf{Ser}(\mathcal S)\).
		\item \unsat: yields an inductive (unbounded) invariant of
		\(N_{\mathrm{int}}\), which back-translates to a proof of
		serializability for \(\mathcal S\).
	\end{itemize}



\item \textbf{Validation.}  
	\begin{itemize}
	\item if \sat we validate the reachable trace and project to to the NS semantics to represent a valid interleaving that results into request/response pairs which cannot be attained in serial executions.
	
	\item if \unsat:
	we generate an inductive invariant over the interleaving Petri Net, which proves the semilinear set cannot be attained via an interleaving.
\end{itemize}
\end{enumerate}


%\guy{should we add/prove the following?}
%
%\begin{proposition}
%	Let $N_{\mathrm{int}}(\mathcal S)=(P,T,\mathsf{pre},\mathsf{post},M_0)$ be the interleaving Petri net constructed above, and let
%	\[
%	\pi\colon\mathbb N^P\to\mathbb N^{P_R}
%	\]
%	be the projection onto the request/response places $P_R$.  Then
%	\[
%	\mathsf{Int}(\mathcal S)
%	\;=\;
%	\{\;\pi(M)\;\mid\;M_0\xrightarrow{*}M\}.
%	\]
%\end{proposition}


\subsection{Time/Space Complexity}

\todo{consult with Jules}


\subsection{Optimizations}
\label{sec:optimizations}

\todo{check}

We apply four optimizations to the base algorithm to control intermediate blow‐up.

\paragraph{(1) Bidirectional pruning.}  
Let \(N=(P,T,\mathsf{pre},\mathsf{post},M_0)\) and let \(\mathit{Tgt}\subseteq P\) be the “nonzero” places extracted from the Presburger constraints.  Define the flip of \(N\), \(\overline N\), by swapping \(\mathsf{pre}\) and \(\mathsf{post}\).  We compute a decreasing sequence of transition‐sets
\[
T^{(0)} = T,\qquad
\begin{aligned}
	R_{\mathrm{fw}}^{(k)} &= \{\,t\in T^{(k)} \mid \exists\,M_0\xrightarrow{*}M\text{ in }N\text{ enabling }t\},\\
	R_{\mathrm{bw}}^{(k)} &= \{\,t\in T^{(k)} \mid \exists\,M\xrightarrow{*}M'\text{ in }\overline N\text{ with }M'(p)>0\;(p\in\mathit{Tgt})\},\\
	T^{(k+1)} &= R_{\mathrm{fw}}^{(k)} \;\cap\; R_{\mathrm{bw}}^{(k)}.
\end{aligned}
\]
We repeat until \(T^{(k+1)}=T^{(k)}\).  In each iteration we record
\(\Delta_{\mathrm{fw}}^{(k)}=R_{\mathrm{fw}}^{(k-1)}\setminus R_{\mathrm{fw}}^{(k)}\)
and
\(\Delta_{\mathrm{bw}}^{(k)}=R_{\mathrm{bw}}^{(k-1)}\setminus R_{\mathrm{bw}}^{(k)}\).
%
This optimization allows to significant;y prune the Petri Net while still over-approximating all reachable markings (see the proof in Appendix~\ref{appendix:BidirectionalProof}).

\paragraph{(2) Semilinear‐set pruning.}  
Recall a semilinear set is \(S=\bigcup_{i=1}^m L_i\) with
\(\displaystyle L_i=\{\,b_i+\sum_{p\in P_i}n_p\,p\mid n_p\in\mathbb N\}\).  
%
We replace each period‐basis \(P_i\) by
\[
P_i \;:=\;\{\,p\in P_i \mid p\notin\mathsf{Span}(P_i\setminus\{p\})\},
\]
dropping any ``redundant” period, and remove any \(L_j\subseteq L_i\) for \(i\neq j\), iterating to a fixpoint so no two components subsume one another.

\paragraph{(3) Generating fewer constraints.}  
Let $\mathrm{comp}(S)=\{L_1,\dots,L_m$ 
be the multiset of linear components of the semilinear set 
\(\displaystyle S=\bigcup_{i=1}^m L_i\), where each 
\(\;L_i=b_i+\langle P_i\rangle\) with \(b_i\in\mathbb N^d\) and 
\(P_i\subseteq\mathbb N^d\).  Define the pruning operator
\[
\mathrm{new}(\mathcal C)
\;=\;
\bigcup\bigl\{\,L\in\mathcal C \;\bigm|\;\nexists\,L'\in\mathcal C\setminus\{L\}:\;L'\subsetneq L\bigr\},
\]
which removes any component strictly containing another.  

Then we replace the naïve semilinear‐set operations (and their semilinear ``meaning'') by
\[
S\;+\;T
\;=\;
\mathrm{new}\bigl(\mathrm{comp}(S)\,\cup\,\mathrm{comp}(T)\bigr),
\]
\[
S\;\cdot\;T
\;=\;
\mathrm{new}\Bigl(\{\,L_i\cdot L'_j \mid L_i\in\mathrm{comp}(S),\;L'_j\in\mathrm{comp}(T)\}\Bigr),
\]
where for
\(\;L_i=b_i+\langle P_i\rangle,\;L'_j=b'_j+\langle P'_j\rangle\) we set
$
L_i\cdot L'_j
=\;(b_i+b'_j)\;+\;\langle\,P_i\cup P'_j\,\rangle.
$
Finally, for Kleene‐star and plus on the regex side one similarly applies
\(\mathrm{new}(\cdot)\) to the collection of “folded” components instead of
building all intermediate ones:
\[
S^*
=\mathrm{new}\Bigl(\bigcup_{k\ge0}\bigl(\mathrm{comp}(S)\bigr)^k\Bigr),
\qquad
S^+
=S\cdot S^*.
\]

\paragraph{(4) Strategic Kleene elimination order.}  
When converting an NFA \(\mathcal A=(Q,\Sigma,\delta,q_0,F)\) to a regex by repeated state‐elimination, then instead of choosing an arbitrary state $q\in Q'$, we pick the next state
\[
q^* = 
	\arg\min_{q\in Q'}\bigl(|\delta_{\mathrm{in}}(q)|+|\delta_{\mathrm{out}}(q)|\bigr)
\]


where \(Q'\subseteq Q\) are the remaining states remaining to be eliminated.  This heuristic keeps the intermediate regex small.


%\newpage
