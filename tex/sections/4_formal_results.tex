\section{Formal results}
\label{sec:formal-results}

\begin{itemize}
	\item Results that we rely on (petri nets, semilinear sets)
	\item The base algorithm described in math (without any optimizations)
	\item Time complexity
	\item proof for correctness of bidirectional pruning
	\item mathematical description of optimizations
\end{itemize}


\subsection{Background}

\paragraph{Petri nets.}
A \emph{Petri net} is a tuple
\[
N = (P, T, \mathsf{pre}, \mathsf{post}, M_0),
\]
where \(P\) is a finite set of \emph{places}, \(T\) a finite set of \emph{transitions},
\(\mathsf{pre},\mathsf{post}:T\to\mathbb N^P\) the input/output vectors, and
\(M_0\in\mathbb N^P\) the initial marking.  A marking \(M\) is \emph{reachable} if
\[
M_0 \;\xrightarrow{\sigma}\; M
\quad\text{for some}\;\sigma\in T^*.
\]
%
The \emph{reachability problem} asks, given $(N,M)$, whether $M$ is reachable from $M_0$.  In the
\emph{unbounded} setting (where places may hold arbitrarily many tokens) this problem is
decidable~\cite{Ma81,Ko82,La92}, and with
Ackermann-complete~\cite{CzWo22} complexity.

\paragraph{Semilinear sets and Park’s theorem.}
A set \(S\subseteq\mathbb N^k\) is \emph{semilinear} iff
\[
S \;=\; \bigcup_{i=1}^m \Bigl\{\mathbf b_i + \sum_{j=1}^{r_i} n_j\,\mathbf p_{i,j}
\;\Big|\; n_j\in\mathbb N\Bigr\}.
\]

Semilinear sets coincide exactly with the sets definable in Presburger arithmetic~\cite{Pr29}.
%
By Parkh's theorem~\cite{Parikh66}, the Parikh Image of any context-free language is semilinear, and there is an effective construction mapping each word in the language to a
finite description of its Parikh Image.

\subsection{Basic Algorithm (without Optimizations)}

Let \(\mathcal S\) be the parsed Network System, with set of global states \(G\).  

\begin{enumerate}
	\item \textbf{Serializability automaton.}  
	We define an NFA
	\[
	\mathcal A_{\mathrm{ser}}(\mathcal S)
	= \bigl(Q,\Sigma,\delta,q_0,F\bigr)
	\quad\text{where }Q = G,
	\]
	\(\Sigma\) is the set of request/response actions, and
	\((q \xrightarrow{a} q')\in\delta\) iff \(\mathcal S\) can move from global
	state \(q\) to \(q'\) via action \(a\) in a purely serial execution.
	
	Its language \(L(\mathcal A_{\mathrm{ser}})\subseteq\Sigma^*\) is exactly
	the set of serial traces, and we define
	\[
	\mathsf{Ser}(\mathcal S)
	\;=\;
	\Parikh\bigl(L(\mathcal A_{\mathrm{ser}})\bigr)
	\;\subseteq\;\mathbb N^{|\Sigma|}.
	\]
	
	\item \textbf{Interleaving Petri net.}  
	Construct
	\[
	N_{\mathrm{int}}(\mathcal S)
	= (P, T, \mathsf{pre}, \mathsf{post}, M_0),
	\]
	whose transitions correspond to individual requests/responses of \(\mathcal S\),
	and whose places are partitioned
	\[
	P = P_G \;\cup\; P_R \;\cup\; P_{\mathrm{aux}},
	\]
	where \(P_G\) encodes the current global state and \(P_R\) records the multiset
	of responses seen so far.  By construction, every marking \(M\) reachable in
	\(N_{\mathrm{int}}\) corresponds to some interleaved execution of \(\mathcal S\).
	
	Define the projection
	\[
	\pi \;:\;\mathbb N^P \;\longrightarrow\;\mathbb N^{P_R}
	\quad\bigl(\pi(M)\bigr)(p)\;=\;M(p)\text{ for }p\in P_R.
	\]
	Then the set of all interleaving Parikh-vectors is
	\[
	\mathsf{Int}(\mathcal S)
	\;=\;
	\bigl\{\;\pi(M)\;\bigm|\;M_0 \xrightarrow{*} M\text{ in }N_{\mathrm{int}}(\mathcal S)\bigr\}.
	\]
	
	\item \textbf{Non-serializable set.}  
	Let
	\(\;\mathsf{NonSer}(\mathcal S)=\mathbb N^{|\Sigma|}\setminus \mathsf{Ser}(\mathcal S)\).
	
	\item \textbf{Reachability encoding.}  
	Ask whether there exists a marking \(M\) of \(N_{\mathrm{int}}(\mathcal S)\) such that
	\[
	M_0 \xrightarrow{*} M
	\quad\wedge\quad
	\pi(M)\in \mathsf{NonSer}(\mathcal S).
	\]
	
	\item \textbf{Decision.}  
	\begin{itemize}
		\item \sat: yields a counterexample interleaving \(M\) with
		\(\pi(M)\notin \mathsf{Ser}(\mathcal S)\).
		\item \unsat: yields an inductive (unbounded) invariant of
		\(N_{\mathrm{int}}\), which back-translates to a proof of
		serializability for \(\mathcal S\).
	\end{itemize}
\end{enumerate}

\newpage
