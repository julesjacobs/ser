\section{Evaluation}
\label{sec:evaluation}

\begin{enumerate}
    \item Benchmarks (describe our benchmarks)
    \item Results (total time, split out SMPT time from our rust code time)
    \item Analysis of optimizations (how much time they save, petri net sizes, semilinear set sizes)
    \item Limitations (examples we cannot solve, future work that would help)
\end{enumerate}


\noindent
\textbf{Experimentation details.}
All experiments ran on a second generation Lenovo Notebook ThinkPad P16s, with 16 AMD cores and a RAM of 58 GB, with a Linux Ubuntu 24.04.2 operating system.
%
We plan on making all our code, benchmarks raw results publicly available with the final version of this paper.
 


\subsection{Benchmarks Overview}
\label{subsec:benchmarks}

\begin{itemize}
	\item \textbf{Core expressions \& multi request workflows}: Benchmarks testing arithmetic, boolean, and simple control expression.
	\item \textbf{Fred (mixed arithmetic)}: Mixed control and arithmetic transformations (Fred series).
	\item \textbf{Stop (circular-increment) series}: Circular increment loops and variants.
	\item \textbf{Concurrency \& locking loops}: Concurrent looping patterns with locking and tricky interactions.
	\item \textbf{Non-deterministic choice \& randomness}: Random choice and non-deterministic branching benchmarks.
	\item \textbf{Networking \& system protocols}: Networking protocols and system-level monitoring.
	\item \textbf{JSON state-machine examples}: Example JSON-encoded state machine workflows.
\end{itemize}


\subsection{Results}

%\FloatBarrier      
%
%\begin{table}[H]
%	\centering
%	\small
%	% increase horizontal padding between columns
%	\setlength{\tabcolsep}{5pt}
%	\renewcommand{\arraystretch}{0.9}
%	\begin{tabular*}{\textwidth}{@{\extracolsep{\fill}}%
%			p{2cm}   % Category
%			p{1.5cm} % Benchmark
%			c c c c c c % Features: If, While, ?, Arith, Yield, Multi-req
%			r r       % certification creation, total time
%			c         % Serializable
%		}
%		\toprule
%		\multicolumn{2}{c}{\textbf{Benchmark}}
%		& \multicolumn{6}{c}{\textbf{Features}}
%		& \multicolumn{2}{c}{\textbf{Runtime (ms)}}
%		& \textbf{Serializable} \\
%		\cmidrule(lr){3-8} \cmidrule(lr){9-10}
%		& 
%		& If & While & \texttt{?} & Arith & Yield & Multi-req
%		& Cert. & Total
%		& \\
%		\midrule
%		
%		\multirow{7}{=}{Core expressions}
%		& \texttt{a1.ser} &  & \cmark &  &  &       &  & –  & –   & \cmark \\
%		& \texttt{a2.ser} &  &        &  &  & \cmark &  & –  & –   &       \\
%		& \texttt{a3.ser} &  &        &  &  &       &  & –  & –   & \cmark \\
%		& \texttt{a4.ser} &  &        &  &  & \cmark & \cmark & –  & –   & \cmark \\
%		& \texttt{a5.ser} &  & \cmark &  &  & \cmark & \cmark & –  & –   &       \\
%		& \texttt{a6.ser} &  &        &  &  & \cmark & \cmark & –  & –   &       \\
%		& \texttt{a7.ser} & \cmark & \cmark &  &  & \cmark &  & –  & –   &       \\
%		\midrule
%		
%		\multirow{4}{=}{State machines}
%		& \texttt{b1.json} & \cmark &        &  &  &    \cmark   & \cmark & –  & –   &       \\
%		& \texttt{b2.json} & \cmark &        &  &  &   \cmark    & \cmark & –  & –   &       \\
%		& \texttt{b3.json} & \cmark &        &  &  &    \cmark   & \cmark & –  & –   &       \\
%		& \texttt{b4.json} & \cmark &        &  &  &   \cmark    & \cmark & –  & –   &       \\
%		\midrule
%		
%		\multirow{8}{=}{Fred (mixed arithmetic)}
%		& \texttt{c1.ser} &  & \cmark &  & \cmark & \cmark & \cmark & –  & –   & \cmark \\
%		& \texttt{c2.ser} &  & \cmark &  & \cmark & \cmark & \cmark & –  & –   & \cmark \\
%		& \texttt{c3.ser} &  & \cmark &  & \cmark & \cmark & \cmark & –  & –   & \cmark \\
%		& \texttt{c4.ser} &  & \cmark &  & \cmark & \cmark & \cmark & –  & –   & \cmark \\
%		& \texttt{c5.ser} &  & \cmark &  & \cmark & \cmark & \cmark & –  & –   & \cmark \\
%		& \texttt{c6.ser} &  & \cmark &  & \cmark & \cmark & \cmark & –  & –   & \cmark \\
%		& \texttt{c7.ser} &  & \cmark &  & \cmark & \cmark & \cmark & –  & –   & \cmark \\
%		& \texttt{c8.ser} &  & \cmark &  & \cmark & \cmark & \cmark & –  & –   & \cmark \\
%		\midrule
%		
%		\multirow{7}{=}{Circular increment}
%		& \texttt{d1.ser} & \cmark &  & \cmark &  &  \cmark &  & –  & –   & \cmark \\
%		& \texttt{d2.ser} & \cmark & \cmark & \cmark &  & \cmark &  & –  & –   & \cmark \\
%		& \texttt{d3.ser} & \cmark &        & \cmark &  &   \cmark &  & –  & –   & \cmark \\
%		& \texttt{d4.ser} & \cmark &        & \cmark &  &   \cmark &  & –  & –   & \cmark \\
%		& \texttt{d5.ser} & \cmark & \cmark & \cmark &  &  \cmark &  & –  & –   & \cmark \\
%		& \texttt{d6.ser} & \cmark & \cmark & \cmark &  &     \cmark &  & –  & –   & \cmark \\
%		& \texttt{d7.ser} & \cmark &        &  &  & \cmark &  & –  & –   & \cmark \\
%		\midrule
%		
%		\multirow{8}{=}{Concurrency \& locking loops}
%		& \texttt{e1.ser} &  & \cmark &  &  & \cmark &  & –  & –   & \cmark \\
%		& \texttt{e2.ser} & \cmark & \cmark &  & \cmark & \cmark & \cmark & –  & –   & \cmark \\
%		& \texttt{e3.ser} & \cmark & \cmark &  & \cmark &   \cmark & \cmark & –  & –   & \cmark \\
%		& \texttt{e4.ser} & \cmark & \cmark &  &  \cmark &   \cmark & \cmark & –  & –   & \cmark \\
%		& \texttt{e5.ser} & \cmark & \cmark &  & \cmark &  \cmark & \cmark & –  & –   & \cmark \\
%		& \texttt{e6.ser} &  \cmark & \cmark & \cmark &  & \cmark &  & –  & –   & \cmark \\
%		& \texttt{e7.ser} & \cmark & \cmark & \cmark &  & \cmark &  & –  & –   & \cmark \\
%		& \texttt{e8.ser} &  & \cmark &  &  &   \cmark & & –  & –   &       \\
%		\midrule
%		
%		\multirow{9}{=}{Non-deterministic \& randomness}
%		& \texttt{f1.ser} & \cmark &    \cmark    & \cmark &  & \cmark &  & –  & –   & \cmark \\
%		& \texttt{f2.ser} & \cmark &   \cmark     & \cmark &  & \cmark &  & –  & –   & \cmark \\
%		& \texttt{f3.ser} &  &        &  & \cmark &   \cmark & \cmark & –  & –   & \cmark \\
%		& \texttt{f4.ser} &  &     \cmark   &  & \cmark & \cmark & \cmark & –  & –   & \cmark \\
%		& \texttt{f5.ser} & \cmark &        & \cmark &  &       &  & –  & –   &       \\
%		& \texttt{f6.ser} & \cmark &        & \cmark &  & \cmark &  & –  & –   &       \\
%		& \texttt{f7.ser} & \cmark &        & \cmark &  &  \cmark &  & –  & –   &       \\
%		& \texttt{f8.ser} & \cmark &        & \cmark &  &   \cmark &  & –  & –   &       \\
%		& \texttt{f9.ser} & \cmark &        & \cmark &  &  \cmark &  & –  & –   &       \\
%		\midrule
%		
%		\multirow{7}{=}{Networking \& system protocols}
%		& \texttt{g1.ser} & \cmark & \cmark &  & \cmark & \cmark & \cmark & –  & –   & \cmark \\
%		& \texttt{g2.ser} & \cmark & \cmark &  & \cmark & \cmark & \cmark & –  & –   & \cmark \\
%		& \texttt{g3.ser} & \cmark & \cmark & \cmark & \cmark & \cmark & \cmark & –  & –   & \cmark \\
%		& \texttt{g4.ser} & \cmark & \cmark & \cmark & \cmark & \cmark & \cmark & –  & –   & \cmark \\
%		& \texttt{g5.ser} & \cmark & \cmark & \cmark & \cmark &   \cmark & \cmark & –  & –   & \cmark \\
%		& \texttt{g6.ser} & \cmark &        & \cmark & \cmark & \cmark &  & –  & –   & \cmark \\
%		& \texttt{g7.ser} & \cmark &        & \cmark & \cmark &       &  & –  & –   & \cmark \\
%		\bottomrule
%	\end{tabular*}
%	\caption{Overview of benchmarks with combined categories and updated serializability markings.}
%\label{tab:benchmarks-all}
%\end{table}
%

\begin{table}[H]
	\centering
	\small
	% increase horizontal padding between columns
	\setlength{\tabcolsep}{5pt}
	\renewcommand{\arraystretch}{0.9}
	\begin{tabular*}{\textwidth}{@{\extracolsep{\fill}}%
			p{2cm}   % Category
			p{1.5cm} % Benchmark
			c        % Serializable (now here)
			c c c c c c % Features: If, While, ?, Arith, Yield, Multi-req
			r r       % certification creation, total time
		}
		\toprule
		\multicolumn{2}{c}{\textbf{Benchmark}}
		& \textbf{Serializable}
		& \multicolumn{6}{c}{\textbf{Features}}
		& \multicolumn{2}{c}{\textbf{Runtime (ms)}} \\
		\cmidrule(lr){1-2} \cmidrule(lr){3-3} \cmidrule(lr){4-9} \cmidrule(lr){10-11}
		& 
		&
		& If & While & \texttt{?} & Arith & Yield & Multi-req
		& Cert. & Total \\
		\midrule
		
		\multirow{7}{=}{Core expressions}
		& \texttt{a1.ser} &      &  & \cmark &  &  &       &  & –  & –   \\
		& \texttt{a2.ser} &      &  &        &  &  & \cmark &  & –  & –   \\
		& \texttt{a3.ser} &      &  &        &  &  &       &  & –  & –   \\
		& \texttt{a4.ser} &      &  &        &  &  & \cmark & \cmark & –  & –   \\
		& \texttt{a5.ser} &      &  & \cmark &  &  & \cmark & \cmark & –  & –   \\
		& \texttt{a6.ser} &      &  &        &  &  & \cmark & \cmark & –  & –   \\
		& \texttt{a7.ser} &      & \cmark & \cmark &  &  & \cmark &  & –  & –   \\
		\midrule
		
		\multirow{4}{=}{State machines}
		& \texttt{b1.json} &      & \cmark &        &  &  & \cmark   & \cmark & –  & –   \\
		& \texttt{b2.json} &      & \cmark &        &  &  & \cmark    & \cmark & –  & –   \\
		& \texttt{b3.json} &      & \cmark &        &  &  & \cmark   & \cmark & –  & –   \\
		& \texttt{b4.json} &      & \cmark &        &  &  & \cmark    & \cmark & –  & –   \\
		\midrule
		
		\multirow{8}{=}{Fred (mixed arithmetic)}
		& \texttt{c1.ser} &      &  & \cmark &  & \cmark & \cmark & \cmark & –  & –   \\
		& \texttt{c2.ser} &      &  & \cmark &  & \cmark & \cmark & \cmark & –  & –   \\
		& \texttt{c3.ser} &      &  & \cmark &  & \cmark & \cmark & \cmark & –  & –   \\
		& \texttt{c4.ser} &      &  & \cmark &  & \cmark & \cmark & \cmark & –  & –   \\
		& \texttt{c5.ser} &      &  & \cmark &  & \cmark & \cmark & \cmark & –  & –   \\
		& \texttt{c6.ser} &      &  & \cmark &  & \cmark & \cmark & \cmark & –  & –   \\
		& \texttt{c7.ser} &      &  & \cmark &  & \cmark & \cmark & \cmark & –  & –   \\
		& \texttt{c8.ser} &      &  & \cmark &  & \cmark & \cmark & \cmark & –  & –   \\
		\midrule
		
		\multirow{5}{=}{Circular increment}
		& \texttt{d1.ser} &      & \cmark & \cmark & \cmark &  & \cmark &  & –  & –   \\
		& \texttt{d2.ser} &      & \cmark &        & \cmark &  &   \cmark &  & –  & –   \\
		& \texttt{d3.ser} &      & \cmark & \cmark & \cmark &  &  \cmark &  & –  & –   \\
		& \texttt{d4.ser} &      & \cmark & \cmark & \cmark &  &     \cmark &  & –  & –   \\
		& \texttt{d5.ser} &      & \cmark &        &  &  & \cmark &  & –  & –   \\
		\midrule
		
		\multirow{7}{=}{Concurrency \& locking loops}
		& \texttt{e1.ser} &      &  & \cmark &  &  & \cmark &  & –  & –   \\
		& \texttt{e2.ser} &      & \cmark & \cmark &  & \cmark & \cmark & \cmark & –  & –   \\
		& \texttt{e3.ser} &      & \cmark & \cmark &  & \cmark &   \cmark & \cmark & –  & –   \\
		& \texttt{e4.ser} &      & \cmark & \cmark &  &  \cmark &   \cmark & \cmark & –  & –   \\
		& \texttt{e5.ser} &      & \cmark & \cmark & \cmark &  & \cmark &  & –  & –   \\
		& \texttt{e6.ser} &      & \cmark & \cmark & \cmark &  & \cmark &  & –  & –   \\
		& \texttt{e7.ser} &      &  & \cmark &  &  &   \cmark &  & –  & –   \\
		\midrule
		
		\multirow{9}{=}{Non-deterministic \& randomness}
		& \texttt{f1.ser} &      & \cmark &    \cmark    & \cmark &  & \cmark &  & –  & –   \\
		& \texttt{f2.ser} &      & \cmark &   \cmark     & \cmark &  & \cmark &  & –  & –   \\
		& \texttt{f3.ser} &      &  &        &  & \cmark &   \cmark & \cmark & –  & –   \\
		& \texttt{f4.ser} &      &  &     \cmark   &  & \cmark & \cmark & \cmark & –  & –   \\
		& \texttt{f5.ser} &      & \cmark &        & \cmark &  &       &  & –  & –   \\
		& \texttt{f6.ser} &      & \cmark &        & \cmark &  & \cmark &  & –  & –   \\
		& \texttt{f7.ser} &      & \cmark &        & \cmark &  &  \cmark &  & –  & –   \\
		& \texttt{f8.ser} &      & \cmark &        & \cmark &  &   \cmark &  & –  & –   \\
		& \texttt{f9.ser} &      & \cmark &        & \cmark &  &  \cmark &  & –  & –   \\
		\midrule
		
		\multirow{7}{=}{Networking \& system protocols}
		& \texttt{g1.ser} &      & \cmark & \cmark &  & \cmark & \cmark & \cmark & –  & –   \\
		& \texttt{g2.ser} &      & \cmark & \cmark &  & \cmark & \cmark & \cmark & –  & –   \\
		& \texttt{g3.ser} &      & \cmark & \cmark & \cmark & \cmark & \cmark & \cmark & –  & –   \\
		& \texttt{g4.ser} &      & \cmark & \cmark & \cmark & \cmark & \cmark & \cmark & –  & –   \\
		& \texttt{g5.ser} &      & \cmark & \cmark & \cmark & \cmark &   \cmark & \cmark & –  & –   \\
		& \texttt{g6.ser} &      & \cmark &        & \cmark & \cmark & \cmark &  & –  & –   \\
		& \texttt{g7.ser} &      & \cmark &        & \cmark & \cmark &       &  & –  & –   \\
		\bottomrule
	\end{tabular*}
	\caption{Overview of benchmarks with combined categories and updated serializability markings. All optimizations were used, as well as a timeout value of 500 seconds.}
\label{tab:benchmarks-all}
\end{table}



\begin{table}[H]
	\centering
	% Load the tabular from the external file:
	\begin{table}[H]
	\centering
	\small
	% increase horizontal padding between columns
	\setlength{\tabcolsep}{5pt}
	\renewcommand{\arraystretch}{0.9}
	\begin{tabular*}{\textwidth}{@{\extracolsep{\fill}}%
			p{1.5cm}   % Category
			p{1.0cm} % Benchmark
			c        % Serializable
			c c c c c c % Features
			r r       % Cert, Total
		}
		\toprule
		\multicolumn{2}{c}{\textbf{Benchmark}}
		& \textbf{Serializable}
		& \multicolumn{6}{c}{\textbf{Features}}
		& \multicolumn{2}{c}{\textbf{Runtime (ms)}} \\
		\cmidrule(lr){1-2} \cmidrule(lr){3-3} \cmidrule(lr){4-9} \cmidrule(lr){10-11}
		&
		&
		& If & While & \texttt{?} & Arith & Yield & Multi-req
		& Cert. & Total \\
		\midrule
		\multirow{7}{=}{Core expressions} & \texttt{a1.ser} & \greencmark &  & \cmark &  &  &       &   & 2 & 47 \\
		 & \texttt{a2.ser} & \xmark &  &        &  &  & \cmark &   & 280 & 296 \\
		 & \texttt{a3.ser} & \greencmark &  &        &  &  &       &   & 1 & 32 \\
		 & \texttt{a4.ser} & \greencmark &  &        &  &  & \cmark & \cmark & 637 & 1{,}071 \\
		 & \texttt{a5.ser} & \greencmark &  & \cmark &  &  & \cmark & \cmark & 3{,}234 & 13{,}624 \\
		 & \texttt{a6.ser} & \xmark &  &        &  &  & \cmark & \cmark & 757 & 775 \\
		 & \texttt{a7.ser} & \greencmark & \cmark & \cmark &  &  & \cmark &   & 4 & 33 \\
		\midrule
		\multirow{4}{=}{State machines} & \texttt{b1.json} & \greencmark & \cmark &        &  &  & \cmark & \cmark & 683 & 968 \\
		 & \texttt{b2.json} & \greencmark & \cmark &        &  &  & \cmark & \cmark & 2{,}063 & 7{,}802 \\
		 & \texttt{b3.json} & \greencmark & \cmark &        &  &  & \cmark & \cmark & 730 & 2{,}080 \\
		 & \texttt{b4.json} & \greencmark & \cmark &        &  &  & \cmark & \cmark & 660 & 1{,}909 \\
		\midrule
		\multirow{8}{=}{Mixed arithmetic} & \texttt{c1.ser} & \xmark &  & \cmark &  & \cmark & \cmark & \cmark & 356{,}195 & 356{,}299 \\
		 & \texttt{c2.ser} & \greencmark &  & \cmark &  & \cmark & \cmark & \cmark & 9{,}858 & 292{,}228 \\
		 & \texttt{c3.ser} & \greencmark &  & \cmark &  & \cmark & \cmark & \cmark & 1{,}886 & 2{,}397 \\
		 & \texttt{c4.ser} & \greencmark &  & \cmark &  & \cmark & \cmark & \cmark & 4{,}336 & 7{,}193 \\
		 & \texttt{c5.ser} & \xmark &  & \cmark &  & \cmark & \cmark & \cmark & 43{,}694 & 43{,}735 \\
		 & \texttt{c6.ser} & \xmark &  & \cmark &  & \cmark & \cmark & \cmark & 629 & 698 \\
		 & \texttt{c7.ser} & \xmark &  & \cmark &  & \cmark & \cmark & \cmark & 797 & 875 \\
		 & \texttt{c8.ser} & \greencmark &  & \cmark &  & \cmark & \cmark & \cmark & 4{,}357 & 8{,}931 \\
		\midrule
		\multirow{5}{=}{Circular increment} & \texttt{d1.ser} & \greencmark & \cmark & \cmark & \cmark &  & \cmark &   & 2{,}391 & 5{,}373 \\
		 & \texttt{d2.ser} & \xmark & \cmark &        & \cmark &  &   \cmark &   & 628 & 731 \\
		 & \texttt{d3.ser} & \greencmark & \cmark & \cmark & \cmark &  &  \cmark &   & 2{,}642 & 10{,}266 \\
		 & \texttt{d4.ser} & \greencmark & \cmark & \cmark & \cmark &  &     \cmark &   & 5{,}604 & 22{,}249 \\
		 & \texttt{d5.ser} & \xmark & \cmark &        &  &  & \cmark &   & 495 & 554 \\
		\midrule
		\multirow{7}{=}{Concurrency \& locking loops} & \texttt{e1.ser} & \greencmark &  & \cmark &  &  & \cmark &   & 351 & 502 \\
		 & \texttt{e2.ser} & \xmark & \cmark & \cmark &  & \cmark & \cmark & \cmark & \texttt{TIMEOUT} & \texttt{TIMEOUT} \\
		 & \texttt{e3.ser} & \xmark & \cmark & \cmark &  & \cmark &   \cmark & \cmark & 24{,}899 & 25{,}039 \\
		 & \texttt{e4.ser} & \xmark & \cmark & \cmark &  &  \cmark &   \cmark & \cmark & 273{,}062 & 273{,}351 \\
		 & \texttt{e5.ser} & \greencmark & \cmark & \cmark & \cmark &  & \cmark &   & 2 & 55 \\
		 & \texttt{e6.ser} & \greencmark & \cmark & \cmark & \cmark &  & \cmark &   & 10 & 114 \\
		 & \texttt{e7.ser} & \greencmark &  & \cmark &  &  &   \cmark &   & 299 & 444 \\
		\midrule
		\multirow{9}{=}{Non-deterministic \& randomness} & \texttt{f1.ser} & \greencmark & \cmark &    \cmark    & \cmark &  & \cmark &   & 388 & 494 \\
		 & \texttt{f2.ser} & \xmark & \cmark &   \cmark     & \cmark &  & \cmark &   & 612 & 676 \\
		 & \texttt{f3.ser} & \xmark &  &        &  & \cmark &   \cmark & \cmark & 653 & 716 \\
		 & \texttt{f4.ser} & \greencmark &  &     \cmark   &  & \cmark & \cmark & \cmark & 1{,}626 & 9{,}515 \\
		 & \texttt{f5.ser} & \greencmark & \cmark &        & \cmark &  &       &   & 7{,}401 & 11{,}301 \\
		 & \texttt{f6.ser} & \xmark & \cmark &        & \cmark &  & \cmark &   & 646 & 830 \\
		 & \texttt{f7.ser} & \xmark & \cmark &        & \cmark &  &  \cmark &   & 400 & 427 \\
		 & \texttt{f8.ser} & \xmark & \cmark &        & \cmark &  &   \cmark &   & 773 & 802 \\
		 & \texttt{f9.ser} & \greencmark & \cmark &        & \cmark &  &  \cmark &   & 10 & 94 \\
		\midrule
		\multirow{7}{=}{Networking \& system protocols} & \texttt{g1.ser} & \xmark & \cmark & \cmark &  & \cmark & \cmark & \cmark & 59{,}312 & 74{,}539 \\
		 & \texttt{g2.ser} & \greencmark & \cmark & \cmark &  & \cmark & \cmark & \cmark & \texttt{TIMEOUT} & \texttt{TIMEOUT} \\
		 & \texttt{g3.ser} & \xmark & \cmark & \cmark & \cmark & \cmark & \cmark & \cmark & 20{,}557 & 20{,}954 \\
		 & \texttt{g4.ser} & \xmark & \cmark & \cmark & \cmark & \cmark & \cmark & \cmark & 6{,}859 & 7{,}047 \\
		 & \texttt{g5.ser} & \greencmark & \cmark & \cmark & \cmark & \cmark &   \cmark & \cmark & 3{,}047 & 12{,}324 \\
		 & \texttt{g6.ser} & \xmark & \cmark &        & \cmark & \cmark & \cmark &   & 8{,}193 & 8{,}285 \\
		 & \texttt{g7.ser} & \greencmark & \cmark &        & \cmark & \cmark &       &   & 6{,}886 & 252{,}752 \\
		\midrule
\bottomrule
	\end{tabular*}
\end{table}

	\caption{Overview of benchmarks with combined categories and updated serializability markings. All optimizations were used, as well as a timeout value of 500 seconds.}
\label{tab:benchmarks-all-2}
\end{table}


\subsection{Optimization Analysis}


%\begin{figure}[htbp]
%	\centering
%	\includegraphics[width=0.4\textwidth]{plots/petri_size_reduction_plot.pdf}
%	\caption{Size reduction of Petri nets through optimization techniques. The plot shows the reduction in the number of places and transitions after applying our optimization passes. Averaged on all Petri Nets of 50 benchmarks (timeout 30 seconds).}
%	\label{fig:petri_size_reduction}
%\end{figure}


%\begin{figure}[htbp]
%	\centering
%	\includegraphics[width=0.4\textwidth]{plots/timeout_10000_cumulative_solved_linear.pdf}
%	\caption{Size reduction of Petri nets through optimization techniques. The plot shows the reduction in the number of places and transitions after applying our optimization passes. Averaged on all Petri Nets of 50 benchmarks (timeout 30 seconds).}
%	\label{fig:runtime}
%\end{figure}



\begin{center}
		\begin{minipage}[t]{0.48\textwidth}
		\centering
		\includegraphics[width=\linewidth]{plots/timeout_10000_cumulative_solved_log.pdf}
		\captionof{figure}{Cumulative number of solved instances over time with a 10\,000-second timeout (log scale).}
		\label{fig:timeout_cumulative_solved_log}
	\end{minipage}\hfill
	\begin{minipage}[t]{0.48\textwidth}
		\centering
		\includegraphics[width=\linewidth]{figures/petri_size_reduction_plot.pdf}
		\captionof{figure}{Reduction of Petri nets size (places and transitions) via bidirectional pruning 
%(timeout 150 seconds)
.}
		\label{fig:petri_size_reduction}
	\end{minipage}
\end{center}







%\begin{table}[H]
%	\centering
%	% Load the tabular from the external file:
%	\begin{table}[H]
	\centering
	\small
	% increase horizontal padding between columns
	\setlength{\tabcolsep}{5pt}
	\renewcommand{\arraystretch}{0.9}
	\begin{tabular*}{\textwidth}{@{\extracolsep{\fill}}%
			p{1.5cm}   % Category
			p{1.0cm} % Benchmark
			c        % Serializable
			c c c c c c % Features
			r r       % Cert, Total
		}
		\toprule
		\multicolumn{2}{c}{\textbf{Benchmark}}
		& \textbf{Serializable}
		& \multicolumn{6}{c}{\textbf{Features}}
		& \multicolumn{2}{c}{\textbf{Runtime (ms)}} \\
		\cmidrule(lr){1-2} \cmidrule(lr){3-3} \cmidrule(lr){4-9} \cmidrule(lr){10-11}
		&
		&
		& If & While & \texttt{?} & Arith & Yield & Multi-req
		& Cert. & Total \\
		\midrule
		\multirow{7}{=}{Core expressions} & \texttt{a1.ser} & \greencmark &  & \cmark &  &  &       &   & 2 & 47 \\
		 & \texttt{a2.ser} & \xmark &  &        &  &  & \cmark &   & 280 & 296 \\
		 & \texttt{a3.ser} & \greencmark &  &        &  &  &       &   & 1 & 32 \\
		 & \texttt{a4.ser} & \greencmark &  &        &  &  & \cmark & \cmark & 637 & 1{,}071 \\
		 & \texttt{a5.ser} & \greencmark &  & \cmark &  &  & \cmark & \cmark & 3{,}234 & 13{,}624 \\
		 & \texttt{a6.ser} & \xmark &  &        &  &  & \cmark & \cmark & 757 & 775 \\
		 & \texttt{a7.ser} & \greencmark & \cmark & \cmark &  &  & \cmark &   & 4 & 33 \\
		\midrule
		\multirow{4}{=}{State machines} & \texttt{b1.json} & \greencmark & \cmark &        &  &  & \cmark & \cmark & 683 & 968 \\
		 & \texttt{b2.json} & \greencmark & \cmark &        &  &  & \cmark & \cmark & 2{,}063 & 7{,}802 \\
		 & \texttt{b3.json} & \greencmark & \cmark &        &  &  & \cmark & \cmark & 730 & 2{,}080 \\
		 & \texttt{b4.json} & \greencmark & \cmark &        &  &  & \cmark & \cmark & 660 & 1{,}909 \\
		\midrule
		\multirow{8}{=}{Mixed arithmetic} & \texttt{c1.ser} & \xmark &  & \cmark &  & \cmark & \cmark & \cmark & 356{,}195 & 356{,}299 \\
		 & \texttt{c2.ser} & \greencmark &  & \cmark &  & \cmark & \cmark & \cmark & 9{,}858 & 292{,}228 \\
		 & \texttt{c3.ser} & \greencmark &  & \cmark &  & \cmark & \cmark & \cmark & 1{,}886 & 2{,}397 \\
		 & \texttt{c4.ser} & \greencmark &  & \cmark &  & \cmark & \cmark & \cmark & 4{,}336 & 7{,}193 \\
		 & \texttt{c5.ser} & \xmark &  & \cmark &  & \cmark & \cmark & \cmark & 43{,}694 & 43{,}735 \\
		 & \texttt{c6.ser} & \xmark &  & \cmark &  & \cmark & \cmark & \cmark & 629 & 698 \\
		 & \texttt{c7.ser} & \xmark &  & \cmark &  & \cmark & \cmark & \cmark & 797 & 875 \\
		 & \texttt{c8.ser} & \greencmark &  & \cmark &  & \cmark & \cmark & \cmark & 4{,}357 & 8{,}931 \\
		\midrule
		\multirow{5}{=}{Circular increment} & \texttt{d1.ser} & \greencmark & \cmark & \cmark & \cmark &  & \cmark &   & 2{,}391 & 5{,}373 \\
		 & \texttt{d2.ser} & \xmark & \cmark &        & \cmark &  &   \cmark &   & 628 & 731 \\
		 & \texttt{d3.ser} & \greencmark & \cmark & \cmark & \cmark &  &  \cmark &   & 2{,}642 & 10{,}266 \\
		 & \texttt{d4.ser} & \greencmark & \cmark & \cmark & \cmark &  &     \cmark &   & 5{,}604 & 22{,}249 \\
		 & \texttt{d5.ser} & \xmark & \cmark &        &  &  & \cmark &   & 495 & 554 \\
		\midrule
		\multirow{7}{=}{Concurrency \& locking loops} & \texttt{e1.ser} & \greencmark &  & \cmark &  &  & \cmark &   & 351 & 502 \\
		 & \texttt{e2.ser} & \xmark & \cmark & \cmark &  & \cmark & \cmark & \cmark & \texttt{TIMEOUT} & \texttt{TIMEOUT} \\
		 & \texttt{e3.ser} & \xmark & \cmark & \cmark &  & \cmark &   \cmark & \cmark & 24{,}899 & 25{,}039 \\
		 & \texttt{e4.ser} & \xmark & \cmark & \cmark &  &  \cmark &   \cmark & \cmark & 273{,}062 & 273{,}351 \\
		 & \texttt{e5.ser} & \greencmark & \cmark & \cmark & \cmark &  & \cmark &   & 2 & 55 \\
		 & \texttt{e6.ser} & \greencmark & \cmark & \cmark & \cmark &  & \cmark &   & 10 & 114 \\
		 & \texttt{e7.ser} & \greencmark &  & \cmark &  &  &   \cmark &   & 299 & 444 \\
		\midrule
		\multirow{9}{=}{Non-deterministic \& randomness} & \texttt{f1.ser} & \greencmark & \cmark &    \cmark    & \cmark &  & \cmark &   & 388 & 494 \\
		 & \texttt{f2.ser} & \xmark & \cmark &   \cmark     & \cmark &  & \cmark &   & 612 & 676 \\
		 & \texttt{f3.ser} & \xmark &  &        &  & \cmark &   \cmark & \cmark & 653 & 716 \\
		 & \texttt{f4.ser} & \greencmark &  &     \cmark   &  & \cmark & \cmark & \cmark & 1{,}626 & 9{,}515 \\
		 & \texttt{f5.ser} & \greencmark & \cmark &        & \cmark &  &       &   & 7{,}401 & 11{,}301 \\
		 & \texttt{f6.ser} & \xmark & \cmark &        & \cmark &  & \cmark &   & 646 & 830 \\
		 & \texttt{f7.ser} & \xmark & \cmark &        & \cmark &  &  \cmark &   & 400 & 427 \\
		 & \texttt{f8.ser} & \xmark & \cmark &        & \cmark &  &   \cmark &   & 773 & 802 \\
		 & \texttt{f9.ser} & \greencmark & \cmark &        & \cmark &  &  \cmark &   & 10 & 94 \\
		\midrule
		\multirow{7}{=}{Networking \& system protocols} & \texttt{g1.ser} & \xmark & \cmark & \cmark &  & \cmark & \cmark & \cmark & 59{,}312 & 74{,}539 \\
		 & \texttt{g2.ser} & \greencmark & \cmark & \cmark &  & \cmark & \cmark & \cmark & \texttt{TIMEOUT} & \texttt{TIMEOUT} \\
		 & \texttt{g3.ser} & \xmark & \cmark & \cmark & \cmark & \cmark & \cmark & \cmark & 20{,}557 & 20{,}954 \\
		 & \texttt{g4.ser} & \xmark & \cmark & \cmark & \cmark & \cmark & \cmark & \cmark & 6{,}859 & 7{,}047 \\
		 & \texttt{g5.ser} & \greencmark & \cmark & \cmark & \cmark & \cmark &   \cmark & \cmark & 3{,}047 & 12{,}324 \\
		 & \texttt{g6.ser} & \xmark & \cmark &        & \cmark & \cmark & \cmark &   & 8{,}193 & 8{,}285 \\
		 & \texttt{g7.ser} & \greencmark & \cmark &        & \cmark & \cmark &       &   & 6{,}886 & 252{,}752 \\
		\midrule
\bottomrule
	\end{tabular*}
\end{table}

%	\caption{Comparison of experiment runs with a 150-second timeout.}
%	\label{tab:semilinear-size-reduction}
%\end{table}


\newpage





%\section{Comprehensive Statistics}
%\begin{table}[H]
%	\centering
%	\input{tables/comprehensive\_stats.tex}
%	\caption{Comprehensive statistics}
%	\label{tab:comprehensive\_stats}
%\end{table}
%
%\section{Pruning Effectiveness}
%\begin{table}[H]
%	\centering
%	\begin{tabular}{llcccc}
\toprule
Example & Disj. & Places & Transitions & Iter. & Reduction \\
\midrule
\texttt{a2} & 0 & 8 → 8 & 7 → 7 & 1 & 0.0\% \\
\texttt{a6} & 0 & 16 → 11 & 18 → 11 & 2 & 38.9\% \\
\texttt{a6} & 1 & 16 → 11 & 18 → 11 & 2 & 38.9\% \\
\texttt{b1} & 0 & 12 → 9 & 10 → 7 & 2 & 30.0\% \\
\texttt{b1} & 1 & 12 → 9 & 10 → 7 & 2 & 30.0\% \\
\texttt{c6} & 0 & 23 → 15 & 32 → 14 & 2 & 56.2\% \\
\texttt{a4} & 0 & 17 → 9 & 14 → 8 & 2 & 42.9\% \\
\texttt{a4} & 1 & 17 → 9 & 14 → 8 & 2 & 42.9\% \\
\texttt{c7} & 0 & 36 → 18 & 57 → 19 & 2 & 66.7\% \\
\texttt{d2} & 0 & 18 → 18 & 25 → 25 & 1 & 0.0\% \\
\texttt{b4} & 0 & 17 → 17 & 21 → 18 & 2 & 14.3\% \\
\texttt{b4} & 1 & 17 → 17 & 21 → 18 & 2 & 14.3\% \\
\texttt{e1} & 0 & 8 → 8 & 7 → 6 & 2 & 14.3\% \\
\texttt{d5} & 0 & 11 → 11 & 10 → 10 & 1 & 0.0\% \\
\texttt{b3} & 0 & 17 → 17 & 18 → 18 & 1 & 0.0\% \\
\texttt{b3} & 1 & 17 → 17 & 18 → 18 & 1 & 0.0\% \\
\texttt{c3} & 0 & 14 → 8 & 14 → 6 & 2 & 57.1\% \\
\texttt{c3} & 1 & 14 → 8 & 14 → 6 & 2 & 57.1\% \\
\texttt{c3} & 2 & 14 → 8 & 14 → 6 & 2 & 57.1\% \\
\texttt{c3} & 3 & 14 → 8 & 14 → 6 & 2 & 57.1\% \\
\texttt{c3} & 4 & 14 → 8 & 14 → 6 & 2 & 57.1\% \\
\texttt{c3} & 5 & 14 → 8 & 14 → 6 & 2 & 57.1\% \\
\texttt{e7} & 0 & 8 → 8 & 7 → 6 & 2 & 14.3\% \\
\texttt{f1} & 0 & 10 → 1 & 15 → 0 & 3 & 100.0\% \\
\texttt{f2} & 0 & 14 → 14 & 25 → 23 & 2 & 8.0\% \\
\texttt{f3} & 0 & 24 → 17 & 26 → 16 & 2 & 38.5\% \\
\texttt{f6} & 0 & 11 → 9 & 13 → 11 & 2 & 15.4\% \\
\texttt{f7} & 0 & 8 → 8 & 9 → 9 & 1 & 0.0\% \\
\texttt{d1} & 0 & 14 → 14 & 19 → 13 & 2 & 31.6\% \\
\texttt{d1} & 1 & 14 → 11 & 19 → 11 & 2 & 42.1\% \\
\texttt{d1} & 2 & 14 → 8 & 19 → 6 & 2 & 68.4\% \\
\texttt{d1} & 3 & 14 → 11 & 19 → 8 & 2 & 57.9\% \\
\texttt{d1} & 4 & 14 → 11 & 19 → 8 & 2 & 57.9\% \\
\texttt{d1} & 5 & 14 → 14 & 19 → 13 & 2 & 31.6\% \\
\texttt{d1} & 6 & 14 → 11 & 19 → 11 & 2 & 42.1\% \\
\texttt{f8} & 0 & 9 → 1 & 11 → 0 & 3 & 100.0\% \\
\texttt{f8} & 1 & 9 → 9 & 11 → 11 & 1 & 0.0\% \\
\texttt{c4} & 0 & 19 → 13 & 20 → 10 & 2 & 50.0\% \\
\texttt{c4} & 1 & 19 → 13 & 20 → 10 & 2 & 50.0\% \\
\texttt{c4} & 2 & 19 → 13 & 20 → 10 & 2 & 50.0\% \\
\texttt{c4} & 3 & 19 → 13 & 20 → 10 & 2 & 50.0\% \\
\texttt{c4} & 4 & 19 → 1 & 20 → 0 & 3 & 100.0\% \\
\texttt{c4} & 5 & 19 → 13 & 20 → 10 & 2 & 50.0\% \\
\texttt{c4} & 6 & 19 → 13 & 20 → 10 & 2 & 50.0\% \\
\texttt{c4} & 7 & 19 → 13 & 20 → 10 & 2 & 50.0\% \\
\texttt{c4} & 8 & 19 → 13 & 20 → 10 & 2 & 50.0\% \\
\texttt{c4} & 9 & 19 → 13 & 20 → 10 & 2 & 50.0\% \\
\texttt{c4} & 10 & 19 → 13 & 20 → 10 & 2 & 50.0\% \\
\texttt{c4} & 11 & 19 → 13 & 20 → 10 & 2 & 50.0\% \\
\texttt{c4} & 12 & 19 → 13 & 20 → 10 & 2 & 50.0\% \\
\texttt{b2} & 0 & 18 → 16 & 24 → 19 & 2 & 20.8\% \\
\texttt{b2} & 1 & 18 → 17 & 24 → 20 & 2 & 16.7\% \\
\texttt{b2} & 2 & 18 → 17 & 24 → 20 & 2 & 16.7\% \\
\texttt{b2} & 3 & 18 → 18 & 24 → 21 & 2 & 12.5\% \\
\texttt{b2} & 4 & 18 → 18 & 24 → 21 & 2 & 12.5\% \\
\texttt{b2} & 5 & 18 → 17 & 24 → 20 & 2 & 16.7\% \\
\texttt{c8} & 0 & 21 → 13 & 22 → 10 & 2 & 54.5\% \\
\texttt{c8} & 1 & 21 → 13 & 22 → 10 & 2 & 54.5\% \\
\texttt{c8} & 2 & 21 → 13 & 22 → 10 & 2 & 54.5\% \\
\texttt{c8} & 3 & 21 → 13 & 22 → 10 & 2 & 54.5\% \\
\texttt{c8} & 4 & 21 → 1 & 22 → 0 & 3 & 100.0\% \\
\texttt{c8} & 5 & 21 → 13 & 22 → 10 & 2 & 54.5\% \\
\texttt{c8} & 6 & 21 → 13 & 22 → 10 & 2 & 54.5\% \\
\texttt{c8} & 7 & 21 → 13 & 22 → 10 & 2 & 54.5\% \\
\texttt{c8} & 8 & 21 → 13 & 22 → 10 & 2 & 54.5\% \\
\texttt{c8} & 9 & 21 → 13 & 22 → 10 & 2 & 54.5\% \\
\texttt{c8} & 10 & 21 → 13 & 22 → 10 & 2 & 54.5\% \\
\texttt{c8} & 11 & 21 → 13 & 22 → 10 & 2 & 54.5\% \\
\texttt{c8} & 12 & 21 → 13 & 22 → 10 & 2 & 54.5\% \\
\texttt{d3} & 0 & 18 → 15 & 25 → 11 & 2 & 56.0\% \\
\texttt{d3} & 1 & 18 → 9 & 25 → 7 & 2 & 72.0\% \\
\texttt{d3} & 2 & 18 → 12 & 25 → 9 & 2 & 64.0\% \\
\texttt{d3} & 3 & 18 → 12 & 25 → 9 & 2 & 64.0\% \\
\texttt{d3} & 4 & 18 → 18 & 25 → 17 & 2 & 32.0\% \\
\texttt{d3} & 5 & 18 → 15 & 25 → 15 & 2 & 40.0\% \\
\texttt{d3} & 6 & 18 → 12 & 25 → 13 & 2 & 48.0\% \\
\texttt{f4} & 0 & 24 → 5 & 26 → 4 & 3 & 84.6\% \\
\texttt{f4} & 1 & 24 → 5 & 26 → 4 & 3 & 84.6\% \\
\texttt{f4} & 2 & 24 → 5 & 26 → 4 & 2 & 84.6\% \\
\texttt{f4} & 3 & 24 → 5 & 26 → 4 & 2 & 84.6\% \\
\texttt{f4} & 4 & 24 → 5 & 26 → 4 & 2 & 84.6\% \\
\texttt{a5} & 0 & 30 → 12 & 39 → 14 & 2 & 64.1\% \\
\texttt{a5} & 1 & 30 → 12 & 39 → 14 & 2 & 64.1\% \\
\texttt{a5} & 2 & 30 → 12 & 39 → 14 & 2 & 64.1\% \\
\texttt{a5} & 3 & 30 → 12 & 39 → 14 & 2 & 64.1\% \\
\texttt{a5} & 4 & 30 → 12 & 39 → 14 & 2 & 64.1\% \\
\texttt{a5} & 5 & 30 → 12 & 39 → 14 & 2 & 64.1\% \\
\texttt{a5} & 6 & 30 → 4 & 39 → 3 & 3 & 92.3\% \\
\texttt{a5} & 7 & 30 → 4 & 39 → 3 & 3 & 92.3\% \\
\texttt{a5} & 8 & 30 → 4 & 39 → 3 & 3 & 92.3\% \\
\texttt{g4} & 0 & 47 → 30 & 94 → 49 & 2 & 47.9\% \\
\texttt{f5} & 0 & 10 → 7 & 10 → 5 & 3 & 50.0\% \\
\texttt{f5} & 1 & 10 → 10 & 10 → 10 & 1 & 0.0\% \\
\texttt{f5} & 2 & 10 → 7 & 10 → 5 & 3 & 50.0\% \\
\texttt{f5} & 3 & 10 → 10 & 10 → 10 & 1 & 0.0\% \\
\texttt{f5} & 4 & 10 → 10 & 10 → 10 & 1 & 0.0\% \\
\texttt{f5} & 5 & 10 → 10 & 10 → 10 & 1 & 0.0\% \\
\texttt{f5} & 6 & 10 → 5 & 10 → 3 & 3 & 70.0\% \\
\texttt{f5} & 7 & 10 → 7 & 10 → 5 & 3 & 50.0\% \\
\texttt{f5} & 8 & 10 → 7 & 10 → 5 & 3 & 50.0\% \\
\texttt{f5} & 9 & 10 → 8 & 10 → 7 & 2 & 30.0\% \\
\texttt{f5} & 10 & 10 → 10 & 10 → 10 & 1 & 0.0\% \\
\texttt{f5} & 11 & 10 → 10 & 10 → 10 & 1 & 0.0\% \\
\texttt{f5} & 12 & 10 → 10 & 10 → 10 & 1 & 0.0\% \\
\texttt{f5} & 13 & 10 → 8 & 10 → 7 & 2 & 30.0\% \\
\texttt{f5} & 14 & 10 → 10 & 10 → 10 & 1 & 0.0\% \\
\texttt{f5} & 15 & 10 → 10 & 10 → 10 & 1 & 0.0\% \\
\texttt{f5} & 16 & 10 → 10 & 10 → 10 & 1 & 0.0\% \\
\texttt{f5} & 17 & 10 → 8 & 10 → 7 & 2 & 30.0\% \\
\texttt{f5} & 18 & 10 → 8 & 10 → 7 & 2 & 30.0\% \\
\texttt{f5} & 19 & 10 → 10 & 10 → 10 & 1 & 0.0\% \\
\texttt{f5} & 20 & 10 → 10 & 10 → 10 & 1 & 0.0\% \\
\texttt{g5} & 0 & 26 → 16 & 34 → 17 & 2 & 50.0\% \\
\texttt{g5} & 1 & 26 → 16 & 34 → 17 & 2 & 50.0\% \\
\texttt{g5} & 2 & 26 → 1 & 34 → 0 & 3 & 100.0\% \\
\texttt{g5} & 3 & 26 → 1 & 34 → 0 & 3 & 100.0\% \\
\texttt{g5} & 4 & 26 → 1 & 34 → 0 & 3 & 100.0\% \\
\texttt{g5} & 5 & 26 → 16 & 34 → 17 & 2 & 50.0\% \\
\texttt{g6} & 0 & 22 → 1 & 24 → 0 & 3 & 100.0\% \\
\texttt{g6} & 1 & 22 → 1 & 24 → 0 & 3 & 100.0\% \\
\texttt{g6} & 2 & 22 → 1 & 24 → 0 & 3 & 100.0\% \\
\texttt{g6} & 3 & 22 → 15 & 24 → 15 & 3 & 37.5\% \\
\texttt{g6} & 4 & 22 → 10 & 24 → 10 & 3 & 58.3\% \\
\texttt{g6} & 5 & 22 → 10 & 24 → 10 & 3 & 58.3\% \\
\texttt{g6} & 6 & 22 → 15 & 24 → 15 & 3 & 37.5\% \\
\texttt{g6} & 7 & 22 → 15 & 24 → 15 & 3 & 37.5\% \\
\texttt{g6} & 8 & 22 → 17 & 24 → 18 & 3 & 25.0\% \\
\texttt{g6} & 9 & 22 → 1 & 24 → 0 & 3 & 100.0\% \\
\texttt{g6} & 10 & 22 → 17 & 24 → 18 & 3 & 25.0\% \\
\texttt{g6} & 11 & 22 → 17 & 24 → 18 & 3 & 25.0\% \\
\texttt{g6} & 12 & 22 → 15 & 24 → 16 & 3 & 33.3\% \\
\texttt{g6} & 13 & 22 → 19 & 24 → 20 & 3 & 16.7\% \\
\texttt{g6} & 14 & 22 → 17 & 24 → 17 & 3 & 29.2\% \\
\texttt{g6} & 15 & 22 → 5 & 24 → 3 & 3 & 87.5\% \\
\texttt{g6} & 16 & 22 → 5 & 24 → 3 & 3 & 87.5\% \\
\texttt{g6} & 17 & 22 → 5 & 24 → 3 & 3 & 87.5\% \\
\texttt{g6} & 18 & 22 → 19 & 24 → 20 & 3 & 16.7\% \\
\texttt{g6} & 19 & 22 → 19 & 24 → 20 & 3 & 16.7\% \\
\texttt{g6} & 20 & 22 → 10 & 24 → 7 & 3 & 70.8\% \\
\texttt{g6} & 21 & 22 → 10 & 24 → 7 & 3 & 70.8\% \\
\texttt{g6} & 22 & 22 → 22 & 24 → 24 & 1 & 0.0\% \\
\texttt{g6} & 23 & 22 → 20 & 24 → 22 & 2 & 8.3\% \\
\texttt{d4} & 0 & 18 → 18 & 25 → 17 & 2 & 32.0\% \\
\texttt{d4} & 1 & 18 → 12 & 25 → 13 & 2 & 48.0\% \\
\texttt{d4} & 2 & 18 → 15 & 25 → 15 & 2 & 40.0\% \\
\texttt{d4} & 3 & 18 → 15 & 25 → 15 & 2 & 40.0\% \\
\texttt{d4} & 4 & 18 → 12 & 25 → 9 & 2 & 64.0\% \\
\texttt{d4} & 5 & 18 → 9 & 25 → 7 & 2 & 72.0\% \\
\texttt{d4} & 6 & 18 → 15 & 25 → 11 & 2 & 56.0\% \\
\texttt{d4} & 7 & 18 → 15 & 25 → 11 & 2 & 56.0\% \\
\texttt{d4} & 8 & 18 → 12 & 25 → 9 & 2 & 64.0\% \\
\texttt{d4} & 9 & 18 → 18 & 25 → 17 & 2 & 32.0\% \\
\texttt{d4} & 10 & 18 → 15 & 25 → 15 & 2 & 40.0\% \\
\texttt{d4} & 11 & 18 → 12 & 25 → 13 & 2 & 48.0\% \\
\texttt{d4} & 12 & 18 → 18 & 25 → 17 & 2 & 32.0\% \\
\texttt{d4} & 13 & 18 → 15 & 25 → 15 & 2 & 40.0\% \\
\texttt{d4} & 14 & 18 → 12 & 25 → 13 & 2 & 48.0\% \\
\texttt{g3} & 0 & 101 → 39 & 159 → 55 & 2 & 65.4\% \\
\texttt{g3} & 1 & 101 → 39 & 159 → 55 & 2 & 65.4\% \\
\texttt{c5} & 0 & 29 → 17 & 46 → 22 & 2 & 52.2\% \\
\texttt{e3} & 0 & 65 → 25 & 99 → 35 & 2 & 64.6\% \\
\texttt{e3} & 1 & 65 → 25 & 99 → 35 & 2 & 64.6\% \\
\texttt{e3} & 2 & 65 → 25 & 99 → 35 & 2 & 64.6\% \\
\texttt{g1} & 0 & 86 → 56 & 285 → 103 & 2 & 63.9\% \\
\texttt{c1} & 0 & 40 → 24 & 78 → 38 & 2 & 51.3\% \\
\texttt{e2} & 0 & 30 → 16 & 40 → 15 & 2 & 62.5\% \\
\texttt{e4} & 0 & 77 → 29 & 123 → 43 & 2 & 65.0\% \\
\texttt{e4} & 1 & 77 → 29 & 123 → 43 & 2 & 65.0\% \\
\texttt{e4} & 2 & 77 → 29 & 123 → 43 & 2 & 65.0\% \\
\texttt{g2} & 0 & 69 → 37 & 135 → 48 & 2 & 64.4\% \\
\texttt{g2} & 1 & 69 → 37 & 135 → 48 & 2 & 64.4\% \\
\texttt{g7} & 0 & 18 → 1 & 15 → 0 & 3 & 100.0\% \\
\texttt{g7} & 1 & 18 → 1 & 15 → 0 & 3 & 100.0\% \\
\texttt{g7} & 2 & 18 → 10 & 15 → 7 & 3 & 53.3\% \\
\texttt{g7} & 3 & 18 → 5 & 15 → 3 & 3 & 80.0\% \\
\texttt{g7} & 4 & 18 → 10 & 15 → 7 & 3 & 53.3\% \\
\texttt{g7} & 5 & 18 → 12 & 15 → 9 & 3 & 40.0\% \\
\texttt{g7} & 6 & 18 → 1 & 15 → 0 & 3 & 100.0\% \\
\texttt{g7} & 7 & 18 → 12 & 15 → 9 & 3 & 40.0\% \\
\texttt{g7} & 8 & 18 → 12 & 15 → 9 & 3 & 40.0\% \\
\texttt{g7} & 9 & 18 → 15 & 15 → 12 & 3 & 20.0\% \\
\texttt{g7} & 10 & 18 → 13 & 15 → 9 & 3 & 40.0\% \\
\texttt{g7} & 11 & 18 → 5 & 15 → 3 & 3 & 80.0\% \\
\texttt{g7} & 12 & 18 → 5 & 15 → 3 & 3 & 80.0\% \\
\texttt{g7} & 13 & 18 → 15 & 15 → 12 & 3 & 20.0\% \\
\texttt{g7} & 14 & 18 → 15 & 15 → 12 & 3 & 20.0\% \\
\texttt{g7} & 15 & 18 → 10 & 15 → 7 & 3 & 53.3\% \\
\texttt{g7} & 16 & 18 → 18 & 15 → 15 & 1 & 0.0\% \\
\texttt{g7} & 17 & 18 → 12 & 15 → 9 & 3 & 40.0\% \\
\texttt{g7} & 18 & 18 → 18 & 15 → 15 & 1 & 0.0\% \\
\texttt{g7} & 19 & 18 → 18 & 15 → 15 & 1 & 0.0\% \\
\texttt{g7} & 20 & 18 → 16 & 15 → 12 & 2 & 20.0\% \\
\texttt{g7} & 21 & 18 → 10 & 15 → 7 & 3 & 53.3\% \\
\texttt{c2} & 0 & 24 → 18 & 26 → 14 & 2 & 46.2\% \\
\texttt{c2} & 1 & 24 → 18 & 26 → 14 & 2 & 46.2\% \\
\texttt{c2} & 2 & 24 → 18 & 26 → 14 & 2 & 46.2\% \\
\texttt{c2} & 3 & 24 → 18 & 26 → 14 & 2 & 46.2\% \\
\texttt{c2} & 4 & 24 → 18 & 26 → 14 & 2 & 46.2\% \\
\texttt{c2} & 5 & 24 → 1 & 26 → 0 & 3 & 100.0\% \\
\texttt{c2} & 6 & 24 → 1 & 26 → 0 & 3 & 100.0\% \\
\texttt{c2} & 7 & 24 → 8 & 26 → 6 & 3 & 76.9\% \\
\texttt{c2} & 8 & 24 → 18 & 26 → 14 & 2 & 46.2\% \\
\texttt{c2} & 9 & 24 → 18 & 26 → 14 & 2 & 46.2\% \\
\texttt{c2} & 10 & 24 → 18 & 26 → 14 & 2 & 46.2\% \\
\texttt{c2} & 11 & 24 → 5 & 26 → 3 & 3 & 88.5\% \\
\texttt{c2} & 12 & 24 → 8 & 26 → 6 & 3 & 76.9\% \\
\texttt{c2} & 13 & 24 → 18 & 26 → 14 & 2 & 46.2\% \\
\texttt{c2} & 14 & 24 → 18 & 26 → 14 & 2 & 46.2\% \\
\texttt{c2} & 15 & 24 → 18 & 26 → 14 & 2 & 46.2\% \\
\texttt{c2} & 16 & 24 → 18 & 26 → 14 & 2 & 46.2\% \\
\texttt{c2} & 17 & 24 → 18 & 26 → 14 & 2 & 46.2\% \\
\texttt{c2} & 18 & 24 → 18 & 26 → 14 & 2 & 46.2\% \\
\texttt{c2} & 19 & 24 → 18 & 26 → 14 & 2 & 46.2\% \\
\texttt{c2} & 20 & 24 → 18 & 26 → 14 & 2 & 46.2\% \\
\texttt{c2} & 21 & 24 → 18 & 26 → 14 & 2 & 46.2\% \\
\texttt{c2} & 22 & 24 → 18 & 26 → 14 & 2 & 46.2\% \\
\texttt{c2} & 23 & 24 → 18 & 26 → 14 & 2 & 46.2\% \\
\texttt{c2} & 24 & 24 → 18 & 26 → 14 & 2 & 46.2\% \\
\bottomrule
\end{tabular}

%	\caption{Petri net pruning effectiveness}
%	\label{tab:pruning\_effectiveness}
%\end{table}
%
%\section{Timing Comparison}
%\begin{table}[H]
%	\centering
%	\begin{tabular}{lccccc}
\toprule
Example & Unopt. & Opt. & Speedup & Create & Check \\n\midrule
\texttt{complex\_while\_with\_yields} & 57.0s & 30.6s & 1.9× & 30.5s & 0ms \\n\texttt{data\_flow} & 2.1s & 1.3s & 1.6× & 795ms & 394ms \\n\texttt{ex} & 986ms & 554ms & 1.8× & 329ms & 176ms \\n\texttt{flag\_non\_ser} & 45.0s & 750ms & 60.0× & 729ms & 0ms \\n\texttt{flag\_non\_ser\_turned\_ser} & 126ms & 96ms & 1.3× & 12ms & 20ms \\n\texttt{fred2\_arith} & 56.7s & 31.1s & 1.8× & 30.9s & 0ms \\n\texttt{fred\_arith\_simplified\_until\_1} & 5.0s & 3.5s & 1.5× & 2.8s & 431ms \\n\texttt{fred\_arith\_simplified\_until\_2} & 44.9s & 9.2s & 4.9× & 5.6s & 2.6s \\n\texttt{fred\_arith\_tricky} & 45.1s & 30.6s & 1.5× & 30.6s & 0ms \\n\texttt{fred\_arith\_tricky2} & 1.6s & 1.0s & 1.6× & 935ms & 0ms \\n\texttt{fred\_arith\_tricky3} & 2.5s & 1.3s & 1.9× & 1.2s & 0ms \\n\texttt{funny} & 1.1s & 653ms & 1.6× & 594ms & 0ms \\n\texttt{if\_while\_with\_req} & 93ms & 80ms & 1.2× & 6ms & 15ms \\n\texttt{incrdecr} & 43.8s & 11.1s & 3.9× & 6.2s & 3.7s \\n\texttt{login\_flow} & 7.6s & 2.8s & 2.7× & 1.0s & 1.5s \\n\texttt{modulo} & 19.6s & 14.5s & 1.4× & 9.7s & 3.5s \\n\texttt{modulo\_nonser} & 4.5s & 1.0s & 4.5× & 735ms & 0ms \\n\texttt{multiple\_requests\_updated} & 2.2s & 1.6s & 1.4× & 956ms & 511ms \\n\texttt{nested\_while} & 62ms & 57ms & 1.1× & 1ms & 2ms \\n\texttt{nondet} & 56.8s & 566ms & 100.4× & 403ms & 61ms \\n\texttt{nondet2} & 1.7s & 1.4s & 1.2× & 1.3s & 0ms \\n\texttt{nondet\_impl} & 1.4s & 745ms & 1.8× & 671ms & 0ms \\n\texttt{nondet\_impl2} & 55.8s & 11.7s & 4.8× & 2.3s & 8.6s \\n\texttt{self\_loop} & 72ms & 56ms & 1.3× & 2ms & 3ms \\n\texttt{self\_loop2} & 170ms & 141ms & 1.2× & 15ms & 33ms \\n\texttt{simple\_nonser} & 1.4s & 1.1s & 1.3× & 1.1s & 0ms \\n\texttt{simple\_nonser2} & 734ms & 586ms & 1.3× & 550ms & 0ms \\n\texttt{simple\_nonser2\_minus\_yields\_is\_ser} & 79ms & 66ms & 1.2× & 1ms & 2ms \\n\texttt{simple\_nonser2\_turned\_ser\_with\_locks} & 883ms & 621ms & 1.4× & 348ms & 192ms \\n\texttt{simple\_nonser3} & 916ms & 623ms & 1.5× & 560ms & 0ms \\n\texttt{snapshot\_isolation\_network\_monitoring} & 48.0s & 10.6s & 4.5× & 10.3s & 0ms \\n\texttt{snapshot\_isolation\_network\_monitoring\_without\_yields} & 16.5s & 16.0s & 1.0× & 2.8s & 12.0s \\n\texttt{state\_machine} & 6.6s & 2.9s & 2.3× & 1.0s & 1.7s \\n\texttt{stateful\_firewall} & 14.0s & 12.9s & 1.1× & 12.8s & 0ms \\n\texttt{stop} & 31.6s & 31.0s & 1.0× & 30.9s & 0ms \\n\texttt{stop2} & 8.0s & 6.6s & 1.2× & 3.3s & 2.6s \\n\texttt{stop3} & 1.6s & 1.0s & 1.6× & 920ms & 0ms \\n\texttt{stop3a} & 31.6s & 31.3s & 1.0× & 31.2s & 0ms \\n\texttt{stop4} & 11.9s & 13.1s & 0.9× & 3.5s & 8.4s \\n\texttt{stop4a} & 15.7s & 25.6s & 0.6× & 7.6s & 15.2s \\n\texttt{tricky2} & 51.6s & 20.4s & 2.5× & 20.2s & 0ms \\n\texttt{tricky3} & 51.7s & 33.3s & 1.6× & 32.9s & 0ms \\n\texttt{tricky3\_ser} & 52.4s & 37.4s & 1.4× & 36.4s & 0ms \\n\bottomrule
\end{tabular}

%	\caption{Comparison of optimized vs.\ unoptimized run times}
%	\label{tab:timing\_comparison}
%\end{table}
%
%\section{Optimization Breakdown}
%\begin{table}[H]
%	\centering
%	\begin{tabular}{lcccccc}
\toprule
Example & No Opt & B & R & G & S & All Opt \\
\midrule
\texttt{data_flow} & 4.1s & 7.3s & -- & -- & 9.3s & 6.6s \\
\texttt{ex} & 2.4s & 3.3s & 4.2s & 5.2s & 3.5s & 2.7s \\
\texttt{flag_non_ser} & 7.1s & 3.4s & 8.7s & 8.8s & 8.2s & 3.1s \\
\texttt{flag_non_ser_turned_ser} & 1.9s & 2.8s & 2.9s & 3.8s & 2.6s & 2.5s \\
\texttt{fred_arith_tricky} & 8.4s & -- & -- & -- & -- & 8.1s \\
\texttt{fred_arith_tricky2} & 3.7s & 6.1s & 6.6s & 7.7s & 7.2s & 4.8s \\
\texttt{fred_arith_tricky3} & 5.2s & -- & -- & -- & -- & 9.5s \\
\texttt{funny} & 2.7s & 3.1s & 2.6s & 4.1s & 2.7s & 2.7s \\
\texttt{if_while_with_req} & 2.6s & 2.2s & 2.0s & 2.9s & 2.1s & 1.7s \\
\texttt{nested_while} & 2.0s & 2.4s & 2.5s & 2.6s & 2.4s & 1.8s \\
\texttt{nondet} & 5.1s & 5.4s & 9.6s & -- & 9.1s & 3.7s \\
\texttt{nondet2} & 4.7s & 7.6s & 8.0s & 8.3s & 7.2s & 6.8s \\
\texttt{nondet_impl} & 2.9s & 4.6s & 4.8s & 5.5s & 5.1s & 3.0s \\
\texttt{self_loop} & 2.0s & 2.4s & 2.4s & 3.2s & 3.1s & 2.3s \\
\texttt{self_loop2} & 3.5s & 6.6s & 6.9s & 6.4s & 6.4s & 5.7s \\
\texttt{simple_nonser} & 3.0s & 4.2s & 4.7s & 4.0s & 4.1s & 4.1s \\
\texttt{simple_nonser2} & 2.9s & 3.7s & 4.0s & 3.4s & 3.6s & 3.9s \\
\texttt{simple_nonser2_minus_yields_is_ser} & 2.4s & 3.1s & 3.6s & 2.7s & 2.4s & 3.5s \\
\texttt{simple_nonser2_turned_ser_with_locks} & 2.9s & 4.5s & 4.5s & 4.1s & 3.7s & 4.2s \\
\texttt{simple_nonser3} & 2.6s & 3.8s & 4.3s & 3.6s & 3.1s & 3.7s \\
\texttt{stop} & 8.8s & -- & -- & -- & -- & 9.5s \\
\texttt{stop3} & 5.4s & 7.2s & 8.2s & 7.4s & 7.2s & 5.4s \\
\bottomrule
\end{tabular}

% B = Bidirectional Pruning, R = Remove Redundant, G = Generate Less, S = Smart Kleene Order

%	\caption{Impact of individual optimizations}
%	\label{tab:optimization\_breakdown}
%\end{table}
%
%\section{Summary Statistics}
%\begin{table}[H]
%	\centering
%	\begin{itemize}
\item Total examples analyzed: 48
\item Serializable: 24 (50.0\%)
\item Not serializable: 15 (31.2\%)
\item Timeouts: 9 (18.8\%)
\item Average pruning effectiveness: 185.0\%
\item Average analysis time: 11.3s
\end{itemize}

%	\caption{Overall summary statistics}
%	\label{tab:summary\_stats}
%\end{table}