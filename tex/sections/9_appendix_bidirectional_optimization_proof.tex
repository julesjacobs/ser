%% Appendix
%\appendix

\section{Proof: Bidirectional Optimization Correctness}
\label{appendix:BidirectionalProof}

%\subsection{Preliminaries}
%
%
%\begin{definition}[Petri Net]
%	A \emph{Petri net} is a tuple
%	\[
%	N = (P,\,T,\,\Pre,\,\Post)
%	\]
%	where
%	\begin{itemize}
%		\item $P$ is a finite set of \emph{places},
%		\item $T$ is a finite set of \emph{transitions},
%		\item $\Pre: P\times T \to \mathbb{N}$ is the \emph{pre-incidence} function,
%		\item $\Post: P\times T \to \mathbb{N}$ is the \emph{post-incidence} function.
%	\end{itemize}
%\end{definition}
%
%\begin{definition}[Marking]
%	A \emph{marking} is a function $M: P \to \mathbb{N}$. We write $M(p)$
%	for the number of tokens in place $p$.  The initial marking is
%	denoted $M_0$.  A transition $t\in T$ is \emph{enabled} at marking
%	$M$ if $\forall p\in P:\,M(p)\ge\Pre(p,t)$.  Firing $t$ yields the
%	new marking
%	\[
%	M' = M - \Pre(\cdot,t) + \Post(\cdot,t),
%	\]
%	written $M \xrightarrow{t} M'$.
%\end{definition}
%
%\begin{definition}[Firing Sequence]
%	A sequence $\sigma = t_1 t_2 \cdots t_k \in T^*$ is \emph{fireable}
%	from $M_0$ if there exist markings $M_1,\dots,M_k$ such that
%	$M_0\xrightarrow{t_1}M_1\cdots\xrightarrow{t_k}M_k$.  We write
%	$M_0 \xrightarrow{\sigma} M_k$.
%\end{definition}
%
%\begin{definition}[Semilinear Target Set]
%	A \emph{semilinear} set $S\subseteq \mathbb{N}^P$ is a finite union of
%	linear sets.  We assume $S$ is given by a finite description of its
%	linear components.  We view $S$ as the \emph{target} set of markings
%	we wish to reach.
%\end{definition}

\subsection{The Bidirectional Pruning Algorithm}

Let $N=(P,T,\Pre,\Post)$, initial marking $M_0$, and target set
$S\subseteq\mathbb{N}^P$ be fixed.

\begin{definition}[Forward Over-Approximation]
	Define the operator $\mathcal{F}:\mathcal{P}(P\cup T)\to\mathcal{P}(P\cup T)$ by
	\[
	X \mapsto X
	~\cup~
	\{\,t\in T \mid \forall p\in P:\; \Pre(p,t)>0 \implies p\in X\}
	~\cup~
	\{\,p\in P \mid \exists t\in X\cap T,\ \Post(p,t)>0\}.
	\]
	Starting from $X_0 = \{\,p\mid M_0(p)>0\}$, iterate
	$X_{i+1} = \mathcal{F}(X_i)$ until a least fixed-point
	$X^*=\bigcup_i X_i$ is reached.  Call $X^*_P = X^*\cap P$ the set of
	\emph{forward-reachable} places.
\end{definition}

\begin{definition}[Backward Over-Approximation]
	Let
	\[
	Y_0 = \{\,p\in P \mid \exists M\in S:\;M(p)\neq0\}
	\]
	be the places unconstrained to zero by the target.  Define
	$\mathcal{B}:\mathcal{P}(P\cup T)\to\mathcal{P}(P\cup T)$ by
	\[
	Y \mapsto Y
	~\cup~
	\{\,t\in T \mid \forall p\in P:\; \Post(p,t)>0 \implies p\in Y\}
	~\cup~
	\{\,p\in P \mid \exists t\in Y\cap T,\ \Pre(p,t)>0\}.
	\]
	Iterate $Y_{i+1} = \mathcal{B}(Y_i)$ until a least fixed-point
	$Y^*=\bigcup_i Y_i$ is reached.  Call $Y^*_P = Y^*\cap P$ the set of
	\emph{backward-relevant} places.
\end{definition}

\begin{definition}[Pruned Net]
	The \emph{pruned} subnet is
	\[
	N' = \bigl(P',\,T',\,\Pre|_{P'\times T'},\,\Post|_{P'\times T'}\bigr)
	\]
	where
	\[
	P' = X^*_P \;\cap\; Y^*_P,
	\quad
	T' = \{\,t\in T \mid
	\forall p:\;\Pre(p,t)>0\implies p\in P',\;
	\forall p:\;\Post(p,t)>0\implies p\in P'
	\}.
	\]
\end{definition}

\subsection{Invariant and Correctness}

Intuitively, $P'$ contains exactly those places that
\emph{may} occur in some firing sequence from $M_0$ to a marking in $S$.

\begin{definition}[Witnessable Place]
	A place $p\in P$ is \emph{witnessable} if there exists a firing
	sequence $\sigma\in T^*$ and markings $M$ and $M'$ such that
	\[
	M_0 \xrightarrow{\sigma_1} M
	\quad\text{and}\quad
	M \xrightarrow{\sigma_2} M'
	\quad\text{with}\quad
	M(p)>0
	\quad\text{and}\quad
	M'\in S.
	\]
	In other words, $p$ can carry a token in some execution from $M_0$ into the target set $S$.
\end{definition}

\begin{theorem}[Pruning Invariant]
	\label{thm:invariant}
	If a place $p$ is witnessable, then $p\in P'$.  Equivalently, the
	pruned net $N'$ \emph{over-approximates} the set of witnessable places.
\end{theorem}

\begin{proof}
	We split the argument into two parts.
	
	\paragraph{(1) Forward-reachability.}
	Suppose $p$ is witnessable.  Then there is a prefix
	$\sigma_1\in T^*$ such that $M_0\xrightarrow{\sigma_1}M$ and
	$M(p)>0$.  By standard Petri-net monotonicity, every place that
	receives a token in the course of $\sigma_1$ must appear in the
	forward fixed-point $X^*_P$.  Hence $p\in X^*_P$.
	
	\paragraph{(2) Backward-relevance.}
	Again, since $p$ is witnessable, there is a suffix
	$\sigma_2\in T^*$ from $M$ to $M'\in S$ with $M(p)>0$.  Working
	backward from $S$, every place that can contribute to satisfying
	semilinear constraints appears in the backward fixed-point $Y^*_P$.
	Thus $p\in Y^*_P$.
	
	\paragraph{Conclusion.}
	Combining (1) and (2) yields $p\in X^*_P\cap Y^*_P = P'$, as desired.
\end{proof}

\subsection{Termination and Complexity}

\begin{lemma}
	Each iteration of $\mathcal{F}$ and $\mathcal{B}$ strictly increases
	the set of included elements (unless already at the fixed point), and
	the total number of elements is finite.  Hence both reach their
	fixed-points in at most $|P|+|T|$ iterations each.
\end{lemma}

\begin{proof}
	Immediate from monotonicity and finiteness.
\end{proof}

\noindent
Therefore the bidirectional pruning converges in polynomial time, and preserves exactly the
places and transitions that \emph{may} appear in some execution from
$M_0$ into $S$.


%\newpage