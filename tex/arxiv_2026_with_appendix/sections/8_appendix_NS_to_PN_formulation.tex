\clearpage

\section{Translating Network Systems to Petri Nets}
\label{appendix:NS-to-PN-formulation}


We denote with \(\mathbf0\) a zero vector of dimension \(|P|\), and with \(\mathbf1_{p}\) a \(|P|\)-sized indicator vector that has 0 in every coordinate except the one corresponding to place \(p \in P\), which has 1. 
% 
The flow functions $\mathsf{pre},\mathsf{post}:T\to\{0,1\}^{|P|}$ assign to each transition $t$ a binary vector over $P$ whose $1$-entries mark the places from which tokens are consumed (for $\mathsf{pre}(t)$) and to which tokens are produced (for $\mathsf{post}(t)$) when $t$ fires.
A transition $t$ is enabled at $M$ iff $\mathsf{pre}(t)\le M$ (component-wise); firing yields
\(
M\xrightarrow{t}M' \quad\text{where}\quad M' = M-\mathsf{pre}(t)+\mathsf{post}(t)
\).	

\medskip
\textit{Construction.}
We generate the Petri net:
\[
N_{\mathrm{int}}(\mathcal S)
= (P,\,T,\,\mathsf{pre},\,\mathsf{post},\,M_0),
\]
where
\[
P
=
P_G \;\cup\; P_{REQ,L} \;\cup\; P_{REQ,RESP}
\]

for 
\[
\begin{aligned}
	P_G
	&= \{\,p_g \mid g\in G\},\quad
	P_{REQ,L}
	= \bigl\{\,p_{({\color{ForestGreen}\blacklozenge_{\mathit{req}}},\ell)}
	\mid {\color{ForestGreen}\blacklozenge_{\mathit{req}}}\in\mathit{REQ},\,\ell\in  L\bigr\},\\[1ex]
	P_{REQ,RESP}
	&= \bigl\{\,p_{({\color{ForestGreen}\blacklozenge_{\mathit{req}}}/{\color{red}\blacklozenge_{\mathit{resp}}})}
	\mid {\color{ForestGreen}\blacklozenge_{\mathit{req}}}\in\mathit{REQ},\,
	{\color{red}\blacklozenge_{\mathit{resp}}}\in\mathit{RESP}\bigr\}.
\end{aligned}
\]


%	\[
%	\begin{aligned}
	%		P_G &= \{\,p_g \mid g\in G\},\;\,
	%		P_{REQ,L} = \{\,p_{{\color{ForestGreen}\blacklozenge_{\mathit{req}}}/\ell}
	%		\mid {\color{ForestGreen}\blacklozenge_{\mathit{req}}}\in\mathit{REQ},\,\ell\in L\},\\
	%		P_{REQ,RESP} &= \{\,p_{{\color{ForestGreen}\blacklozenge_{\mathit{req}}}/{\color{red}\blacklozenge_{\mathit{resp}}}}
	%		\mid {\color{ForestGreen}\blacklozenge_{\mathit{req}}}\in\mathit{REQ},\,
	%		{\color{red}\blacklozenge_{\mathit{resp}}}\in\mathit{RESP}\}.
	%	\end{aligned}
%	\]

%	\[
%	P_G 
%	= \{\,p_g \mid g\in G\}
%	\quad 
%%	P_L 
%%	= \{\,p_\ell \mid \ell\in L\}
%%	\quad
%	P_{REQ,L} =
%	\{\,p_{{\color{ForestGreen}\blacklozenge_{\mathit{req}}}/\ell} \mid
%	{\color{ForestGreen}\blacklozenge_{\mathit{req}}}\in \mathit{REQ}, \ell\in L\},%\\
%	\quad
%	P_{REQ,RESP}=
%	\{\,p_{{\color{ForestGreen}\blacklozenge_{\mathit{req}}}/{\color{red}\blacklozenge_{\mathit{resp}}}} \mid
%	{\color{ForestGreen}\blacklozenge_{\mathit{req}}}\in \mathit{REQ}, {\color{red}\blacklozenge_{\mathit{resp}}}\in \mathit{RESP}\},
%	\]
%	\{\,p_{{\color{ForestGreen}\blacklozenge_{\mathit{req}}}/{\color{red}\blacklozenge_{\mathit{resp}}}} \mid
%	{\color{ForestGreen}\blacklozenge_{\mathit{req}}}\in \mathit{REQ}, {\color{red}\blacklozenge_{\mathit{resp}}}\in \mathit{RESP}\},
%	\]

%	\[
%	P_{REQ,RESP}=
%	\{\,p_{{\color{ForestGreen}\blacklozenge_{\mathit{req}}}/{\color{red}\blacklozenge_{\mathit{resp}}}} \mid
%	{\color{ForestGreen}\blacklozenge_{\mathit{req}}}\in \mathit{REQ}, {\color{red}\blacklozenge_{\mathit{resp}}}\in \mathit{RESP}\},
%	\]

with \(G\) being the set of global states, \(L\) being the set of local states (in the case of a \toolname-derived NS, this is the coupling of the local variable assignments of an in-flight request and its remaining \toolname{} program to execute), \(\mathit{REQ}\) denotes the request labels; and \(\mathit{RESP}\) denotes the response labels.

\medskip
Transitions are partitioned as:
\[
T = T_{\mathit{req}} \;\cup\; T_{\delta}\;\cup\;T_{\mathit{resp}}
\]
where

%	\[
%	T_{\mathit{req}} = \{\,t_{({\color{ForestGreen}\blacklozenge_{\mathit{req}}},\ell)} \mid {(\color{ForestGreen}\blacklozenge_{\mathit{req}}},\ell)\in\mathit{req}\},\quad
%	T_{\delta} = \{\,t_{(\ell,g)\to(\ell',g')} \mid (\ell,g)\xrightarrow{}(\ell',g')\in\delta\},\quad
%	T_{\mathit{resp}} = \{\,t_{(\ell,{\color{red}\blacklozenge_{\mathit{resp}}})} \mid (\ell,{\color{red}\blacklozenge_{\mathit{resp}}})\in\mathit{resp}\}.
%	\]

\begin{align*}
	T_{\mathit{req}}
	&= \{\,t_{({\color{ForestGreen}\blacklozenge_{\mathit{req}}},\ell)} \mid {(\color{ForestGreen}\blacklozenge_{\mathit{req}}},\ell)\in\mathit{req}\},\\[1ex]
	T_{\delta}
	&= \bigl\{\,t_{((\ell,g),(\ell',g'))} 
	\mid ((\ell,g),(\ell',g'))\in\delta\bigr\},\quad
	T_{\mathit{resp}}
	= \{\,t_{(\ell,{\color{red}\blacklozenge_{\mathit{resp}}})} \mid (\ell,{\color{red}\blacklozenge_{\mathit{resp}}})\in\mathit{resp}\}.
\end{align*}



%\smallskip
%Given a local state \(\ell\) which resulted from a request \({\color{ForestGreen}\blacklozenge_{\mathit{req}}}\) (either directly or downstream due to program execution) --- the transitions are:
Their \(\mathsf{pre}\) and \(\mathsf{post}\) flow functions are:
\[
\begin{alignedat}{3}
	\mathsf{pre}\bigl(t_{({\color{ForestGreen}\blacklozenge_{\mathit{req}}},\ell)}\bigr)
	&= \mathbf0, &
	\mathsf{post}\bigl(t_{({\color{ForestGreen}\blacklozenge_{\mathit{req}}},\ell)}\bigr)
	&= \mathbf1_{p_{({\color{ForestGreen}\blacklozenge_{\mathit{req}}},\ell)}}, 
	&&\text{for }({\color{ForestGreen}\blacklozenge_{\mathit{req}}},\ell)\in\mathit{req},\\
	\mathsf{pre}\bigl(t_{((\ell,g),(\ell',g'))}\bigr)
	&= \mathbf1_{p_{({\color{ForestGreen}\blacklozenge_{\mathit{req}}},\ell)}} + \mathbf1_{p_g}, &
	\mathsf{post}\bigl(t_{((\ell,g),(\ell',g'))}\bigr)
	&= \mathbf1_{p_{({\color{ForestGreen}\blacklozenge_{\mathit{req}}},\ell')}} + \mathbf1_{p_{g'}}, 
	&&\text{for }{{\color{ForestGreen}\blacklozenge_{\mathit{req}}}\in\mathit{REQ}}, ((\ell,g),(\ell',g'))\in\delta,\\
	\mathsf{pre}\bigl(t_{(\ell,{\color{red}\blacklozenge_{\mathit{resp}}})}\bigr)
	&= \mathbf1_{p_{({\color{ForestGreen}\blacklozenge_{\mathit{req}}},\ell)}}, &
	\mathsf{post}\bigl(t_{(\ell,{\color{red}\blacklozenge_{\mathit{resp}}})}\bigr)
	&= \mathbf1_{p_{({\color{ForestGreen}\blacklozenge_{\mathit{req}}}/{\color{red}\blacklozenge_{\mathit{resp}}})}}, 
	&&\text{for }{{\color{ForestGreen}\blacklozenge_{\mathit{req}}}\in\mathit{REQ},(\ell,\color{red}\blacklozenge_{\mathit{resp}}})\in\mathit{resp}
	%\ %(\ell\text{ the matching local state).
		%		}
\end{alignedat}
\]

Where for the last two cases, \({\color{ForestGreen}\blacklozenge_{\mathit{req}}}\) concerns requests that eventually give rise to a local state \(\ell \in L\) that originated downstream (during execution).

\medskip
The initial marking is a single token on the single place representing the initial global state $g_0$ of the NS:
\[
M_0(p_{g_0}) = 1,
\quad
M_0(p) = 0 \text{ for all }p\neq p_{g_0},
\]
%	where \(g_0\) is the initial global state of the network system \(\mathcal S\).  



Define the projection \(\pi\) to solely include the markings of places representing completed request/response pairs.
% (with the exception of a single token on a state in \(P_G\)).
%\[
%\pi \;:\;\mathbb N^P \;\longrightarrow\;\mathbb N^{P_R}
%\quad\bigl(\pi(M)\bigr)(p_{({{\color{ForestGreen}\blacklozenge_{\mathit{req}}}/{\color{red}\blacklozenge_{\mathit{resp}}}})})\;=\;M(p_{({{\color{ForestGreen}\blacklozenge_{\mathit{req}}}/{\color{red}\blacklozenge_{\mathit{resp}}}})})\text{ for }p_{({{\color{ForestGreen}\blacklozenge_{\mathit{req}}}/{\color{red}\blacklozenge_{\mathit{resp}}}})}\in P_{REQ,RESP}.
%\]
Then, the multiset of all  (${{\color{ForestGreen}\blacklozenge_{\mathit{req}}}/{\color{red}\blacklozenge_{\mathit{resp}}}}$) pairs of the NS, obtained by \textit{any} interleaving, is:
\[
\mathsf{Int}(\mathcal S)
\;=\;
\bigl\{\;\pi(M)\;\bigm|\;M_0 \xrightarrow{}^{*} M\text{ in }N_{\mathrm{int}}(\mathcal S)\bigr\}.
\]