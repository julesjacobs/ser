\section{Related Work}
\label{sec:relatedWork}

Serializability was first formalized by Eswaran et al.~\cite{EsGrKoTr76} as a correctness condition for concurrent transaction execution. Their work also introduced conflict serializability, a stricter variant that requires equivalence to a serial schedule solely by reordering non-conflicting operations. Papadimitriou~\cite{Pa79} later proved that determining serializability even for a given, single interleaving is NP-hard. 
%
Moreover, although conflict serializability is more conservative measure than serializability, it is easier to enforce during runtime by various approaches. 
%
These approaches are typically categorized as either \textit{pessimistic} locking approaches, e.g, Two-Phase Locking~\cite{BeHaGo87}, or alternatively --- \textit{optimistic} locking approaches, e.g., Optimistic Concurrency Control (OCC)~\cite{KuRo81, BuMo06}.
%
Both approaches ensure acyclicity in the conflict graph --- a necessary and sufficient condition for conflict serializability. However, because these approaches \textit{ignore program semantics}, these may incorrectly reject executions that, although not conflict-serializable, are still valid serializable executions, i.e., have the same result as a serial execution.
%

Alur et al.~\cite{AlMcPe96} established the complexity of deciding conflict serializability for bounded transaction systems. Bouajjani et al.~\cite{BoEmEnHa13} later extended this result to unbounded systems, proving the problem remains decidable and EXPTIME-complete. Their key insight reveals that while the conflict graph becomes infinite, cycle detection, and thus conflict serializability, is independent of transaction count. By modeling transactions via Vector Addition Systems (equivalent to Petri Nets), they provide a finite framework for analyzing infinite behaviors. This approach inspired our use of Petri Nets to capture Int(S). 

A separate research direction attempts to \textit{directly} verify serializability, without limiting it to conflict restrictions, as done in other works. Towards this end, the expressive Temporal Logic of Actions (TLA)~\cite{La94} is used to encode a formal specification that validates whether only serializable executions occur. While TLA can naturally encode ``real'' serializability (based on final-state equivalence), existing TLA-based approaches~\cite{SoVaVi20, Ho24} remain limited to bounded transaction systems. This limitation stems from TLA/TLA+ model checkers like TLC and Apalache~\cite{YuMaLa99, KoKuTr19}, which require finite-state verification and cannot handle unbounded transaction counts.

%
While these contributions represent significant advances, to our knowledge, our work is the first to:
(i) Decide serializability universally --- \textit{considering all executions} purely through program semantics and final states, independent of read/write conflicts; 
(ii) Support \textit{unbounded} transaction systems; and
(iii) Provide a complete end-to-end implementation.


Several works have proposed relaxations of the (strong) consistency notion of serializability guarantees. Rastogi et al.~\cite{RaMeBrKoSi93} introduced predicate-wise serializability (PWSR), which preserves database invariants while permitting non-atomic transactions. 
%
Other relaxations focus on weaker consistency models: Lamport’s causal consistency~\cite{La78}, later generalized to shared memory as causal memory~\cite{AhNeBuKoHu95} and implemented in systems like COPS~\cite{LlFrKaAn11}, has inspired extensive research on model checking and complexity analysis~\cite{BoEnGuHa17,ZeBiBoEnEr19,LaBo20}. 
%
Recent work by Brutschy et al.~\cite{BrDiMuVe18} further bridges these concepts by statically detecting non-serializable behaviors in causally consistent databases.


In addition, our work builds on both theoretical and practical advances in Petri net research~\cite{Mu89, Es96, Re12, EsNi24}. The undecidability we prove for equivalence of interleavings stems from Hack’s seminal result~\cite{Ha76, HaThesis76} showing the undecidability of reachability set equivalence for Petri Nets \textcolor{red}{--- a result inspired by similar (unpublished) work by M. Rabin}.\guy{remove part in red?} This undecidability originates in a series of reductions from Hilbert’s 10th problem, specifically the possibility of determining whether there exists an integer root for Diophantine equations, a problem that was later proven undecidable by Matijasēvič~\cite{Ma70}.
%
Jančar~\cite{Ja95} later provided an alternative proof to this undecidability result, by showing that Petri nets can simulate universal (and thus undecidable) 2-counter Minsky machines~\cite{Mi67}. In addition, Jančar further strengthened the original result by proving that undecidability holds even for Petri nets with just five unbounded places.

Furthermore, our approach also builds on Petri net reachability algorithms, which determine whether a given marking is attainable. While the solution is straightforward for bounded nets (through exhaustive enumeration), the solution for the unbounded case is highly nontrivial, and was first solved by Mayr~\cite{Ma81}, with subsequent improvements by Kosaraju~\cite{Ko82} and Lambert~\cite{La92}. Recent work~\cite{CzWo22} has also established this problem is Ackermann-complete, implying that, although decidable, it is practically infeasible to solve on large nets.

These theoretical advances in Petri Net reachability have given rise to a plethora of practical tools, including K-Reach~\cite{DiLa20}, DICER~\cite{XiZhLi21}, MARCIE~\cite{HeRoSc13}, and others. 
%
Specifically, our implementation leverages SMPT~\cite{AmDa23}, a state-of-the-art Petri Net reachability tool that combines SMT-solving with structural invariants~\cite{AmBeDa21, AmDaHu22}. At a high level, SMPT formulates reachability as satisfiability queries (dispatched to the Z3 solver~\cite{DeBj08}) while curtailing the search space by proactively inferring invariants on the net's structure.
%
%We refer the reader to a survey by Esparza and Nielsen~\cite{EsNi94} (recently republished in~\cite{EsNi24}) for a comprehensive summary of additional decidability results pertaining to Petri Nets.


 
 
\todo{decide about 1992 paper: Modeling Serializability via Process Equivalence in Petri Nets}







%Serializability first introduced by Eswaran et al.~\cite{EsGrKoTr76}. It is the first to put forth serializability as a correctness condition for concurrent transaction execution.
%The paper also covers conflict serializability --- a strictly stronger consistency property than serializability, that does not only require that the final state of the system be attained by an equivalent serial execution, but also, that this equivalence be attained by allowing only specific (``non-conflicting'') operations to be reordered.
%%
%Papadimitriou~\cite{Pa79} proved that it is NP-hard to decide whether even the history of a single interleaving is serializable. 
%%
%Moreover, although conflict serializability is more conservative measure than serializability, it is easier to enforce during runtime by various approaches. 
%%
%These approaches are typically categorized as either \textit{pessimistic} locking approaches, e.g, 2-Phase Locking~\cite{BeHaGo87}, or alternatively --- \textit{optimistic} locking approaches, e.g., Optimistic Concurrency Control (OCC)~\cite{KuRo81, BuMo06}.
%%
%
%Furthermore, most work, both in theory and in practice, focuses on proving theorems on conflict serializability, due to is being more straightforward, and corresponding to the programs dependency graph, and \textit{without taking the actual semantics into account}.
%%
%Alur et al.~\cite{AlMcPe96} cover the complexity for deciding conflict serializability, given a bound on the number of transactions. 
%%
%This work was later continued by Bouajjaniet al.~\cite{BoEmEnHa13}, which demonstrate that the problem of deciding whether a program with an \textit{unbounded} number of transaction is conflict serializable, is also decidable and is EXPTIME-complete. The authors show that although the conflict graph in such a case is infinite (and hence, infeasible to traverse) --- conflict serializability can still be decided as the size of the cycle (if it exists), surprisingly, does not depend on the number of transactions. The authors also emulate multiple transactions in a shared memory system with a Vector addition system, and equivalent object to a Petri Net. We took inspiration by defining a Petri Net to capture Int(S). 
%
%In another line of work, there is an attempt to \textit{directly} validate (regular, non-conflicting) serializability by encoding this specification in the highly expressive \textit{Logic of Temporal Actions} (TLA)~\cite{La94}. 
%%
%Although, unlike the aforementioned works, TLA itself allows encoding serializability in the original form (focusing on the final state of the variables), such works~\cite{SoVaVi20, Ho24} cannot validate this behavior for an \textit{unbounded} number of transactions. This is because, although TLA/TLA+ allow encoding the properties of interest, their model checkers (such as TLC and Apalache)~\cite{YuMaLa99, KoKuTr19} can only operate on a finite and predefined number of transactions.
%%
%Although these works present important progress, as far as we are aware, our work is the first to decide serializability for all executions, based solely on the program semantics and final state, regardless of read/write conflicts. Furthermore, ours is the first to handle the unbounded case; and to supply an actual end-to-end implementation.
%
%
%Other work relaxes the (strong) consistency notion of serializability and allows weaker consistency notions. For example, Rastogi et al.~\cite{RaMeBrKoSi93}
%introduce \textit{predicate-wise serializability} (PDSR) --- a relaxation of serializability in which transactions might not be atomic, but are still required to maintain some desired database consistency predicate
%%
%Furthermore, other relaxations focus on weaker consistency models. One such model is causal consistency, which was put forth by Lamport~\cite{La78}, en extended to shared memory systems as \textit{causal memory}~\cite{AhNeBuKoHu95}. (include causal + consistency, designed in COPS~\cite{LlFrKaAn11}). The have been a plethora of works on model checking systems that adhere to causal consistency, and hence the complexity of such procedures~\cite{BoEnGuHa17,ZeBiBoEnEr19,LaBo20}.
%%
%We also note that some work combine various consistency notions. These include the recent work by Brutschy et al.~\cite{BrDiMuVe18}, who put form a method to statically detect non-serializable executions on top of
%causally-consistent databases.
%
%
%
%
%Our work also builds upon both theoretical literature, as well as practical results, pertaining to Petri Nets~\cite{Mu89, Es96, Re12}.
%%
%Firstly, our undecidability result is based on a classic result by Hack~\cite{Ha76, HaThesis76}, showing that, given two Petri Nets, it is undecidable to answer whether they have equivalent reachability sets. Hack based his result on the work of Rabin (which was never published). These undecidability results follow from a series of reductions, originating from Hilbert's 10th problem, i.e., deciding if a Diophantine polynomial has an integer root (a problem that was proved undecidable by Matijas{\'e}vi{\v{c}}~\cite{Ma70}).
%%
%Later, Jan{\v{c}}ar~\cite{Ja95} proved this result by demonstrating that Petri Nets can simulate 2-counter Minsky Machines~\cite{Mi67}, which are universally computable and hence undecidable. Moreover, Jan{\v{c}}ar strengthened the original result and proved that reachability equivalence is undecidable even for Petri Nets with five unbounded places~\cite{Ja95}.
%%
%
%Our decision procedure itself is based on an algorithm for deciding whether a given marking is reachable, for a Petri Net.
%%
%Mayr~\cite{Ma81} was the first to put forth an algorithm for this problem given a (potentially, unbounded) Petri Net (note that for a bounded case this is straightforward, as you can enumerate all reachable markings.)
%%
%Mayr's reachability algorithm was later improved and simplified by Kosaraju~\cite{Ko82}, and then again by Lambert~\cite{La92}.
%%
%Very recently, this problem was also proven to be Ackermann complete~\cite{CzWo22}, implying that, although decidable, it is practically infeasible to solve on large nets.
%%
%Furthermore, these theoretical algorithms have inspired various tools, such as K-Reach~\cite{DiLa20}, DICER~\cite{XiZhLi21}, MARCIE~\cite{HeRoSc13}, and others. 
%%
%Specifically, our tool employs SMPT~\cite{AmDa23}, a state-of-the-art Petri Net reachability tool, which employs an SMT-based approach~\cite{AmBeDa21, AmDaHu22}. SMPT curtails the search space by reducing the reachability problem to a satisfiability query (that is subsequently dispatched to the Z3 solver~\cite{DeBj08}) and inferring invariants on the net's structure.
%%
%We refer the reader to a survey by Esparza and Nielsen~\cite{EsNi94} (recently republished in~\cite{EsNi24}) for a comprehensive summary on additional decidability results pertaining to Petri Nets.

